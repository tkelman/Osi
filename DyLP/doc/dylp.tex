\documentclass[titlepage]{article}
\usepackage{loubookman}
\usepackage{loustandard}
\usepackage{graphics}
\usepackage{amsmath}
\usepackage{loumath}
\usepackage{codedocn}

%\newcommand{\mypath}{/cs/mitacs1/lou/Bonsai/Doc/Dylp.Coin090412}
\input{dylpabsdir}
\newcommand{\figures}{\mypath/Figures}

\newsavebox{\tmpbox}

\newcommand{\bonsai}{\textbf{bonsai}\xspace}
\newcommand{\glpk}{GLPK\xspace}
\newcommand{\consys}{\textsc{consys}\xspace}
\newcommand{\dylp}{\textsc{dylp}\xspace}
\newcommand{\bonsaiG}{\textbf{bonsaiG}\xspace}
\newcommand{\coin}{\textsc{Coin-OR}\xspace}


\newcommand{\netlib}{Netlib\xspace}
\newcommand{\miplib}{MIPLIB\xspace}

\renewcommand{\textfraction}{.1}
\renewcommand{\topfraction}{.9}
\renewcommand{\bottomfraction}{\topfraction}


\title{\bfseries \dylp: a dynamic LP code}
\author{Lou Hafer}
\date{December, 2005}


\begin{document}

\maketitle

\thispagestyle{empty}
\vspace*{\fill}

\noindent
Copyright (C) 2005, 2006, 2007, 2008, 2009 Lou Hafer

\noindent
This document is also available as SFU-CMPT TR 2005-18.

\noindent
All rights reserved. This documentation is made available under the terms of
the Common Public License v1.0 which accompanies this distribution.
A copy of the CPL v1.0 can also be obtained from the URL
\texttt{http://www.ibm.com/developerworks/library/os-cpl.html}

\begin{abstract}\noindent
\dylp is a full implementation of the dynamic simplex algorithm for
linear programming.
Dynamic simplex attempts to maintain a reduced active constraint system
by regularly
purging loose constraints and variables with unfavourable reduced costs, and
adding violated constraints and variables with favourable reduced costs.
In abstract, the code alternates between primal and dual simplex algorithms,
using dual simplex to reoptimise after updating the constraint set and primal
simplex to reoptimise after updating the variable set.
\end{abstract}

\section{Introduction}
\label{sec:Intro}

\dylp is a linear programming (LP) code designed to be used as the
underlying LP code in a branch-and-cut integer linear programming (IP) code.
It emphasises convenience of use by the client, particularly with
respect to fixing variables and adding and deleting constraints and
variables.
The target user population is IP algorithm developers; as such, \dylp
emphasises controllability and convenience over efficiency and is capable of
producing copious amounts of output for use in debugging.

\dylp implements a dynamic simplex algorithm
along the lines set out by Padberg in \cite[\S6.6]{Pad95}.
The core idea is that, at any given time, many of the constraints of a LP
problem are loose, and many nonbasic variables are unlikely to ever be
considered for pivoting because their reduced costs are very unfavourable.
A rough outline of the algorithm, neglecting unboundedness,
infeasibility, and implementation issues, is as follows.

From the problem supplied by the client, \dylp chooses an initial subset of
constraints and variables to become the active system.
This system is solved to optimality with primal simplex.
\dylp then enters a minor loop where it deactivates variables whose
reduced costs are worse than a threshold, 
activates variables whose reduced costs are favourable,
and reoptimises the system with primal simplex.
This minor loop is repeated until there are no more variables suitable for
entry.
Next, \dylp deactivates any loose constraints, activates
any constraints which are violated at the current basic solution,
and reoptimises with dual simplex.
On regaining feasibility, it returns to primal simplex and the
deactivate/activate variable loop.
When there are no variables with favourable reduced costs
among the inactive variables and no violated constraints
among the inactive constraints, the solution is optimal.

The primal simplex algorithm used by \dylp is a two-phase algorithm.
Phase I uses a dynamically modified objective to attain a primal feasible
solution.
Both phase I and phase II use a projected steepest edge (PSE) pricing
algorithm outlined by Forrest \& Goldfarb \cite[algorithm `dynamic']{For92}.
There are two antidegeneracy methods.
The first, referred to as `anti-degeneracy lite', attempts to resolve ties
among degenerate pivots by choosing the pivot in such a way as to make tight
a hyperplane which has a desirable alignment.
The second, applied when the first takes too long to resolve the degeneracy,
is a perturbation algorithm which
builds on a method described by Ryan \& Osborne \cite{Rya88}.

The dual simplex algorithm provides only a second phase with
dual steepest edge (DSE) pricing \cite[algorithm `steepest 1']{For92},
standard or generalised pivoting,
and implementations of anti-degeneracy lite and perturbation-based
antidegeneracy in the dual space.
In the context of \dylp, it is the subordinate simplex, used for
reoptimisation after adding constraints and as the initial simplex when the
problem is dual feasible but not primal feasible.

The active and inactive constraint systems are maintained with the \consys
subroutine library \cite{Haf98b}.
Basis factoring and pivoting are handled using the basis maintenance package
from \glpk \cite{GLPK,Mak01}.

\dylp is written in C and provides a native C interface.
It can be used as a standalone simplex LP code with only a minimal shell
required to generate the constraint system.

In the context of a branch-and-cut code, \dylp expects that the dominant
mode of use will be successive calls to
reoptimise a constraint system that is incrementally modified between calls.
On request, it will maintain its internal state (constraint system,
basis inverse, and support data structures) between calls to support
efficient hot starts for reoptimisation.
Because \dylp maintains this internal state, it does not provide a native
capability to interleave optimisation and reoptimisation of distinct
constraint systems.
\dylp provides two specialised interface routines to support two queries
commonly required in
a branch-and-cut context, pricing a new variable and pricing a dual pivot.

\dylp can be used with \coin \cite{COIN} software through
the C++ \coderef{}{OsiDylp} OSI interface class.
An OSI interface object maintains a copy of the constraint system as
well as providing an interface to the underlying solver.
Multiple \pgmid{OsiDylp} objects with distinct constraint systems can exist
simultaneously and calls to optimise and reoptimise the systems can be
interleaved.
There is some loss of efficiency as the state of the underlying solver is
changed, but the necessary bookkeeping is handled by the \pgmid{OsiDylp}
objects.

The next section specifies the notation used for the
primal and dual problems in the remainder of the report.
Sections \ref{sec:UpdatingFormulas} through \ref{sec:Startup}
describe individual components of the implementation.
Sections \ref{sec:DynamicSimplex} through \ref{sec:ConstraintManagement}
describe the simplex algorithms and the variable and constraint management
algorithms used in \dylp.
%Section \ref{sec:Performance} gives performance data for \dylp on the
%\netlib and \miplib problem suites.
Sections \ref{sec:DylpInterface} through \ref{sec:DylpDebugging} describe
the interface and parameters provided by \dylp.


\section{Notation}
\label{sec:Notation}

\dylp works naturally with the minimisation problem
\begin{equation}
\begin{split}
\min \enspace  cx & \\
      Ax & \leq b \\
l \leq x & \leq u
\end{split} \label{Eqn:BoundedPrimal}
\end{equation}
Add slack variables $s$ and partition $\begin{bmatrix} A & I \end{bmatrix}$
into basic and nonbasic portions as
\begin{equation*}
\begin{split}
\begin{bmatrix} B & N \end{bmatrix} =
\left[
\begin{array}{cc|cc}
B^t & 0 & N^t & I^t \\
B^l & I^l & N^l & 0
\end{array}
\right]
\end{split}
\end{equation*}
with corresponding partitions
\begin{math}
\trans{\begin{bmatrix} x^B & s^B & x^N & s^N \end{bmatrix}}
\end{math}
for $x$, $s$, and
\begin{math}
\trans{\begin{bmatrix} b^t & b^l\end{bmatrix}}
\end{math}
for $b$.
The objective $c$ is augmented with 0's in the columns corresponding to the
slack variables, and partitioned as
\begin{math}
\begin{bmatrix} c^B & 0 & c^N & 0 \end{bmatrix}
\end{math}.
The basis inverse will be 
\begin{equation}
\inv{B} = \begin{bmatrix}
	     \inv[0]{(B^t)} & 0 \\
	     -B^l\inv[0]{(B^t)} & I^l
	    \end{bmatrix}
	    \label{Eqn:PrimalBasisInverse}
\end{equation}
We then have
\begin{equation}
\begin{split}
\begin{bmatrix} x^B \\ s^B \end{bmatrix} & = \,
\inv{B}b - \inv{B} N \begin{bmatrix} x^N \\ s^N \end{bmatrix} \\
%
& = \begin{bmatrix}
      \inv[0]{(B^t)} b^t \\ b^l - B^l\inv[0]{(B^t)} b^t
    \end{bmatrix} -
    \begin{bmatrix}
      \inv[0]{(B^t)} N^t & \inv[0]{(B^t)} \\
      N^l - B^l\inv[0]{(B^t)}N^t & -B^l\inv[0]{(B^t)}
    \end{bmatrix}
    \begin{bmatrix} x^N \\ s^N \end{bmatrix}
\end{split} \label{Eqn:PrimalBasicVars}
\end{equation}
and
\begin{equation}
\begin{split}
z & = \begin{bmatrix} c^B & 0 \end{bmatrix}
      \trans{\begin{bmatrix} x^B & s^B \end{bmatrix}} +
      \begin{bmatrix} c^N & 0 \end{bmatrix}
      \trans{\begin{bmatrix} x^N & s^N \end{bmatrix}} \\
  & = \begin{bmatrix} c^B & 0 \end{bmatrix}\inv{B}b +
      \left(
        \begin{bmatrix} c^N & 0 \end{bmatrix} -
        \begin{bmatrix} c^B & 0 \end{bmatrix} \inv{B} N
      \right)
      \trans{\begin{bmatrix} x^N & s^N \end{bmatrix}} \\
  & = c^B \inv[0]{(B^t)} b^t +
      \begin{bmatrix}
	c^N - c^B \inv[0]{(B^t)} N^t & -c^B\inv[0]{(B^t)} \end{bmatrix}
      \trans{\begin{bmatrix} x^N & s^N \end{bmatrix}}
\end{split} \label{Eqn:PrimalObj}
\end{equation}
The quantities
$\trans{\begin{bmatrix} x^B & s^B \end{bmatrix}} =
\overline{b} = \inv{B}b$ are the values of the basic
variables, the quantities
$y = \begin{bmatrix} c^B & 0 \end{bmatrix}\inv{B}$ are the dual
variables, and the quantities
$\overline{c} = \left(
		  \begin{bmatrix} c^N & 0 \end{bmatrix} -
		  \begin{bmatrix} c^B & 0 \end{bmatrix} \inv{B} N
		\right)$
are the reduced costs.
A row or column of $\inv{B}N$ (as appropriate to the context) will be
denoted $\overline{a}_k$ (the single subscript distinguishes it from an
individual element $\overline{a}_{ij}$).
A row or column of $\inv{B}$ (as appropriate to the context) will be
denoted $\beta_k$.
When discussing pivot selection calculations, $\Delta_j$ will be the
change in nonbasic variable $x_j$ or $s_j$.

The dual problem is formed by first converting \eqref{Eqn:BoundedPrimal} to
$\max\: -cx$, giving
\begin{equation*}
\begin{split}
\min \enspace & yb \\
        & y A \geq -c \\
        & y \geq 0
\end{split}
\end{equation*}
Add surplus variables $\sigma$ and partition
$\trans{\begin{bmatrix}A & -I \end{bmatrix}}$
into basic and nonbasic portions as
\begin{equation*}
\addtolength{\extrarowheight}{3pt}
\begin{bmatrix} \mathcal{B} \\ \mathcal{N} \end{bmatrix} = 
\left[
\begin{array}{cc}
0  & -I^\mathcal{B} \\
B^t & N^t \\
\hline
-I^\mathcal{N} & 0 \\
B^l & N^l
\end{array}
\right]
\end{equation*}
with corresponding partitions
\begin{math}
\begin{bmatrix}
  \sigma^\mathcal{B} & y^\mathcal{B} & \sigma^\mathcal{N} & y^\mathcal{N}
\end{bmatrix}
\end{math}
for $y$, $\sigma$, and
\begin{math}
\begin{bmatrix} c^B & c^N \end{bmatrix}
\end{math}
for $c$.
The right-hand side $b$ is augmented with 0's in the rows corresponding to the
surplus variables and partitioned as
\begin{math}
\trans{\begin{bmatrix} 0 & b^t & 0 & b^l \end{bmatrix}}
\end{math}.
The basis inverse will be 
\begin{equation*}
\inv[-2]{\mathcal{B}} =
\begin{bmatrix}
  \inv[0]{(B^t)} N^t & \inv[0]{(B^t)} \\
  -I^\mathcal{B} & 0
\end{bmatrix}.
\end{equation*}
We then have
\begin{equation}
\begin{split}
\begin{bmatrix} \sigma^\mathcal{B} & y^\mathcal{B} \end{bmatrix} & =
    (-c)\inv[-2]{\mathcal{B}} -
    \begin{bmatrix} \sigma^\mathcal{N} & y^\mathcal{N} \end{bmatrix}
    \mathcal{N}\inv[-2]{\mathcal{B}} \\
& = \begin{bmatrix}
      c^N - c^B \inv[0]{(B^t)} N^t & -c^B \inv[0]{(B^t)}
    \end{bmatrix} -
    \begin{bmatrix} \sigma^\mathcal{N} & y^\mathcal{N} \end{bmatrix}
    \begin{bmatrix}
      -\inv[0]{(B^t)} N^t & -\inv[0]{(B^t)} \\
      B^l\inv[0]{(B^t)} N^t - N^l & B^l\inv[0]{(B^t)}
    \end{bmatrix} \label{Eqn:DualBasicVars}
\end{split}
\end{equation}
and
\begin{equation}
\begin{split}
z & = \begin{bmatrix} \sigma^\mathcal{B} & y^\mathcal{B} \end{bmatrix}
      \trans{\begin{bmatrix} 0 & b^t \end{bmatrix}} +
      \begin{bmatrix} \sigma^\mathcal{N} & y^\mathcal{N} \end{bmatrix}
      \trans{\begin{bmatrix} 0 & b^l \end{bmatrix}} \\
  & = (-c)\inv[-2]{\mathcal{B}} b^\mathcal{B} +
      \begin{bmatrix} \sigma^\mathcal{N} & y^\mathcal{N} \end{bmatrix}
      (b^\mathcal{N} - \mathcal{N}\inv[-2]{\mathcal{B}} b^\mathcal{B}) \\
  & = -c^B \inv[0]{(B^t)} b^t +
      \begin{bmatrix} \sigma^\mathcal{N} & y^\mathcal{N} \end{bmatrix}
      \begin{bmatrix}
	\inv[0]{(B^t)} b^t \\ b^l - B^l\inv[0]{(B^t)} b^t
      \end{bmatrix}
\end{split} \label{Eqn:DualObj}
\end{equation}
When discussing pivot selection calculations, $\delta_j$ will be the
change in nonbasic dual variable $y_j$ or $\sigma_j$.

There are several points to note about the relationship between primal and dual
simplex in the \dylp implementation.

First, \dylp does not solve $\max \; -cx$ as a surrogate for $\min \, cx$.
It minimises $cx$ directly by algorithmic design.
Hence the dual variables $y = c^B \inv{B}$ have the wrong sign for the dual
problem, and are calculated solely as a convenience.
The dual algorithm actually works with the reduced costs
$\overline{c}^N = c^N - c^B \inv{B} N$, which are the correct dual variable
values
(compare \eqnref{Eqn:PrimalObj} with \eqnref{Eqn:DualBasicVars}).

Second, because primal simplex provides
$\inv{B} N = - \mathcal{N} \inv[-2]{\mathcal{B}}$
(compare \eqnref{Eqn:PrimalBasicVars} with \eqnref{Eqn:DualBasicVars}),
the relevant calculation when determining the leaving dual variable is
$\overline{c}_k + \overline{a}_{ik} \delta_i$, rather than
$ \overline{c}_k - \overline{a}_{ik} \delta_i$.

Throughout the remainder of the report, let $e_k \in R^d$ be a row or
column vector of appropriate
dimension (as determined by the context), with a 1 in
position $k$ and 0's in all other positions.

\section{Updating Formul\ae}
\label{sec:UpdatingFormulas}

For purposes of the updating formul\ae, the distinction between original
variables $x$ and slack variables $s$ is not important.
For simplicity, $x_k$ is used to represent both original variables and slack
variables in this section.
In the same vein, $c^B$ and $c^N$ will denote
$\begin{bmatrix} c^B & 0 \end{bmatrix}$ and
$\begin{bmatrix} c^N & 0 \end{bmatrix}$, respectively.

\subsection{Basis Updates}
\label{sec:BasisUpdates}

While these formul\ae{} are not applied directly to update the basis, they are
useful in deriving update formul\ae{} for other values.

Suppose that $x_i$ will leave basis position $k$ and be replaced by $x_j$.
The new basis $B'$ can be expressed as $B' = B - a_i e_k + a_j e_k$.
Premultiplying by $B^{\,-1}$ and postmultiplying by $(B')^{\,-1}$, we have
\begin{align}
\begin{split}\label{Eqn:BasisUpdate}
B^{\,-1}B'(B')^{\,-1} & = B^{\,-1}B(B')^{\,-1} -
			  B^{\,-1}a_i e_k(B')^{\,-1} +
			  B^{\,-1}a_j e_k(B')^{\,-1} \\
B^{\,-1} & = (B')^{\,-1} - \overline{a}_i \beta'_k + \overline{a}_j \beta'_k \\
(B')^{\,-1} & = B^{\,-1} + \overline{a}_i \beta'_k - \overline{a}_j \beta'_k \\
\end{split}
\end{align}
Since $x_i$ was basic, $\overline{a}_i = e_k$. This gives
\begin{equation*}
(B')^{\,-1} = B^{\,-1} + e_k\beta'_k - \overline{a}_{j} \beta'_k.
\end{equation*}
Premultiplying by $e_l$ to obtain an update formula for row $l$, we have
\begin{align}
\begin{alignedat}{2}
\beta'_l & = \beta_l - \frac{\overline{a}_{lj}}{\overline{a}_{kj}}\beta_k &
	\qquad l \neq k & \\
\beta'_k & = \frac{1}{\overline{a}_{kj}}\beta_k 
\end{alignedat} \label{Eqn:betaupdate}
\end{align}

\subsection{Primal Variable Updates}

Updating the primal variables is straightforward and follows directly from
\eqref{Eqn:PrimalBasicVars}.

Both primal and dual pivots calculate the change in the entering primal
variable, $\Delta_j$.
The entering variable $x_j$ is set to $u_j+\Delta_j$ or
$l_j+\Delta_j$, for $x_j$ entering from its upper or lower bound,
respectively.
The leaving variable $x_i$ is set to $u_i$ or $l_i$, for
$x_i$ leaving at its upper or lower bound, respectively.
The remaining basic variables $x_k$, $k \neq i$, are updated according to
the formula
\begin{equation*}
x_k = \overline{b}_k - \overline{a}_{kj}\Delta_j.
\end{equation*}

\subsection{Dual Variable Updates}
\label{sec:DualUpdates}

Updating the dual variables is simple in the final implementation, but a little
work is necessary to derive the updating formula.
The difficulty lies in the fact that the dual variables of interest are
$y = \begin{bmatrix} y^\mathcal{B} & y^\mathcal{N} \end{bmatrix}$, \ie, a
mixture of basic and nonbasic dual variables.
Direct application of \eqref{Eqn:DualBasicVars} is not possible.

Assume that the leaving variable $x_i$ occupies row $k$ in the basis $B$.
The new vector of basic costs, $(c')^B$, can be expressed as
$(c')^B = c^B - [0 \ldots c_i \ldots 0] + [0 \ldots c_j \ldots 0]$, where
$c_i$ and $c_j$ occur in the $k$\textsuperscript{th} position.
From \eqnref{Eqn:BasisUpdate}, it is easy to show
$B(B')^{-1} = I + a_i(\beta')_k - a_j(\beta')_k$.

We can proceed to derive the update formul\ae{} for $y$ as follows:
\begin{align*}
y' & = (c')^B (B')^{-1} \\
   & = c^B B^{\,-1} B (B')^{-1} - c_i(\beta')_k + c_j(\beta')_k \\
   & = y(I + a_i(\beta')_k - a_j(\beta')_k) -
       c_i(\beta')_k + c_j(\beta')_k \\
   & = y + (c_j - y a_j)(\beta')_k - (c_i - y a_i)(\beta')_k . \\
%
\intertext{Recognising that $\overline{c}_j = c_j - y a_j$ is the reduced
cost of $x_j$ before the basis change, and noting that
$\overline{c}_i = c_i - y a_i = 0$ since $x_i$ was basic, we have}
%
y' & = y + \overline{c}_j(\beta')_k . \\
%
\intertext{As a further observation, note that
$(\beta')_k = \beta_k/\overline{a}_{kj}$, so we can update $y$ using
a row of $B^{\,-1}$ as}
%
y' & = y + \overline{c}_j\beta_k/\overline{a}_{kj}.
\end{align*}


\section{Pricing Algorithms}
\label{sec:PricingAlgorithms}

\subsection{Projected Steepest Edge Pricing}
\label{sec:PSEPricing}

The primal simplex algorithm in \dylp uses projected steepest edge (PSE)
pricing;
the algorithm used is described as dynamic projected steepest edge
(`dynamic') in Forrest and Goldfarb \cite{For92}.

To understand the operation of projected steepest edge (PSE) pricing, it
will be helpful to start with the definition of a direction of motion.
The values of the basic and nonbasic variables can be expressed as
\begin{equation} \label{eqn:allPrimalDirs}
\begin{bmatrix} x^B \\ x^N \end{bmatrix} =
  \begin{bmatrix} b \\ l/u \end{bmatrix} -
  \begin{bmatrix} \inv{B} A^N \\ -I \end{bmatrix} \Delta
\end{equation}
where $l/u$ is intended to indicate use of the lower or upper bound as
appropriate for the particular nonbasic variable.
When a given nonbasic variable $x_j$ is moved by an amount $\Delta_j$, the
values of $x$ will change as
\begin{equation} \label{eqn:onePrimalDir}
  -\begin{bmatrix} \inv{B} a_j \\ -e_j \end{bmatrix} \Delta_j =
  -\begin{bmatrix} \overline{a}_j \\ -e_j \end{bmatrix} \Delta_j =
  \eta_j\Delta_j
\end{equation}
The vector $\eta_j$ is the direction of motion as $x_j$ is changed;
alternatively, it is the edge of the polyhedron which is traversed as $x_j$ is
changed.
Let $\gamma_j = \norm{\eta_j}$ be the norm of $\eta_j$.

For pricing, it can be immediately seen that
$c\eta_j = c_j - c^B \, \overline{a}_j$ is the reduced cost $\overline{c}_j$.
Dantzig pricing chooses an entering variable $x_j$ such
that $\overline{c}_j$ has appropriate sign and the largest magnitude over all
reduced costs, but it can be misled by differences in scaling from
one column to the next.
Steepest edge (SE) pricing scales $\overline{c}_j$ by $\gamma_j$, choosing
an entering variable $x_j$ with $\overline{c}_j$ of appropriate sign
and the largest $\displaystyle \abs{\frac{c \eta_j}{\norm{\eta_j}}}$,
effectively
calculating the change in objective value over a unit vector in the direction
of motion.
This gives a uniform pricing comparison, using the slope of the edge.

Projected steepest edge (PSE) pricing uses `projected' column
norms which are calculated using a vector $\tilde{\eta}_j$ which contains only
the components of $\eta_j$ included in a reference frame.
Initially, this reference frame contains only the nonbasic variables, so that
$\tilde{\gamma}_j = 1$ for all $x_j \in x^N$.
In order to avoid calculating $\tilde{\gamma}_j$ from scratch each time a
column must be priced, the norms are iteratively updated.

To derive the update formul\ae{} for $\tilde{\gamma}_j$, it is useful to start
with the update formul\ae{} for the full vector $\eta_j$.
As mentioned in \secref{sec:DualUpdates},
for $x_i$ leaving basis position $k$ and
$x_j$ entering, $B\inv[0]{(B')} = I + a_i(\beta')_k - a_j(\beta')_k$.
Taking this one step further, 
$\inv[0]{(B')} = \inv{B} + \overline{a}_i(\beta')_k - \overline{a}_j(\beta')_k$.
Then for an arbitrary column $a_p$,
\begin{align}
\inv[0]{(B')} a_p & = \inv{B} a_p + \overline{a}_i(\beta')_k a_p -
	\overline{a}_j(\beta')_k a_p \notag \\
\overline{a}'_p & = \overline{a}_p +
	e_k ( \frac{\overline{a}_{kp}}{\overline{a}_{kj}} ) -
	\overline{a}_j ( \frac{\overline{a}_{kp}}{\overline{a}_{kj}} )
	\label{Eqn:abarupdate}
\end{align}
(recalling that $(\beta')_k = \beta_k/\overline{a}_{kj}$).

To see that (\ref{Eqn:abarupdate}) amounts to
$\eta'_p = \eta_p - \eta_j( \dfrac{\overline{a}_{kp}}{\overline{a}_{kj}} )$,
it's helpful to expand the vectors:
\begin{equation*}
\overline{a}'_p =
  \begin{bmatrix} \overline{a}_{1p} \\ \vdots \\
		  \overline{a}_{kp} \\ \vdots \\
		  \overline{a}_{mp} \end{bmatrix} +
  \begin{bmatrix} 0 \\ \vdots \\
		  1 \\ \vdots \\
		  0 \end{bmatrix}\frac{\overline{a}_{kp}}{\overline{a}_{kj}} -
  \begin{bmatrix} \overline{a}_{1j} \\ \vdots \\
		  \overline{a}_{kj} \\ \vdots \\
		  \overline{a}_{mj} \end{bmatrix}
		  \frac{\overline{a}_{kp}}{\overline{a}_{kj}} .
\end{equation*}
With a little thought, it can be seen that the middle term represents one
half of the
permutation which moves $x_j$ into the basic partition of $\eta'_j$.
(The other half moves $x_i$ into the nonbasic partition).
When updating $\eta_i$, the update formula can be collapsed to
$\eta'_i = - \eta_j/\overline{a}_{kj}$, since $\overline{a}_{ki} = 1$.
Summarising, the update formul\ae{} for the edge directions $\eta_j$ are
\begin{equation}
\begin{split}
\eta'_p & = \eta_p - \eta_j( \frac{\overline{a}_{kp}}{\overline{a}_{kj}} ),
\qquad p \neq i \\
\eta'_i & = - \eta_j/\overline{a}_{kj}. \label{Eqn:etaupdate}
\end{split}
\end{equation}

In fact, the code actually stores and updates $\gamma_j^{\,2}$.
With (\ref{Eqn:etaupdate}) in hand, derivation of the update formul\ae{}
are straightforward:
\begin{align}
\begin{split}
(\gamma^{\,\prime}_p)^2 & = \eta'_p \cdot \eta'_p \\
  & = (\eta_p - \eta_j( \frac{\overline{a}_{kp}}{\overline{a}_{kj}} )) \cdot
      (\eta_p - \eta_j( \frac{\overline{a}_{kp}}{\overline{a}_{kj}} )) \\
  & = \eta_p \cdot \eta_p -
      2( \frac{\overline{a}_{kp}}{\overline{a}_{kj}} )\eta_j \cdot \eta_p +
      ( \frac{\overline{a}_{kp}}{\overline{a}_{kj}} )^2 \eta_j \cdot \eta_j \\
  & = \gamma_p^{\,2} -
      2( \frac{\overline{a}_{kp}}{\overline{a}_{kj}} )
      \begin{bmatrix} \overline{a}^T_j & e^T_j \end{bmatrix}
      \begin{bmatrix} \overline{a}_p \\ e_p \end{bmatrix} +
      ( \frac{\overline{a}_{kp}}{\overline{a}_{kj}} )^2 \gamma_j^{\,2} \\
  & = \gamma_p^{\,2} -
      2( \frac{\overline{a}_{kp}}{\overline{a}_{kj}} )
      (\overline{a}^T_j \inv{B}) a_p +
      ( \frac{\overline{a}_{kp}}{\overline{a}_{kj}} )^2 \gamma_j^{\,2}
      \label{Eqn:gammapupdate}
\end{split} \\[.5ex]
\begin{split}
(\gamma^{\,\prime}_i)^2 & = \eta'_i \cdot \eta'_i \\
  & = \eta_j/\overline{a}_{kj} \cdot \eta_j/\overline{a}_{kj} \\
  & = \gamma_j^{\,2}/\overline{a}^2_{kj}  \label{Eqn:gammaiupdate}
\end{split}
\end{align}

Equations (\ref{Eqn:etaupdate}) can be used directly to update the
$\tilde{\eta}_j$.
To adapt (\ref{Eqn:gammapupdate}) and (\ref{Eqn:gammaiupdate}) for the
$\tilde{\gamma}_j$,
a little algebra should serve to see that it's sufficient to substitute
$\tilde{a}_j$ in (\ref{Eqn:gammapupdate}), as well as using
$\tilde{\gamma}_p$ and $\tilde{\gamma}_j$.

It is straightforward to observe that when equations (\ref{Eqn:etaupdate})
are premultiplied by $c$, they can be used to update the reduced costs as
\begin{align*}
\begin{split}
\overline{c}'_p & = \overline{c}_p -
	\overline{c}_j( \frac{\overline{a}_{kp}}{\overline{a}_{kj}} )
\qquad p \neq i \\
\overline{c}'_i & = - \overline{c}_j/\overline{a}_{kj}.
\end{split}
\end{align*}

\subsection{Dual Steepest Edge Pricing}
\label{sec:DSEPricing}

The dual simplex in \dylp uses dual steepest edge (DSE) pricing; the algorithm
used is described as dual algorithm 1 (`steepest 1') in
Forrest and Goldfarb \cite{For92}.

The values $\overline{b} = \inv{B}b$ are the reduced costs of the nonbasic
dual variables.
Analogous to Dantzig pricing in the primal case, one can choose a entering
dual variable $y_i$ such that $\overline{b}_i$ has appropriate sign and the
largest magnitude over all reduced costs, but there is the same problem with
scaling.
The version of dual steepest edge (DSE) pricing implemented in \dylp
scales $\overline{b}_i = \beta_i b$ by $\rho_i = \norm{\beta_i}$,
choosing
a leaving variable $x_i$ with $\overline{b}_i$ of appropriate sign
and the largest $\displaystyle \abs{\frac{\beta_i b}{\norm{\beta_i}}}$,
effectively
calculating the change in the dual objective value over a unit vector in
the dual direction of motion in the space of the dual variables.
This gives a uniform pricing comparison, using the slope of the dual edge.

In the next few paragraphs, an alternative motivation of the algorithm is
presented which (perhaps) clarifies the relationship between dual
algorithm 1 and dual algorithm 2 in that paper%
\footnote{Those who have
read \cite{For92} are warned that the author's notation is in no way compatible
with that of Forrest and Goldfarb.}.

To see how DSE operates within the context of the revised primal simplex
tableau, we can refer back to equations \eqnref{Eqn:DualBasicVars} and
\eqnref{Eqn:DualObj} from \secref{sec:Notation}, repeated here:
\begin{equation}
\begin{split}
\begin{bmatrix} \sigma^\mathcal{B} & y^\mathcal{B} \end{bmatrix} & =
    (-c)\inv[-2]{\mathcal{B}} -
    \begin{bmatrix} \sigma^\mathcal{N} & y^\mathcal{N} \end{bmatrix}
    \mathcal{N}\inv[-2]{\mathcal{B}} \\
  & = \begin{bmatrix}
	c^N - c^B \inv[0]{(B^t)} N^t & -c^B \inv[0]{(B^t)}
      \end{bmatrix} -
      \begin{bmatrix} \sigma^\mathcal{N} & y^\mathcal{N} \end{bmatrix}
    \begin{bmatrix}
      -\inv[0]{(B^t)} N^t & -\inv[0]{(B^t)} \\
      B^l\inv[0]{(B^t)} N^t - N^l & B^l\inv[0]{(B^t)}
    \end{bmatrix}
\end{split} \tag{\ref{Eqn:DualBasicVars}}
\end{equation}
and
\begin{equation}
\begin{split}
z & = \begin{bmatrix} \sigma^\mathcal{B} & y^\mathcal{B} \end{bmatrix}
      \begin{bmatrix} 0 & b^t \end{bmatrix}^T +
      \begin{bmatrix} \sigma^\mathcal{N} & y^\mathcal{N} \end{bmatrix}
      \begin{bmatrix} 0 & b^l \end{bmatrix}^T \\
  & = (-c)\inv[-2]{\mathcal{B}} b^\mathcal{B} +
      \begin{bmatrix} \sigma^\mathcal{N} & y^\mathcal{N} \end{bmatrix}
      (b^\mathcal{N} - \mathcal{N}\inv[-2]{\mathcal{B}} b^\mathcal{B}) \\
  & = -c^B \inv[0]{(B^t)} b^t +
      \begin{bmatrix} \sigma^\mathcal{N} & y^\mathcal{N} \end{bmatrix}
      \begin{bmatrix}
	 \inv[0]{(B^t)} b^t \\ b^l - B^l\inv[0]{(B^t)} b^t
      \end{bmatrix}
\end{split} \tag{\ref{Eqn:DualObj}}
\end{equation}
Recall that the values of the dual basic variables are the reduced costs of
the primal problem, and the reduced costs of the dual variables are the values
of the primal basic variables (\cf equations \eqnref{Eqn:PrimalBasicVars} and
\eqnref{Eqn:PrimalObj}).

By analogy to the primal pivoting rules, for dual simplex
we want to choose a nonbasic dual variable which will move us in a direction
of steepest descent.
If the nonbasic dual is to increase, its reduced cost must be less than 0 in
order to see a reduction in the dual objective.
This corresponds to the case of a primal variable which will be increased and
driven out of the basis at its lower bound with a positive primal reduced
cost.
If the nonbasic dual is to decrease, its reduced cost must be greater than
0 in order to see a reduction in the dual objective.
This corresponds to the case of a primal variable which will be decreased and
driven out of the basis at its upper bound with a negative primal reduced cost.

The actual direction of motion in the full dual space ($y$ and $\sigma$) is
specified by a row of
\begin{equation*}
\mathcal{N}\inv[-2]{\mathcal{B}} = \begin{bmatrix}
			-\inv[0]{(B^t)} N^t & -\inv[0]{(B^t)} \\
			B^l\inv[0]{(B^t)} N^t - N^l & B^l\inv[0]{(B^t)}
		      \end{bmatrix},
\end{equation*}
a vector which is not readily available in the revised
primal simplex\footnote{%
It's necessary to calculate one such row $\overline{a}_i$ once the entering
dual variable has been selected, but only one.
For the typical problem in which the number of variables greatly
exceeds the number of constraints, the norms of these vectors are expensive to
calculate when initialising the pricing algorithm, and the updates are
expensive.
The algorithm which uses the full dual direction of motion is the one that
Forrest and Goldfarb describe as dual algorithm 2.}.
However, one can make an argument that there's no need to consider the
component of the direction of motion in the subspace of the dual surplus
variables when choosing the entering dual variable.
(More positively, we can take the view that we're only interested in motion
in the polyhedron $\{y \in R^m \mid yA \geq -c, y \geq 0\}$ defined by the
dual variables.)
Changes in the surplus variables cannot affect the objective directly, as
they account for the 0's in the augmented and partitioned $b$ vector.
Algebraically, we can see that the dual basic portion
of $b$, $\begin{bmatrix}0 & b^t \end{bmatrix}^T$, guarantees that there will
never be any direct contribution from the columns
of $\mathcal{N}\inv[-2]{\mathcal{B}}$ involving $N$.
The component of motion in the space of the dual variables $y$ is then simply
the rows $\beta_i$ of $\inv{B}$, which are easily available from the primal
tableau.
(The analogous action in the primal problem --- ignore the component of
$\eta_j$ in the subspace of the primal slack variables --- offers no
computational advantage.)

Given a rationale for taking the rows $\beta_i$ of $\inv{B}$ as the
component of interest in the dual direction of motion, what remains is to work
out the details.
Since we're aiming for a steepest edge algorithm, we'll be interested in
iteratively updating $\norm{\beta_i}^2 = \beta_i \cdot \beta_i$, the square
of the norm of a row $\beta_i$.
Given the update formul\ae{} for $\beta_i$ derived in
\secref{sec:BasisUpdates},
the development of the update formul\ae{} for $\rho_i = \norm{\beta_i}^2$ is
straightforward algebra.
Let $x_i$ be the leaving variable and $x_j$ be the entering variable, and
assume $x_i$ occupies row $k$ of the basis $B$ before the update.
We have
\begin{align}
\begin{alignedat}{2}
\rho'_i & = \rho_i -
	    2\frac{\overline{a}_{lj}}{\overline{a}_{kj}}\beta_i\cdot\beta_k +
	    (\frac{\overline{a}_{lj}}{\overline{a}_{kj}})^2\rho_k &
	    \qquad\qquad i \neq k \\
\rho'_k & = (\frac{1}{\overline{a}_{kj}})^2\rho_k
\end{alignedat}
\end{align}
Since the update will be performed for all rows in the basis, it's worth
calculating the vector $\tau = \inv{B}\beta^T_k$ to obtain all the inner
products $\beta_i \cdot \beta_k$ in one calculation.


\section{Anti-Degeneracy Using a Perturbed Subproblem}
\label{sec:PerturbedAntiDegeneracy}

In both primal and dual simplex, \dylp implements an anti-degeneracy
algorithm using a perturbed subproblem.
It builds on a method described by Ryan \& Osborne \cite{Rya88}
in which all variables are assumed to have lower bounds
of zero and upper bounds of infinity.

The original algorithm is easily described in terms of the primal problem.
When degeneracy is detected, a restricted subproblem is formed consisting only
of the constraints involved in the degeneracy (\ie, constraints $i$ such that
$\overline{b}_i = 0$).
The values $\overline{b}_i$ are given (relatively) large perturbations and
pivots are performed within the context of the restricted subproblem until a
direction of recession from the degenerate vertex is found (indicated by
apparent unboundedness).
The original unperturbed values of $\overline{b}_i$ are then restored (since
all pivots were, in actuality, simply changes of basis while remaining at the
degenerate vertex) and the full problem is resumed.

An alternative view goes directly back to the constraints involved in
the degeneracy.
By perturbing their right-hand-side values $b_i$, the single vertex formed by
the constraints is fractured into many vertices.
For the simple case of $0 \leq x \leq \infty$, we have
$\overline{b} = B^{\,-1} b$, so perturbing $\overline{b}$ by the vector
$\xi$ is equivalent to perturbing $b$ by the vector $-B\xi$.

In dual simplex, this algorithm can be implemented directly.
The restricted subproblem is formed from the dual constraints (primal columns)
corresponding to basic dual variables (primal reduced costs) whose value
is zero.
The perturbation is introduced directly to the values $\overline{c}_j$,
taking care to maintain dual feasibility.
The perturbation is maintained by the incremental update of the dual variables
and reduced costs after each pivot.
When accuracy checks are performed, the correct value of zero can be
substituted on the fly for the perturbed values.

The trick to implementing this algorithm in the context of variables with
arbitrary upper and lower bounds is to distinguish between apparent motion
due to the introduced perturbations and real motion (along a direction of
recession) which is nonetheless limited by a bound on a variable.
\dylp uses an array, \pgmid{dy_brkout}, to record the direction of
change (away from the current bound) required for nondegenerate but
bounded motion.

A second, more subtle problem, is that the perturbation for a given variable
must be sufficiently small to avoid a false indication of a nondegenerate
pivot.
\dylp scales the perturbation to be at most $.001(u_i - l_i)$, but there is
no easy way to guarantee that this is sufficiently small.
Consider two variables $x_i$ and $x_k$, and assume that they occupy rows
$i$ and $k$ in the basis, with perturbed values
$\tilde{b}_i$ and $\tilde{b}_k$, respectively.
For concreteness, assume that each was originally degenerate at its lower
bound, so that a pivot which resulted in one variable leaving at its upper
bound would be nondegenerate.
For $\overline{a}_{ij}$ and $\overline{a}_{kj}$ of appropriate sign to move
$x_i$ toward $l_i$ and $x_k$ toward $u_k$, given a situation where
$|\overline{a}_{ij}| \ll |\overline{a}_{kj}|$, it is not possible to assure
that
\begin{displaymath}
\frac{\tilde{b}_i - l_i}{\overline{a}_{ij}} <
\frac{u_k - \tilde{b}_k}{\overline{a}_{kj}}
\end{displaymath}
without actually testing each pair.
In this case, the perturbation introduced for $x_i$ is too large, and the
resulting $\Delta_{ij}$ \textit{appears} to allow $x_k$ to become the limiting
variable, leaving the basis with a bounded but nondegenerate change.
When \dylp detects this problem, it will reduce the perturbation by a factor
of 10 and form the restricted subproblem again.
If a (small) limit on the number of attempts is exceeded, \dylp simply gives
up and takes a degenerate pivot.

A second problem occurs when a perturbation is so small as to be
indistinguishable next to the bound.
Specifically, the test to determine if a variable $x_i$ is at bound is
$\pgmid{dy_tols.zero}(1+|\mathit{bnd}_i|) < |x_i - \mathit{bnd}_i|$.
If $\mathit{bnd}_i$ is large, the perturbation can be swamped.
This situation can arise if $u_i$ and $l_i$ as given to \dylp are nearly equal,
or due to reduction of the perturbation as described in the previous
paragraph.


\section{Lightweight Anti-Degeneracy Measures Based on Hyperplane Alignment}
\label{sec:AntiDegenLite}

\newcommand{\nblb}{\,\underline{\makebox[\width-3pt][c]{$\scriptstyle \,N$}}}
\newcommand{\nbub}{\;\overline{\makebox[\width-3pt][c]{$\scriptstyle N\,$}}}

In addition to the perturbed subproblem anti-degeneracy algorithm described in
\secref{sec:PerturbedAntiDegeneracy}, \dylp provides a light-weight
anti-degeneracy mechanism based on hyperplane alignment.
In the code and documentation, this is referred to as `anti-degen lite'.

Each constraint $a_k x \leq b_k$ defines an associated hyperplane at equality.
In the absence of degeneracy, a simplex pivot consists of moving away from one
hyperplane along an edge until another hyperplane blocks further progress.
The hyperplane being left becomes loose, and the blocking hyperplane becomes
tight.
The choice of entering variable $x_j$ determines the constraint that will
become loose,
and the choice of leaving variable $x_i$ determines the
constraint that will become tight.

Ideally, the choice of constraints is unique, but life is seldom ideal.
Most often the lack of uniqueness is due to degeneracy, in which one
or more basic variables are at their upper or lower bounds.
Geometrically, there are more tight constraints than required to define the
current extreme point.
In this case the change of basis that occurs with the pivot will not result in
a move to a new extreme point.

This section describes a suite of measures based on hyperplane alignment which
try to better the odds of selecting hyperplanes which will form an edge that
escapes from the degenerate extreme point.

Because all constraints at a degenerate vertex are tight, some terminology will
be useful to describe the changes associated with a pivot.
For this section only, the terms activate and deactivate will be used as
follows:
\begin{itemize}
  \item
  When the slack variable for a constraint moves to the basic partition, the
  constraint is deactivated.
  When the slack variable moves to the nonbasic partition, the constraint is
  activated.

  \item
  When an architectural variable moves to the basic partition, the relevant
  bound constraint is deactivated.
  When an architectural variable moves to the nonbasic partition, the
  relevant bound constraint is activated.
\end{itemize}

\subsection{Activation of Constraints}

In both the primal and dual simplex algorithms, the constraint which is
activated by a pivot depends on the leaving variable and its direction of
motion.
Before discussing the types of alignment calculations, it will be useful to
discuss the activation of constraints.
Knowing the type of constraint (`$\leq$' or `$\geq$') is necessary because it
determines the direction of the normal with respect to the feasible region.

\dylp assumes that the majority of explicit constraints of the primal problem
are of the form $a_k x \leq b_k$.
It also understands range constraints of the form
$\check{b}_k \leq a_k x \leq b_k$.
These are implemented by placing an upper bound on the associated slack
variable $s_k$, but for purposes of determining the constraint to be activated
we need to recognise that there are really two
constraints, $a_k x \geq \check{b}_k$ and $a_k x \leq b_k$.

Bounded variables are handled implicitly by the primal simplex algorithm.
When a bounded variable becomes nonbasic at its lower bound, the constraint
$x_k \geq l_k$ is activated; when it becomes nonbasic at its upper bound,
the constraint $x_k \leq u_k$ is activated.

A final complication is introduced in phase I of the primal simplex, where
it's possible to approach a constraint from the `wrong' side in the process
of finding a primal feasible basic solution.
For example, if a slack variable $s_k < 0$ will increase and leave the basis
at 0, the constraint which is becoming tight is actually $a_k x \geq b_k$.
\aside{Is this really a valid insight? In terms of blocking motion, it's true.
In terms of alignment with the objective, for example, I have doubts.}

Turning to the dual problem, the question of what constraint is being activated
is substantially obscured by the mechanics of running the dual simplex
algorithm from the primal data structures.
A much clearer picture can be obtained by expanding the primal system to
include explicit upper and lower bound constraints and examining the
resulting dual constraints
(\cite[\S3.4]{For92}, or see \cite{Haf98a} for an extended development).
Briefly, let $y$ be the dual variables associated with the original
explicit constraints $a_k x \leq b_k$
(the architectural constraints), $\check{y}$ be the dual variables associated
with the lower bound constraints, and $\hat{y}$ be the dual variables
associated with the upper bound constraints.
A superscript $\underline{N}$ will represent the set of primal variables at
their lower bound, $\overline{N}$ the set of primal variables at their upper
bound, and $B$ the set of basic primal variables.
The set of dual constraints can then be written as
\begin{equation*}
\begin{aligned}
yB - \check{y}^{B}I + \hat{y}^{B}I & = c^B \\
y\underline{N} - \check{y}^{\nblb}I +
	\hat{y}^{\nblb}I & = c^{\nblb} \\
y\overline{N} - \check{y}^{\nbub}I +
	\hat{y}^{\nbub}I & = c^{\nbub}
\end{aligned}
\end{equation*}
where the first term in each dual constraint comes from the primal
architectural constraints, the second term from the lower bound constraints,
and the third term from the upper bound constraints.
The variables $\check{y}^{B}$, $\hat{y}^{B}$, $\hat{y}^{\nblb}$,
and $\check{y}^{\nbub}$ are dual nonbasic and therefore have the value
zero.
(They are associated with primal bound constraints which are not tight.)
We can rewrite the dual constraints as
\begin{equation*}
\begin{aligned}
yB = c^B \\
y\underline{N} - \check{y}^{\nblb}I = c^{\nblb} \\
y\overline{N} + \hat{y}^{\nbub}I = c^{\nbub}
\end{aligned}
\end{equation*}
We can then interpret the constraints
$y\underline{N} - \check{y}^{\nblb}I = c^{\nblb}$
as $y\underline{N} \geq c^{\nblb}$, with $\check{y}^{\nblb}$ acting as the
surplus variables.
Similarly, the constraints
$y\overline{N} + \hat{y}^{\nbub}I = c^{\nbub}$ can be interpreted as
$y\overline{N} \leq c^{\nbub}$, with $\hat{y}^{\nbub}$ acting as the slack
variables.

With this interpretation in hand, it's easy to determine the hyperplane that's
activated by a pivot.
When a dual variable $\check{y}^{\nblb}_k$ is driven out of the basis at 0
($x_k$ enters rising from its lower bound),
the constraint $y a_k \geq c_k$ becomes tight.
When a dual variable $\hat{y}^{\nbub}_k$ is driven out of the basis at 0
($x_k$ enters decreasing from its upper bound),
the constraint $y a_k \leq c_k$ becomes tight.
This interpretation is uniform for the original primal variables as well as
the primal slack variables.

For the most common case of a primal constraint $a_i x \leq b_i$, with
associated slack $s_i$, $0 \leq s_i \leq \infty$, the dual constraint
reduces to $y_i \geq 0$, and this is handled as an implicit bound by the
dual simplex algorithm implemented in \dylp.
(Range constraints complicate the interpretation, but not the mechanics, of
the implementation.
Again, see \cite{Haf98a} for a detailed explanation.)

In the sections which follow, the alignment calculations are developed in
terms of the most common constraint form ($a_k x \leq b_k$ in the case of
the primal simplex, and $y a_k \geq c_k$ in the case of the dual simplex).
Accommodating the different constraint types described in this section is
simply a matter of correcting the sign of the calculation as needed to account
for the direction of the constraint normal.


\subsection{Alignment With Respect to the Objective Function}

The primal objective used in \dylp is $\min cx$.
We need to move in the direction $-c$ until
we reach an extreme point of the polytope where the cone formed by the normals
of the active constraints includes $-c$.

If the goal is to travel in the direction $-c$, one approach would be to
leave each vertex by moving along the edge which most nearly points in the
direction $-c$.
The edges traversed by the simplex algorithm are simply the intersections of
active hyperplanes.
If we're trying to construct an edge with which we can leave a degenerate
vertex, we could choose to activate a hyperplane $a_k x = b_k$ such that
$-c$ most nearly lies in the hyperplane, on the
theory that its intersection with other active hyperplanes at the vertex is
more likely to produce an edge with the desired orientation.
This is the `Aligned' strategy, because we want the hyperplanes most closely
aligned with the normal of the objective.

Going to the other extreme, at the optimal vertex it must be true that the
active
hyperplanes block further motion in the direction $-c$, and $-c$ must lie
within the cone of normals of the active hyperplanes.
One can make the argument that a good choice of hyperplane would the one that
most nearly blocks motion in the direction $-c$, as it's likely to be active
at the optimal vertex.
This is called the `Perpendicular' strategy, because we want the hyperplanes
which are most nearly perpendicular to the normal of the objective.

For constraints $a_k x \leq b_k$ the normal points out of the feasible region.
Let the alignment of the normal $a_k$ with $-c$ be calculated as
$\displaystyle \frac{a_i \cdot c}{\norm{a_i}\norm{c}}$.
Then for the Perpendicular strategy, we want to select the hyperplane
$a_i x = b_i$
such that $\displaystyle i = \arg \max_k \frac{a_k \cdot c}{\norm{a_k}}$ over
all constraints $a_k x \leq b_k$ in the degenerate set.

For the Aligned strategy, the criteria is a bit more subtle.
If $a_k \cdot -c = 0$, $-c$ lies in the hyperplane $a_k x = b_k$.
Selecting the hyperplane $i$ such that
$\displaystyle i = \arg \min_k \abs{\frac{a_k \cdot c}{\norm{a_k}}}$
is not quite sufficient.
Where possible, \dylp attempts to choose hyperplanes which are tilted in the
direction of the objective, so as to bound the problem.
The preferred hyperplane is $a_i x = b_i$ such that
$\displaystyle i = \arg 
  \min_{\{k \mid a_k \cdot c \geq 0\}} \frac{a_k \cdot c}{\norm{a_k}}$
over the constraints in the degenerate set.
If $a_k \cdot c < 0$ for all $k$, the preferred hyperplane is chosen as
$\displaystyle i = \arg \max_k \frac{a_k \cdot c}{\norm{a_k}}$.

The dual objective used in \dylp is $\min yb$, but we must be careful here to
to include the effect of the bounds on the primal variables.
The objective is properly stated as
$\min \begin{bmatrix} y & \check{y} & \hat{y} \end{bmatrix}
\trans{\begin{bmatrix} b & -l & u \end{bmatrix}}$,
and we will need to include the
coefficients of $\check{y}$ and $\hat{y}$ in the constraint normals.
(In the primal we could ignore this effect, because the objective coefficients
associated with the slack variables are uniformly zero.)

For dual constraints $y a_k \geq c_k$, the normal
$\begin{bmatrix} a_k & -e_k & 0 \end{bmatrix}$ will point into the
feasible region
and \dylp calculates the alignment of 
$\begin{bmatrix} -b & l & -u \end{bmatrix}$ with the hyperplane as
$\displaystyle \frac{b \cdot a_k + l_k}
  {(\norm{a_k}+1)\norm{\begin{bmatrix} b & -l & u \end{bmatrix}}}$, so that a
positive result identifies a constraint which blocks motion in
the direction of the objective.
For a constraint $y a_k \leq c_k$, the calculation is
$\displaystyle \frac{(-b) \cdot a_k - u_k}
  {(\norm{a_k}+1)\norm{\begin{bmatrix} b & -l & u \end{bmatrix}}}$.
Selection of a specific leaving variable $\check{y}^{\nblb}_k$ or
$\hat{y}^{\nbub}_k$ is done using the same criteria outlined for the
Perpendicular and Aligned cases in the primal problem.

\subsection{Alignment With Respect to the Direction of Motion}

The selection of an entering variable specifies the desired direction of
motion for the pivot.
At a degenerate vertex, we cannot move in the desired direction because the
set of active hyperplanes does not contain this edge.
Intuitively, activating a hyperplane which is closely aligned with the desired
direction of motion might increase the chance of being able to move in that
direction.

For the primal simplex, the direction of motion
derived in \secref{sec:PSEPricing}
is $\eta_j = \trans{\begin{bmatrix} -B^{\,-1}a_j & -e_j \end{bmatrix}}$.
The normal of a constraint $a_k x \leq b_k$ points out of the feasible region.
The alignment of $\eta_j$ and the normal $a_k$ is calculated
as $\displaystyle \frac{a_k \cdot \eta_j}{\norm{a_k}\norm{\eta_j}}$, so that
a positive value identifies a hyperplane which blocks motion in the
direction $\eta_j$.

It's important to note that normal $a_k$ in this calculation is that of the
inequality --- the coefficient associated with the slack $s_k$
is \textit{not} included.
This means that $a_k \cdot \eta_j \equiv -\overline{a}_{kj}$.
For a bound constraint, the relation is obvious by inspection.
If, for example, the constraint is $x_k \leq u_k$, the normal is $e_k$, and
$e_k \cdot -\overline{a}_j = -\overline{a}_{kj}$.
For an architectural constraint, it's necessary to look at the calculation
in a way that separates the contributions of the architectural and slack
variables, and basic and nonbasic variables.
We are interested in the structure of the product
$\begin{bmatrix} B & N \end{bmatrix}
 \begin{bmatrix} -B^{\,-1}N \\ I \end{bmatrix}$
for loose constraints which will be activated by pivoting the associated slack
variable out of the basis.
Breaking up the matrices as detailed in \secref{sec:Notation}, we have
\begin{align*}
\begin{bmatrix} B^l & I & N^l & 0 \end{bmatrix}
\begin{bmatrix} -B^{\,-1}N \\ I \end{bmatrix} & = 
\begin{bmatrix} B^l & I & N^l & 0 \end{bmatrix}
\begin{bmatrix}
 -\begin{bmatrix} (B^t)^{-1} & 0 \\ -B^l(B^t)^{-1} & I \end{bmatrix}
 \begin{bmatrix} N^t \\ N^l \end{bmatrix} \\
 \begin{bmatrix} I & 0 \\ 0 & I \end{bmatrix}
\end{bmatrix} \\
& = 
\begin{bmatrix} B^l & I & N^l & 0 \end{bmatrix}
\begin{bmatrix}
  -(B^t)^{-1} N^t & -(B^t)^{-1} \\
  B^l(B^t)^{-1}N^t - N^l & B^l(B^t)^{-1} \\
  I & 0 \\
  0 & I
\end{bmatrix} \\
& =
\begin{bmatrix}
-B^l(B^t)^{-1} N^t + B^l(B^t)^{-1} N^t - N^l + N^l &
-B^l(B^t)^{-1} + B^l(B^t)^{-1}
\end{bmatrix}
\end{align*}
Removing the contribution due to the basic slack variables, we have
$\begin{bmatrix} -B^l(B^t)^{-1} N^t + N^l &  -B^l(B^t)^{-1} \end{bmatrix}$.
Because the leaving variable for the pivot is a slack, the pivot element
$\overline{a}_{kj}$ will be drawn from the component
$\begin{bmatrix} B^l(B^t)^{-1}N^t - N^l & B^l(B^t)^{-1} \end{bmatrix}$ in
$-B^{-1}N$, and the equivalence is verified.

To finish the alignment calculation for the purposes of selecting a leaving
variable, all that is needed is to perform the normalisation by
$\norm{a_k}\norm{\eta_j}$, and since $\norm{\eta_j}$ is constant during the
selection of the leaving variable, we need only divide by $\norm{a_k}$ for
comparison purposes.
The selection of a leaving variable using the Aligned strategy is as outlined
in the previous section.

Given that $a_k \cdot \eta_j \equiv -\overline{a}_{kj}$, it's worth taking
a moment to consider a common tie-breaking rule for selecting the leaving
variable --- pick the variable with the largest $\abs{\overline{a}_{kj}}$,
to maintain numerical stability.
In fact, this amounts to selecting a hyperplane to activate using an
unnormalised variation of the Perpendicular strategy.
The obvious corollary is that using the Aligned strategy presents a potential
danger to numerical stability by deliberately choosing small pivots.

For the dual simplex, the direction of motion $\zeta_i$ is more complicated.
Fortunately, we need only consider the portion of $\zeta_i$ in the space of
the dual variables $y$.
As derived in \secref{sec:DSEPricing},
this is simply row $\beta_i$ of $B^{\,-1}$.
For the dual constraints $y a_k \geq c_k$, the normal points into the feasible
region.
To maintain the convention that the alignment calculation should produce a
positive result if the constraint blocks motion, the alignment calculation
used by \dylp is
$\displaystyle -\frac{\zeta_i \cdot a_k}{\norm{\zeta_i}\norm{a_k}}$.
Given that we're only interested in the portion of $\zeta_i \cdot a_k$
contributed by the dual variables $y$,
it's immediately apparent that the alignment
calculation can be reduced to
$\displaystyle -\frac{\overline{a}_{ik}}{\norm{a_k}}$ for purposes of selecting
the leaving dual variable.
The final selection of a leaving dual variable using the Aligned or
Perpendicular strategy proceeds as outlined in the previous section.


\section{The LP Basis}
\label{sec:LPBasis}

\dylp requires three capabilities from a basis maintenance module:
\begin{itemize}
\item	Factoring of the basis to create the basis inverse.

\item	Update of the basis inverse for a pivot.

\item	Premultiplication (`ftran') of a column vector by the basis inverse,
	and postmultiplication (`btran') of a row vector by the basis inverse.
\end{itemize}
\dylp uses the basis maintenance module from \glpk to provide these services.
Knowledge of the structure and operation of the \glpk subroutines is
confined to a set of interface subroutines in the file \coderef{dy_basis.c}{}.
The majority of these are straightforward interface functions whose sole
purpose is to hide the \glpk structures and to mediate between \glpk and the
remainder of the code.

\subsection{The \glpk Basis Module Interface}
\label{sec:GLPKBasisModule}

Very roughly, the \glpk basis maintenance module has a two-layer structure.
The top layer (\coderef{glpinv.c}{}) provides the basic services for a
generic basis inverse.
In turn, the top layer calls on a second layer (\coderef{glpluf.c}{}) to
provide a specific
implementation of the basis inverse data structures and algorithms.
Dynamic Markowitz pivoting with partial threshold pivot selection is
used to factor a basis.

The routine \pgmid{dy_initbasis} is used to initialise the basis module.
The capacity of the basis, algorithm options, and numeric tolerances are
set at initialisation (\vid \secref{sec:GLPKBasisInit}).
The basis is deleted by the routine \pgmid{dy_freebasis}.
Changing the basis capacity is implemented in \dylp by saving options and
tolerances for the existing basis, deleting the existing basis,
and creating a new basis of the appropriate size.
The capacity is checked each time the basis is factored; changes are
invisible to clients.
The \glpk basis module will resize its own internal data structures whenever
it determines that this is required.

In the main, \dylp uses the basis module in a standard way for
factoring and pivoting.
There are some departures from \glpk defaults:
\begin{itemize}
  \item
  The initial size of the sparse vector working area is tripled.

  \item
  The limit on element growth (\pgmid{luf.max_gro}) is reduced from
  $10^{12}$ to $10^6$.

  \item
  The minimum value for elements on the diagonal of the factorisation
  (\pgmid{luf_basis.upd_tol} is reduced from $10^{-6}$ to $10^{-10}$.

  \item
  Instead of a fixed default of $.1$, the pivot stability tolerance is
  dynamically adjusted in a range between $.01$ and $.95$ based on \dylp's
  assessment of the numerical stability of the current basis.
  The number of pivot candidates examined when factoring the basis is
  also adjusted in the range $4$ to $10$.
  More candidates are considered as the stability requirement is raised
  in the hope of finding a numerically stable candidate without compromising
  sparsity.
\end{itemize}

The routine \pgmid{dy_setpivparms} is provided to adjust the pivot stability
tolerance and pivot candidate limit.
Adjustment of the pivot selection parameters is done according to a
fixed schedule of tolerance and limit values
kept in the static data structure \coderef{dy_basis.c}{pivtols}.
The client specifies an integer delta which is used to select a pair of
values from the schedule.

Pre- and post-multiplication of vectors by the basis inverse are provided by
the routines \pgmid{dy_ftran} and \pgmid{dy_btran}, respectively.

\subsection{Factoring}
\label{sec:BasisFactoring}

For factoring the basis, the routine \pgmid{dy_factor} provides
significant error recovery functions
on top of the basic abilities of \glpk.
The call structure is shown in Figure 
\ref{fig:FactorCallGraph}.
\begin{figure}[htb]
\centering
\includegraphics{\figures/factorcalls}
\caption{Call Graph for \protect\pgmid{dy\_factor}} \label{fig:FactorCallGraph}
\end{figure}

A singular basis can occur because of a simplex pivot attempt or as the result
of a change in the coefficients of the basis because the client has fixed
variables and then requested a warm or hot start.
The factoring routine \pgmid{glp_inv_decomp} detects a singular basis and
reports the unpivoted rows and columns, but does not attempt to fix the basis.
\pgmid{adjust_basis} uses the information reported by \pgmid{glp_inv_decomp} to
attempt to patch the basis,
substituting columns associated with slack variables for the set of columns
identified as singular.
This sequence is repeated until the basis is successfully factored.

In the larger context of \dylp, patching the basis is the least of
the work.
\pgmid{dy_factor} will call \pgmid{adjust_therest} to adjust the \dylp data
structures as necessary to reflect the exchange of variables between the
basic and nonbasic partitions.
Depending on the phase, this can include updating the structures which
maintain the basis, recalculating the primal (\pgmid{dy_calcprimals})
and dual (\pgmid{dy_calcduals}) variables,
recalculating the reduced costs (\pgmid{dy_calccbar}),
resetting the DSE or PSE norms (\pgmid{dy_dseinit} and \pgmid{dy_pseinit},
respectively), clearing the list of variables marked
ineligible for pivoting (\pgmid{dy_clrpivrej}), and backing out a perturbed
subproblem (\pgmid{dy_degenout}). 

\pgmid{glp_inv_decomp} will abort an attempt to factor the basis if the current
pivot selection parameters give rise to numerical instability (detected as
excessive growth in the magnitude of the coefficients of the factored basis).
\pgmid{dy_factor} will make repeated tries to factor the basis, tightening
the pivot selection parameters before each attempt.
It will admit failure only if the numerical instability remains after the
pivot selection tolerances have been tightened as much as possible, so that
each pivot chosen is the maximum coefficient remaining in the unpivoted
portion of the basis.

\subsection{Pivoting}
\label{sec:BasisPivoting}

Pivoting is performed by \pgmid{dy_pivot}, which confirms the numerical
stability of the pivot element and calls \pgmid{glp_inv_update} to pivot
the basis.

To be judged numerically stable, a prospective pivot coefficient
$\overline{a}_{ij}$ must exceed the product
of the \glpk stability multiplier (\pgmid{luf.piv_tol}), the \dylp pivot
selection multiplier (\pgmid{dy_tols.pivot}),
and the maximum element in the transformed column
$\overline{a}_j = B^{\,-1}a_j$ (primal
simplex) or row $\overline{a}_i = \beta_i N$ (dual simplex).
Standard defaults in \dylp are
$5 \times 10^{-2}$ for the \glpk stability multiplier
and $1 \times 10^{-5}$ for the \dylp pivot selection multiplier, so that
the pivot coefficient is required to satisfy
$\abs{\overline{a}_{ij}} > (5 \times 10^{-7})(\max_k \abs{\overline{a}_{kj}})$
(primal simplex) or
$\abs{\overline{a}_{ij}} > (5 \times 10^{-7})(\max_k \abs{\overline{a}_{ik}})$
(dual simplex).
The routine \pgmid{dy_chkpiv} is supplied to perform this test, and is used
as a qualification test by the routines which select the leaving primal
variable in primal simplex and the entering primal variable in dual simplex.
The check performed in \pgmid{dy_pivot} should not fail, but is retained as a
precaution.

If $\overline{a}_{ij}$ is rejected as numerically unstable, the pivot attempt
is aborted.
In primal simplex, the entering variable $x_j$ will be placed on the rejected
pivot list.
For dual simplex, the leaving variable $x_i$ is placed on the rejected pivot
list.
Recovery from pivoting problems and the handling of the rejected pivot list are
discussed in \secref{sec:ErrorRecovery}.

A pivot can also fail
if it results in a singular basis or if the basis representation runs
out of space.
The implementation of \glpk requires that the basis be reloaded and factored
to recover from these errors; this is orchestrated by \pgmid{dy_duenna}
and discussed in \secref{sec:ErrorRecovery}.

Note that \pgmid{glp_inv_update} expects to be supplied with $L^{-1}a_j$ as a
hidden parameter.
\glpk provides the capability to control whether a call to \pgmid{glp_inv_ftran}
sets this hidden parameter.
This capability is exposed to clients as the second parameter
to \pgmid{dy_ftran}.



 
\section{Accuracy Checks and Maintenance}
\label{AccuracyChecks}

Primal and dual accuracy checks, primal and dual feasibility checks, and
factoring of the basis can be requested through the routine \pgmid{dy_accchk};
each action can be requested separately.

\dylp refactors the basis and performs accuracy checks at regular intervals,
based on a count of pivots which actually change the basis.
By default, primal and dual accuracy checks are performed at twice this
frequency.
During phase II of the primal and dual simplex algorithms, the appropriate
feasibility check is performed following each accuracy check.
\pgmid{dy_duenna} tracks the pivot count and requests checks and factoring
at the scheduled intervals.

\pgmid{dy_accchk} uses \pgmid{dy_factor} to factor the basis and recalculate
the primal and dual variables.
When the basis has been factored and has passed the accuracy checks, the
routine \pgmid{groombasis} checks that the status of the basic variables
matches their values and makes any necessary adjustments.

Failure of an accuracy check will cause the basis to be refactored.
Failure of an accuracy check immediately after refactoring will cause the
current pivot selection tolerances to be tightened by one increment before
another attempt is made.
\pgmid{dy_accchk} will repeat this cycle until the accuracy checks are
satisfied or there's no more room to tighten the pivot selection parameters.
On the other hand, each time that an accuracy check is passed without
refactoring the basis, the current pivot selection tolerances are loosened by
one increment, to a floor given by the minimum pivot selection tolerance.

The minimum pivot selection tolerance is reset to the loosest possible setting
at the start of each simplex phase.
If \pgmid{groombasis} detects and corrects major status errors (indicating
that an unacceptable amount of inaccuracy accumulated since the basis was
last factored), it will raise the minimum pivot selection tolerance.
Similarly, if the primal phase~I objective is found to be incorrect, or
primal or dual feasibility is lost when attempting to verify an optimal
solution, the current and minimum pivot selection tolerances will be raised
before returning to simplex pivots.
Raising the minimum pivot selection tolerance provides long-term control
(for the duration of a simplex phase) over reduction in the current pivot
selection tolerance.

The primal accuracy check is $B x^B = b - N x^N$.
Comparisons are made against the scaled tolerance
$\norm[1]{b}(\pgmid{dy_tols.pchk})$.
To pass the primal accuracy check, it must be that
\begin{equation*}
\norm[1]{(b - N x^N) - B x^B} \leq \norm[1]{b}(\pgmid{dy\_tols.pchk})
\end{equation*}

The dual accuracy check is $y B = c^B$.
Comparisons are made against the scaled tolerance
$\norm[1]{c}(\pgmid{dy_tols.dchk})$.
To pass the dual accuracy check, it must be that
\begin{equation*}
\norm[1]{c^B - y B} \leq \norm[1]{c}(\pgmid{dy\_tols.dchk})
\end{equation*}

The primal feasibility check is $l \leq x \leq u$.
For each variable, it must be true that
$x_j \geq l_j - (\pgmid{dy_tols.pfeas}) (1 + \abs{l_j})$ and
$x_j \leq u_j + (\pgmid{dy_tols.pfeas}) (1 + \abs{u_j})$.
In the implementation, only the basic variables are actually tested; nonbasic
variables are assumed to be within bound as an invariant property of the
simplex algorithm.
$\pgmid{dy_tols.pfeas}$ is scaled from \pgmid{dy_tols.zero} as
\begin{equation*}
\pgmid{dy\_tols.pfeas} =
  \min ( 1, \log \left( \frac{\norm[1]{x_B}}{\sqrt{m}} \right))
  (\pgmid{dy\_tols.zero})(\pgmid{dy_tols.pfeas_scale}).
\end{equation*}

The dual feasibility check is $\overline{c} = c^N - y N$ of appropriate
sign.
For each variable, it must be true that
$\overline{c}_j \leq \pgmid{dy_tols.dfeas}$ for $x_j$ nonbasic
at $u_j$ and $\overline{c}_j \geq -\pgmid{dy_tols.dfeas}$ for $x_j$ nonbasic
at $l_j$.
$\pgmid{dy_tols.dfeas}$ is scaled from \pgmid{dy_tols.cost} as
\begin{equation*}
\pgmid{dy\_tols.dfeas} =
  \min ( 1, \log \left( \frac{\norm[1]{y_k}}{\sqrt{m}} \right))
  (\pgmid{dy\_tols.cost})(\pgmid{dy_tols.dfeas_scale}).
\end{equation*}

\section{Scaling}
\label{sec:Scaling}

\dylp provides the capability for row and column scaling of the original
LP problem.
This section develops the algebra used for scaling and describes
some additional details of the implementation.
The following section (\secref{sec:Solutions}) describes unscaling in the
context of generating solution, ray, and tableau vectors for the client.

Let $R$ be a diagonal matrix used to scale the rows of the LP problem and $S$
be a diagonal matrix used to scale the columns of the LP problem.
The original problem \eqnref{Eqn:BoundedPrimal} is scaled as
\begin{align*}
 \min \:(cS)(\inv{S}x) & \\
     (RAS)(\inv{S}x) & \leq (Rb) \\
     (\inv{S}l) \leq (\inv{S}x) & \leq (\inv{S}u)
\intertext{to produce the scaled problem}
  \min \breve{c}\breve{x} & & \\
     \breve{A}\breve{x} & \leq \breve{b} \\
     \breve{l} \leq \breve{x} & \leq \breve{u} \\
\end{align*}
where $\breve{A} = RAS$, $\breve{b} = Rb$, $\breve{c} = cS$,
$\breve{l} = \inv{S}l$, $\breve{u} = \inv{S}u$, and $\breve{x} = \inv{S}x$.
\dylp then treats the scaled problem as the original problem.

By default, \dylp will calculate scaling matrices $R$ and $S$ and scale the
constraint system unless the coefficients satisfy the conditions
$.5 < \min_{ij} \abs{a_{ij}}$ and $\max_{ij} \abs{a_{ij}} < 2$.
The client can forbid scaling entirely, or supply a pair of vectors that will
be used as the diagonal coefficients of $R$ and $S$.

A few additional details are helpful to understand the implementation.
The first is that \dylp uses row scaling to convert `$\geq$' constraints to
`$\leq$' constraints.
Given a constraint system with `$\geq$' constraints, \dylp will generate
scaling vectors with coefficients of $\pm 1.0$ even if scaling is otherwise
forbidden.
If scaling is active for numerical reasons, the relevant row scaling
coefficients will be negated.

\dylp scales the original constraint system before generating
logical variables.
Nonetheless, it is desirable to maintain a coefficient of 1.0 for each logical.
The row scaling coefficient $r_{ii}$ for constraint $i$ is already determined.
To keep the coefficients of logical variables at $1.0$, the column
scaling factor is chosen to be $1/r_{ii}$ and the column scaling matrix $S$
is conceptually extended to include logical variables.


\section{Startup}
\label{sec:Startup}

\dylp provides a cold, warm, and hot start capability.
For a cold start, \dylp selects a set of constraints and variables to be the
initial active constraint system and then crashes a basis.
For a warm start, \dylp expects that the caller will supply a basis but assumes
that the active constraint system and other data structures need to be built
to this specification.
For a hot start, \dylp assumes that its internal data structures are valid
except for possible modifications to variable bounds, objective coefficients,
or right-hand-side coefficients.
It will incorporate these modifications and continue with simplex iterations.

\dylp will default to attempting a hot start unless specifically requested
to perform a warm or cold start.
For all three start types, \dylp will evaluate the constraint system for
primal and dual
feasibility, choosing primal simplex unless the constraint system is
dual feasible and primal infeasible.

It is not possible to perform efficient and foolproof checks to
determine if the client has violated the restrictions imposed for a hot start.
At minimum, such a check would require a coefficient by coefficient comparison
of the constraint system supplied as a parameter with the copy held by \dylp
from the previous call.
It is the responsibility of the client to notify \dylp if variable bounds,
objective coefficients, or right-hand-side coefficients have been changed.
\dylp will scan for changes and update its copy of the constraint system
only if the client indicates a change.

Section~\ref{sec:DylpInterface} provides detailed information on the options
used to control \dylp's startup actions.

The startup sequence for \dylp is shown in Figure~\ref{fig:DylpStartupFlow}.
\begin{figure}
\includegraphics{\figures/startupflow}
\caption{\dylp startup sequence} \label{fig:DylpStartupFlow}
\end{figure}
The first actions are determined by the purpose of the call.
The call may be solely to free retained data structures; if so, this is
done and the call returns.
The next action is to determine the type of start --- hot, warm,
or cold --- requested by the client.
If a warm or cold start is requested, any state retained from the previous
call is useless and all retained data structures are freed.
For all three types of start, options and tolerances are updated to reflect
the parameters supplied by the client.

For a warm or cold start, the constraint system is examined to see if it
should be scaled, and the options specified by the client are examined
to see if scaling is permitted.
If this assessment determines that scaling is advisable and permitted, the
constraint system is scaled as described in \secref{sec:Scaling}.
The original constraint system is cached and replaced by the scaled copy.
In the case of a hot start, the existing scaled copy, if present, is retrieved
for use.
The original system is not consulted again until the solution is packaged for
return to the client.

Following scaling, the active constraint system is constructed for a warm or
cold start, or modified for a hot start;
\S\S\ref{sec:ColdStart}~--~\ref{sec:HotStart} describe the actions in
detail.
At the completion of this activity, the active constraint system is assessed
for primal and dual feasibility and an appropriate simplex phase is chosen.

Once the constraint system is constructed, common initialisation
actions are performed:
Data structures are initialised for PSE and DSE pricing, for the
perturbation-based antidegeneracy algorithm, and for the pivot rejection
algorithm.

To complete the startup sequence, \dylp evaluates the constraint system and
client options to determine if it should perform constraint activation
or variable activation or deactivation before starting simplex iterations.
Variable deactivation is mutually exclusive to constraint and variable
activation; the former is considered only during a cold start, the latter
only during a warm or hot start.

An initial round of variable deactivation is performed during a cold start
if the number of active variables exceeds the number specified
by the \pgmid{coldvars} option.
This activity is intended to reduce the initial size of constraint systems
with very large numbers of variables (\eg, set covering formulations).

Constraint or variable activation, or both, are performed during a
warm or hot start if requested by the client.
Constraint activation is performed before variable activation.
If initial constraint activation is requested, \dylp will add all violated
constraints to the active system.
If constraints are added, primal feasibility will be lost, and \dylp will
reassess the choice of initial simplex phase.

If initial variable activation is requested, the action taken depends on the
initial simplex phase.
If \dylp will enter primal simplex, variables with favourable primal reduced
costs are activated, evaluated under the phase~I or phase~II objective as
appropriate.
For dual simplex, variables which will tend to bound the dual problem are
selected for activation:
For each infeasible primal basic variable (nonbasic dual variable with
favourable reduced cost), primal variables with optimal reduced costs
(feasible dual constraints) which will bound motion in the direction of
the incoming dual variable are selected for activation.

\subsection{Cold Start}
\label{sec:ColdStart}

\dylp performs a cold start in two phases.
The first phase, implemented in \pgmid{dy_coldstart}, constructs the initial
active constraint system.
The second phase, implemented in \pgmid{dy_crash}, constructs the initial
basis.

To construct the initial active constraint system, \pgmid{dy_coldstart} first
checks to see if the client has specified that the full constraint system
should be used.
In this case, the active system will be the entire constraint system and the
dynamic simplex algorithm will reduce to a single execution of either primal or
dual simplex.

If the client specifies that \dylp should work with a partial constraint
system, the constraints are first separated into equalities and inequalities.
All equalities are included in the initial active system.

The remaining inequalities are sorted, using the angle of the
constraint normal $a_i$ to the objective function normal $c$ as the figure
of merit,
\begin{equation*}
a_i \angle c = \frac{180}{\pi} \cos^{-1} \frac{a_i \cdot c}{\norm{a_i}\norm{c}}
\end{equation*}
Consider a minimisation objective and `$\leq$' inequalities.
The normals of the inequalities point out of the feasible region, and the
normal of the objective function will point into the feasible region at
optimality.
Hence a constraint whose normal forms an angle near \degs{180} with the normal
of the objective should be more likely to be active at optimum.
A constraint whose normal forms an angle near \degs{0} is more likely to define
a facet on the far side of the polytope.
Unfortunately, `more likely' is not certainty, and it's easy to
construct simple
two-dimensional examples where the normal of one of the constraints active
at optimality forms an acute angle with the normal of the objective function.

\dylp allows the client to specify one or two angular intervals and a sampling
fraction which are used to select inequalities to add to the initial
active system.
By default, the initial system will be populated with 50\% of the
inequalities which form angles in the intervals
$[\degs{0},\degs{90})$ and $(\degs{90},\degs{180}]$.
(\textit{I.e.}, inequalities whose normals are perpendicular to the objective
normal are excluded entirely, and half of all other inequalities will be
added to the initial active system.)
The inequalities selected will be spread evenly across the specified range(s).
\dylp will activate all variables referenced by each constraint.

Once the initial constraint system is populated, \pgmid{dy_crash} is called to
select an initial basis.
\dylp offers three options for the initial basis, called `logical', `slack',
and `architectural'.
A logical basis is the standard unit basis composed of slack and artificial
(logical) variables for the active constraints.
A slack basis again uses slack variables for inequalities, but attempts to
select architectural variables for equalities, including artificial variables
only if necessary.
An architectural basis attempts to choose architectural variables for all
constraints, selecting slack and artificial variables only when necessary.

There are many qualities which are desirable in an initial basis, and they
are often in conflict.
A logical basis is trivially easily to construct, factor, and invert, and
has excellent numerical stability.
On the other hand, such a basis is hardly likely to be the optimal basis.
When choosing architectural variables, free variables are highly desirable
since they will never leave the basis.
In addition, \dylp's basis construction algorithm tries to select architectural
variables which will form an approximately lower-diagonal matrix and
provide numerically stable pivots.
Constructing a matrix which is approximately lower-diagonal minimises fill-in
when the basis is factored.
Several of the ideas implemented in \dylp's initial basis
construction algorithms are described by Bixby in~\cite{Bix92a}.

Since \dylp makes an effort to populate the constraint system with
constraints that should be tight at optimality, an architectural basis
is the default.

\subsection{Warm Start}
\label{sec:WarmStart}

The routine \pgmid{dy_warmstart} implements a warm start.
The client is expected to supply an initial basis,
expressed as a set of active constraints and corresponding basic variables.
By default, \dylp will activate all variables referenced by each constraint.
As an option, the client can specify an initial set of active variables.

\subsection{Hot Start}
\label{sec:HotStart}

For a hot start, \dylp assumes that all internal data structures are exactly as
they were when it last returned to the client.
Changes to the constraint system must be confined to the
right-hand-side, objective, and variable upper and lower bound vectors,
so that the basis factorisation and inverse are not affected.
The client is responsible for indicating to \dylp which of these vectors have
been changed.
The routine \pgmid{dy_hotstart} scans the changed vectors and orchestrates
any updates to the corresponding data structures in the active constraint
system.
Unlike a cold or warm start, the basis is \textit{not} factored prior to
resuming pivots.
\dylp assumes that the basis was refactored as part of the normal preoptimality
sequence prior to the last return to the client and that no intervening
pivots have occurred.
Any numerical problems arising from the modifications specified by the
client will be picked up in the normal course of dynamic simplex execution.


\section{Dynamic Simplex}
\label{sec:DynamicSimplex}

\subsection{Normal Algorithm Flow}
\label{sec:NormalDynamicFlow}

Figure~\ref{fig:DylpFlow} gives the normal flow of the dynamic simplex
algorithm implemented in \dylp.
\begin{figure}
\centering
\includegraphics{\figures/dylpnormalflow}
\caption{Dynamic Simplex Algorithm Flow} \label{fig:DylpFlow}
\end{figure}
The outcomes included in the normal flow of the algorithm are primal
optimality, infeasibility, and unboundedness, and dual optimality and
unboundedness.
Other outcomes (\eg, loss of dual feasibility during dual simplex, or
numerical instability) are discussed in \secref{sec:ErrorRecovery}.

The implementation of the dynamic simplex algorithm is structured as
a finite state machine, with six normal states,
primal simplex, dual simplex,
deactivate variables, activate variables,
deactivate constraints, and activate constraints;
two user-supplied states, generate variables and generate constraints;
and three error recovery states,
force primal feasibility, force dual feasibility,
and force full constraint system.
State transitions are determined by the previous state,
the type of simplex in use, and the outcome of actions in a state.

As described in \secref{sec:Startup}, \dylp establishes an initial active
constraint system, determines whether the system is primal or dual feasible,
and chooses the appropriate simplex as the starting phase.

The most common execution pattern is as described in the Introduction:
The initial active constraint system is neither primal or dual feasible.
Primal simplex is used to solve this system to optimality.
A minor loop then activates variables with favourable reduced cost and
reoptimises using primal phase~II\@.
This loop repeats until no variables can be activated; at this point the
solution is optimal for the active constraints, over all variables.
The algorithm then attempts to activate violated constraints; if none are
found, the solution is optimal for the original problem.
After violated constraints are activated, loose constraints are deactivated
and dual simplex is used to reoptimise.
When an optimal solution is reached, the algorithm attempts to activate
variables with favourable reduced cost and return to the
`primal phase~II -- activate variables' minor loop.
If no variables can be activated, the algorithm attempts to activate violated
constraints.
If none are found, the solution is optimal for the original problem.
If violated constraints are activated, then an attempt is made to activate
dual feasible variables and dual simplex is used to reoptimise.

There is an obvious asymmetry in the use of primal and dual simplex.
When primal simplex reaches an optimal solution, the
`primal phase~II -- activate variables' minor loop iterates until no useful
variables remain to be activated.
Only then does the algorithm activate violated constraints and move to dual
simplex.
The analogous minor loop for dual simplex would be to add violated constraints
(dual variables with favourable reduced costs) and reoptimise with dual
simplex until no violated constraints remain.
Instead, the algorithm attempts to add variables and return to primal simplex;
failing that, it will add both violated constraints and dual feasible
variables (satisfied dual constraints).
The purpose of this asymmetry is two-fold:
It acknowledges that primal infeasibility is much more likely than primal
unboundedness when solving LPs in the context of a branch-and-cut algorithm,
and it attempts to avoid the large swings in the values of primal variables
which often accompany dual unboundedness.
Dual simplex moves between primal infeasible basic
solutions which can be at a large distance from the primal feasible region and
at a large distance from one another in the primal space.
This presents a challenge for numerical stability.
Because the primal simplex remains within the primal feasible region,
primal unboundedness does not present the same difficulty.

To avoid cycling by repeatedly deactivating and reactivating the same
constraint when the dimension of the optimal face is greater than one,
constraint deactivation is skipped unless there has been an
improvement in the objective function since the previous constraint
deactivation phase.
This guarantees that the simplex will not return to a previous extreme
point.

If primal simplex finds that the active system is infeasible, the algorithm
will attempt to activate variables with favourable reduced cost under
the phase~I objective function (\vid \secref{sec:PrimalSimplex}) and
resume primal phase~I\@.
If no variables can be found, the original problem is infeasible.

If primal simplex finds that the active system is unbounded, the algorithm
first attempts to activate bounding constraints which will not cause the loss
of primal feasibility.
If such constraints can be found, execution returns to primal phase~II\@.
If no such constraints can be found, or primal feasibility is not an issue,
all violated constraints are added and execution moves to dual simplex.
If no violated constraints can be found, the full constraint system is
activated.
If primal simplex again returns an indication of unboundedness, the original
problem is declared to be unbounded.
The effort expended before indicating a problem is unbounded acknowledges
that unboundedness is expected to be extremely rare in \dylp's
intended application.

If dual simplex finds that the active system is dual unbounded (primal
infeasible), the algorithm first attempts to activate dual bounding
constraints (primal variables) which will not cause the loss of dual
feasibility.
If such dual constraints can be found, execution returns to dual simplex.
If no such dual constraints can be found, the algorithm will attempt to
activate variables with favourable reduced cost under the primal phase~I
objective function and continue with primal phase~I\@.

\subsection{Error Recovery}
\label{sec:ErrorRecovery}

A substantial amount of \dylp's error recovery capability is hidden within the
primal and dual simplex algorithms.
It is also possible to use the capabilities present in a dynamic
simplex algorithm to attempt error recovery at this level.
The dynamic simplex algorithm modifies the constraint system as part
of its normal execution.
This ability can be harnessed to force a transition from one simplex to
another when one simplex runs into trouble.
The actions described in this section are fully integrated with the actions
described in \secref{sec:NormalDynamicFlow}.
They are described separately to avoid reducing Figure~\ref{fig:DylpFlow} to
an incomprehensible snarl of state transitions.

\noindent
\textbf{Primal Simplex}

The error recovery actions associated with the primal simplex algorithm are
shown in Figure~\ref{fig:DynErrRecPrimal}.
\begin{figure}
\centering
\includegraphics{\figures/primalerrorflow}
\caption{Error Recovery Actions for Primal Simplex Error Outcomes}
\label{fig:DynErrRecPrimal}
\end{figure}
There are five conditions of interest, excessive change in the value of primal
variables (excessive swing), stalling (stall), inability to perform a pivot
(punt), numerical instability (accuracy check), and other errors (other
error).

Excessive change (`swing') in the value of a primal variable during primal
simplex is taken as an
indication that the primal problem is verging on unboundedness.
Swing is defined as $(\textrm{new value})/(\textrm{old value})$.
\dylp's default tolerance for this ratio is $10^{15}$.
The action taken is the same as that used for normal detection of
unboundedness, with the exception that the algorithm will always return to
primal simplex.

When primal simplex stalls or is forced to punt, the strategy is to attempt to
modify the constraint system so that the simplex algorithm will be able to
choose a new pivot and again make progress toward one of the standard outcomes
of optimality, infeasibility, or unboundedness.
The specific actions vary slightly depending on whether primal feasibility has
been achieved.

If primal simplex is still in phase~I, the first action is to try to activate
variables which have a favourable reduced cost under the phase~I objective.
If this succeeds, execution returns to primal simplex.
If no variables can be found, the algorithm will attempt to activate violated
constraints; if successful, execution returns to primal simplex.
If no variables or constraints have been activated, there is no point in
returning to primal simplex as the outcome will be unchanged.
In this case, the algorithm will attempt to force dual feasibility by
deactivating variables whose reduced costs are not dual feasible (\ie,
deactivate unsatisfied dual constraints).
If this succeeds, the algorithm will deactivate loose constraints (dual
variables) to reduce the chance of dual unboundedness and continue with dual
simplex.
Failing all the above, the ultimate action is to active the full constraint
system and attempt to solve it with primal or dual simplex.
This can be done only once, to avoid a cycle in which the full system is
activated, pared down while forcing primal or dual feasibility, and then
reactivated when lesser measures again fail.

When a stall or punt occurs in primal phase~II, the first action is again to
attempt to activate variables with a favourable reduced cost.
However, if no new variables can be found, the algorithm immediately attempts
to force dual feasibility.
Only if this can be achieved will it proceed to activate violated constraints,
deactivate loose constraints, and proceed to dual simplex.
Failure to force dual feasibility or to activate any constraints causes forced
activation of the full constraint system as described above.

Both the primal and dual simplex algorithm incorporate extensive checks and
error recovery actions to detect and recover from numerical instability.
By the time a simplex gives up and reports that it cannot overcome numerical
problems, there is little to be done but force activation of the full
constraint system for one last attempt.

Other errors indicate algorithmic failures within the simplex algorithms (\eg,
failure to acquire resources, or conditions not anticipated by the code)
and no attempt is made to recover at the dynamic simplex level.

\noindent
\textbf{Dual Simplex}

The error recovery actions associated with the dual simplex algorithm are
shown in Figure~\ref{fig:DynErrRecDual}.
\begin{figure}
\centering
\includegraphics{\figures/dualerrorflow}
\caption{Error Recovery Actions for Dual Simplex Error Outcomes}
\label{fig:DynErrRecDual}
\end{figure}
In addition to the five outcomes cited for primal simplex, loss of dual
feasibility (lost dual feasibility) can be reported by the dual simplex
algorithm.
(Loss of primal feasibility is handled internally by the primal simplex, which
simply returns to phase~I simplex iterations.)

When the dual simplex algorithm loses feasibility, the algorithm will attempt
to force dual feasibility by deleting the offending dual constraints (primal
variables).
If this succeeds, it will attempt to activate feasible dual constraints and
return to dual simplex.
If dual feasibility cannot be restored, the algorithm attempts to activate
variables with favourable reduced costs under the primal phase~I objective and
executes primal phase~1\@.

Excessive change in the value of primal variables during dual simplex is taken
as an indication that the dual algorithm is moving between basic solutions
which are far outside the primal feasible region and far from each other.
When excessive change in a primal variable is detected, the algorithm
attempts to activate primal constraints which will bound this motion.
If this is successful, execution of dual simplex resumes.
General activation of violated primal constraints is not attempted as it is
less likely to bound the primal swing.
If no bounding constraints can be found, the algorithm attempts to activate
feasible dual constraints and return to dual simplex.
If no such constraints can be found, 
the algorithm attempts to activate
variables with favourable reduced costs under the primal phase~I objective and
executes primal phase~1\@.

When dual simplex reports that it has stalled or cannot execute necessary
pivots, the algorithm first attempts to activate violated primal constraints.
If such constraints can be activated, execution returns to dual simplex.
If no constraints can be found, the algorithm attempts to force primal
feasibility by deactivating violated primal constraints.
Depending on the result of this action, the algorithm attempts to activate
variables with favourable reduced costs under the primal phase~I or phase~II
objective and executes primal simplex.

Loss of numerical stability and other errors are handled as for primal
simplex.

\section{Dual Simplex}

\dylp will choose dual simplex whenever the current basic solution is dual
feasible but not primal feasible.
The primary role of dual simplex in \dylp is reoptimisation following
the addition of violated constraints.
The implementation reflects this role and does not provide a dual phase~I for
achieving dual feasibility.
The dual simplex implementation incorporates dual steepest edge (DSE)
pricing (\secref{sec:DSEPricing}),
standard (\secref{sec:DualStdSelectInVar}) and
generalised (\secref{sec:DualGenSelectInVar}) pivoting, and
perturbation-based (\secref{sec:PerturbedAntiDegeneracy}) and
alignment-based (\secref{sec:AntiDegenLite})
antidegeneracy algorithms.

Because the dual simplex implementation does not provide a phase~I,
a number of exceptional conditions will cause
\dylp fall back from dual simplex to primal simplex.

In dynamic simplex, apparent primal infeasibility can result because only a
subset of the variables are present in the active constraint system.
In some cases, the variables needed to regain feasibility cannot be activated
into the nonbasic partition while maintaining dual feasibility.
In the context of the dual problem, the problem is unbounded and any
dual constraint which would bound it would also make the current basic
solution dual infeasible.
\dylp implements a variable activation procedure which can pivot a
single variable into the basis as it is activated in order to maintain dual
feasibilty.
It is still possible, however, to reach a basic solution where multiple
pivots are required to regain dual feasibility for any candidate variable.
When this occurs, \dylp reverts to primal simplex.

If primal infeasible variables remain but they cannot be pivoted because
their pivot coefficients do not satisfy the current pivot selection tolerances,
\pgmid{dy_dual} will punt and \dylp will return to phase~I of the primal
simplex algorithm in the hope that addition of variables and/or the
application of primal pivoting rules will allow pivoting to continue.
In addition, if the dual simplex terminates due to stalling or loss of
feasibility, \dylp will try the primal simplex algorithm before giving up.

Figure \ref{fig:DualCallGraph} shows the call structure of the dual simplex
implementation.
\begin{figure}[htb]
\centering
\includegraphics{\figures/dualcalls}
\caption{Call Graph for Dual Simplex} \label{fig:DualCallGraph}
\end{figure}

\subsection{Dual Top Level}

Dual simplex is executed when the dynamic simplex state machine enters state
\pgmid{dyDUAL}.
If required, DSE pricing is initialised by calculating the square of the
norms of the rows of the basis inverse (\vid \secref{sec:DSEPricing})
and the dual simplex routine \pgmid{dy_dual} is called.
\pgmid{dy_dual} is a trivial shell which calculates
the objective (\pgmid{dy_calcobj}) and calls the dual phase~II
routine \pgmid{dual2} to do the optimisation.

\subsection{Dual Phase II}

The overall flow of phase~II of the dual algorithm is shown in Figure
\ref{fig:DualPhaseIIFlow}.
\begin{figure}[htbp]
\begin{center}
\scalebox{.9}{\includegraphics{\figures/dual2flow}}
\end{center}
\caption{Dual Phase II Algorithm Flow} \label{fig:DualPhaseIIFlow}
\end{figure}
The body of the routine is structured as two nested loops.
The outer loop handles startup and termination, and the inner loop handles
the majority of routine pivots.

On entry to \pgmid{dual2}, the outer loop is entered and \pgmid{dy_dualout}
is called to select the initial leaving variable.
Then the inner loop is entered and \pgmid{dy_dualpivot} is called to perform
the pivot.
\pgmid{dy_dualpivot} (\vid \secref{sec:DualPivoting}) will
calculate the coefficients of the pivot row (\pgmid{dualpivrow}),
select an entering variable (\pgmid{dualin}),
pivot the basis (\pgmid{dy_pivot}),
update the primal and dual variables (\pgmid{dualupdate}),
and update the DSE pricing information and reduced costs (\pgmid{dseupdate}).
For a routine pivot, \pgmid{dseupdate} will also select a leaving variable
for the next pivot.
\pgmid{dy_duenna} evaluates the outcome of the pivot, handles error detection
and recovery where possible, and performs the routine maintenance activities
of accuracy checks and refactoring of the basis.
If there are no problems, the pivoting loop iterates, using the leaving
variable selected in \pgmid{dseupdate}.
The loop continues until optimality is reached, the problem is determined to
be primal infeasible (dual unbounded), or an exception or fatal error occurs.

One common reason for a failure to select a leaving variable for the next
pivot is that the current pivot was aborted due to numerical problems
(an unsuitable pivot coefficient being the most common of these).
In this case, \pgmid{dseupdate} never executes.
Once \pgmid{dy_duenna} has taken the necessary corrective action, the flow
of control escapes to the outer loop and calls \pgmid{dy_dualout} to select
a new leaving variable.

Another common reason for failure to select a leaving variable is that all
candidates were previously flagged as unsuitable pivots.
In this case, \pgmid{dy_dualout} will indicate a `punt' and
\pgmid{dy_dealWithPunt} will be called to reevaluate the flagged variables.
If it is able to make new candidates available, control returns to
\pgmid{dy_dualout} for another attempt to find a leaving variable.
If all flagged variables remain unsuitable, control flow moves to the
preoptimality actions with an indication that dual simplex has punted.

When \pgmid{dy_dualout} indicates optimality (primal feasibility) or
\pgmid{dy_dualpivot} indicates optimality, dual unboundedness (primal
infeasibility), or loss of dual feasibility,
the inner loop ends and \pgmid{preoptimality} is called for confirmation.
\pgmid{preoptimality} will refactor the basis, check for accuracy, recompute
the primal and dual variables, and confirm dual and primal feasibility status.
If there are no surprises, dual phase~II terminates with an indication of
optimality, dual unboundedness, or loss of dual feasibility.

Loss of dual feasibility stems from loss of numeric accuracy, but it cannot
be corrected within dual phase~II.
The error recovery actions taken by the dynamic simplex algorithm are described
in \secref{sec:ErrorRecovery}.

Loss of primal feasibility can occur for two distinct reasons.
In the less common case, loss of primal feasibility stems from loss of numeric
accuracy.
The pivot selection rules are tightened and dual simplex iterations
are resumed.
When the number of false indications of optimality exceeds a hard-coded limit
(currently 15), dual simplex terminates with a fatal error.

The more common reason for apparent loss of primal feasibility at the
termination of dual simplex is that it is ending with a punt, as
described above.
The variables flagged as unsuitable for pivoting are not primal feasible, and
when the flags are removed to perform the preoptimality checks, primal
feasibility is revealed as an illusion.
No further action is possible within dual simplex; the reader is again referred
to \secref{sec:ErrorRecovery}.

Other errors (\eg, stalling, accuracy checks, \etc) not shown in
Figure~\ref{fig:DualPhaseIIFlow} can occur and result in termination of the
dual simplex algorithm with the appropriate error indication.

\subsection{Pivoting}
\label{sec:DualPivoting}

\dylp offers two flavours of dual pivoting: A standard dual pivot algorithm in
which a single primal variable is selected and pivoted into the basis, and a
generalised dual pivot algorithm \cite[\S10.2]{Mar03} in which multiple primal
variables may undergo bound-to-bound flips prior to the basis pivot.
The choice of standard or generalised dual pivoting can be controlled with an
option; \dylp will use generalised pivoting by default.

Figure~\ref{fig:DualPivotCallGraph} shows the call structure of the dual
pivot algorithm.
The routine \pgmid{dualin} implements standard dual pivoting;
\pgmid{dualmultin} implements generalised dual pivoting.

\begin{figure}[htb]
\centering
\includegraphics{\figures/dualpivcalls}
\caption{Call Graph for Dual Pivoting} \label{fig:DualPivotCallGraph}
\end{figure}

The first activity in \pgmid{dy_dualpivot} is the calculation of the
coefficients of the pivot row, $\overline{a}_{i} = \beta_i N$, by the routine
\pgmid{dualpivrow}.
With the leaving primal variable and the basis inverse row in hand, one of
\pgmid{dy_dualin} or \pgmid{dualmultiin} are called to select the entering
variable.
(If generalised dual pivoting is in use, \pgmid{dualmultiin} will perform
any bound-to-bound flips before returning.)

Once the entering and leaving variables have been chosen, the actual pivot
is performed in several steps.
Prior to the pivot, the vector $\tau = B^{\,-1}\beta^T_k$ is calculated for
use during the update of the DSE pricing information.
The basis is pivoted next; this involves calls to \pgmid{dy_ftran} and
\pgmid{dy_pivot}, as outlined in \secref{sec:BasisPivoting}.
If the basis change succeeds, the primal and dual variables are updated by
\pgmid{dualupdate} using the iterative update formul\ae{} of
\secref{sec:UpdatingFormulas}, and then the DSE pricing information
and reduced
costs are updated by \pgmid{dseupdate}, using the update formul\ae{} of
\secref{sec:DSEPricing}.
As a side effect, \pgmid{dseupdate} will select a leaving variable for the
next pivot.

\subsection{Selection of the Leaving Variable}

The selection of the leaving primal variable $x_i$ (entering dual
variable $y_i$) is made using the dual steepest edge criterion
described in \secref{sec:DSEPricing}.
As outlined above, the normal case is that the leaving variable for the
following pivot will
be selected as \pgmid{dseupdate} updates the DSE pricing information
for the current pivot.
In various exceptional circumstances where this does not occur, the routine
\pgmid{dy_dualout} is called to make the selection.

\subsection{Standard Dual Pivot}
\label{sec:DualStdSelectInVar}

For the standard dual pivot algorithm, the selection of the entering
primal variable (leaving dual variable) is
made using the usual dual pivoting rules and a set of tie-breaking strategies.

Let $x_i$ be the leaving primal variable, for simplicity of exposition
occupying basis position $i$.
$\beta_{i}$ is obtained by calling \pgmid{dy_btran} to calculate
$e_{i}B^{\,-1}$, where $e_{i}$ is the unit row vector
with $1$ in position~${i}$.
The pivot coefficient for a variable $x_k$ is
$\overline{a}_{ik} = \beta_i a_k$.
Let $y_i$ be the dual variable associated with the constraint in basis
position $i$ and
let $y_k$ be the dual variable associated with the tight bound constraint
for the nonbasic primal variable $x_k$.

Abstractly, we need to check
$y_k = \overline{c}_k + \overline{a}_{ik}\delta_{ik}$ to
find the maximum allowable $\delta_{ij}$ such that
$y_k \geq 0 \: \forall k \in \mathcal{B}$ and $y_j = 0$ for some $j$.
The index $j$ of the entering primal variable $x_j$ will be
\begin{equation} \label{eq:AbstractDualPivot}
j = \arg \mathop{\min}_{k} \abs{\frac{\overline{c}_k}{\overline{a}_{ik}}}
\end{equation}
for suitable $x_k \in N$.

In practice, it's impossible to explain `suitable $x_k$'
properly without going deep into the details of the workings of the
revised dual simplex algorithm (\vid \cite{Haf98a}).
Table~\ref{Tbl:DualPivotRules} gives the rules in tabular form, from the
perspective that when all is said and done, the leaving primal variable must
end up nonbasic at bound and the sign of the reduced cost must be appropriate
for that bound.
\begin{table}
\renewcommand{\arraystretch}{2.5}\setlength{\tabcolsep}{.75\tabcolsep}
\begin{tabular}{*{7}{>{$}c<{$}}>{$\displaystyle}c<{$}}

\text{leaving } x_i & \text{entering } y_i & \text{resulting } \overline{c}_i &
\text{entering } x_j & \text{leaving } y_j & \text{initial } \overline{c}_j &
\text{pivot } \overline{a}_{ij} &
-\frac{\overline{c}_j}{\overline{a}_{ij}} = \overline{c}_i \\[.5\baselineskip]

\nearrow \mathit{lb} & 0 \nearrow & \geq 0 &
\mathit{lb} \nearrow & \searrow 0 & \geq 0 &
< 0 & -\frac{(+)}{(-)} = (+) \\

& & & \mathit{ub} \searrow & \nearrow 0 & \leq 0 &
> 0 & -\frac{(-)}{(+)} = (+) \\

\searrow \mathit{ub} & 0 \searrow & \leq 0 &
\mathit{lb} \nearrow & \searrow 0 & \geq 0 &
> 0 & -\frac{(+)}{(+)} = (-) \\

& & & \mathit{ub} \searrow & \nearrow 0 & \leq 0 &
< 0 & -\frac{(-)}{(-)} = (-) \\
\end{tabular}
\caption{Summary of Dual Simplex Pivoting Rules} \label{Tbl:DualPivotRules}
\end{table}
Interpreting the table, the second line says that if the leaving variable
will be made primal feasible by rising to its lower bound, the resulting
reduced cost must be positive to retain primal optimality, hence the
corresponding dual variable must enter by rising from zero.
If the entering primal variable will be decreasing from its upper bound, the
current reduced cost must be negative, hence the corresponding dual variable
must leave by rising to zero\footnote{%
Properly accounting for these \textit{apparently} negative dual variables is
the difficulty in trying to explain pivoting from the dual simplex perspective.
In fact, negative dual variables are an artifact of running the dual simplex
algorithm using representation and data structures appropriate for primal
simplex with implicit bound constraints.}.
The final columns simply illustrate that the sign of the pivot is well-defined
from the update formula.

\dylp provides a selection of tie-breaking strategies when there are multiple
candidates with equal $\abs{\delta_{ik}} = \delta_{\mathrm{min}}$.
The simplest is to select the first variable $x_k$ such that $\delta_{ik} = 0$.
A slightly more sophisticated strategy is to scan all variables $x_k$
eligible to enter and pick $x_j$ such that
$j = \arg \max_{k \in K} \abs{\overline{a}_{ik}}$,
$K = \{ k \mid \abs{\delta_{ik}} =  \delta_{\mathrm{min}} \}$;
\dylp will use this strategy by default.
\dylp also provides four additional strategies based on hyperplane alignment
as described in \secref{sec:AntiDegenLite}.
An option allows the tie-breaking strategy to be selected by the client.

In case of degeneracy, the perturbed subproblem anti-degeneracy algorithm
described in \secref{sec:PerturbedAntiDegeneracy} is also available.
The client can control the use of perturbed subproblems through two options
which specify whether a perturbed subproblem can be used, and how many
consecutive degenerate pivots must occur before the perturbed subproblem
is created.
By default, \dylp uses perturbed subproblems aggressively and will
introduce one when faced with a second consecutive degenerate pivot.

\subsection{Generalised Dual Pivot}
\label{sec:DualGenSelectInVar}

Suppose that an entering dual variable $y_i$ has been chosen, and the ratio
test of equation
\eqnref{eq:AbstractDualPivot} has been used to select a leaving variable
$y_j$ and determine the change $\delta_{ij}$ in $y_i$ required to drive
$y_j = \overline{c}_j$ to zero.
Generalised dual pivoting asks the question ``What happens when we push
past this limit?''

Immediately, dual feasibility is lost as the value of $y_j$ changes sign.
But $\ldots$ suppose that the corresponding nonbasic primal variable $x_j$
has both an upper and lower bound.
If the value of this variable is changed to the opposite bound (`flipped'),
the sign of $y_j$ is
again correct with respect to the value of $x_j$ and dual feasibility is
restored.
Flipping $x_j$ will change the value of any basic primal variable $x_k$
where $\overline{a}_{kj} = \beta_k a_j \neq 0$.
In particular, the value of $x_i$ will move toward feasibility.
In terms of dual simplex, the reduced cost $\overline{b}_i = x_i$ of $y_i$
will be reduced.
If $\overline{b}_i$ is not yet reduced to zero, $y_i$ can still be used as
the entering dual variable (albeit with a less favourable reduced cost) and
the ratio test can be repeated to determine
a new leaving dual variable $y_{j'}$.
Repeating this procedure will identify a maximum sequence of primal variable
flips.
The sequence ends for one of two reasons:
\begin{itemize}
  \item
  The primal variable $x_{f}$ associated with a dual variable $y_f$ has
  only one finite bound and cannot be flipped.

  \item
  Flipping the primal variable $x_f$ will push $x_i$ over its
  bound and into feasibility.
  In dual simplex terms, $y_i$ will acquire an unfavourable reduced cost
  and will no longer be a suitable choice for the entering dual variable.
\end{itemize}
The dual variable $y_f$ corresponding to
$x_f$ becomes the leaving dual variable.
The dual basis pivot will have $y_f$ leaving and $y_i$ entering; the
corresponding primal pivot has $x_i$ leaving and $x_f$ entering.
This sequence of primal variable flips culminating in a final pivot is
generalised dual pivoting.
Note that it's possible to choose any variable within the maximum sequence of
flips and use it as the pivot variable.

\dylp implements generalised dual pivoting by first collecting the set
of potential leaving dual variables $y_k$ (and associated entering primal
variables $x_k$).
This set is then sorted using nondecreasing value of $\abs{\delta_{ik}}$ 
and numerical stability of the pivot as the primary and secondary sort criteria.
(Numerical stability is a binary condition for this purpose; a pivot is either
acceptable or not.)
The tertiary sort criterion varies according to whether $\delta_{ik} = 0$
or $\delta_{ik} \neq 0$.
\begin{itemize}
  \item
  For variables with $\delta_{ik} = 0$, give preference to primal variables
  which can be flipped to their opposite bound.

  \item
  For variables with $\delta_{ik} \neq 0$, give preference to variables which
  cannot be flipped.
\end{itemize}
Any remaining ties are broken with a preference for pivot coefficients with
better numerical stability (compared as an analog value).
This final tie-breaking criterion is important when flipping a sequence
of variables because numerical
stability is relative to the largest coefficient value
$\abs{\overline{a}_{iq}} = \max_k \abs{\overline{a}_{ik}}$
in a column.
An unstable pivot has a small ratio
$\abs{\overline{a}_{ik}/\overline{a}_{iq}}$; this implies a high probability
that when $x_k$ is flipped, other basic primal variables (at the least, $x_q$)
will incur large changes.
Stability of primal variable values is thus improved by preferring large
pivot coefficients.

A nondegenerate dual pivot is clearly preferable to a degenerate pivot, and
this motivates the preference for flippable variables within the set of
candidates with $\delta_{ik} = 0$.
Ideally, all variables in this group can be flipped; failing this, it's
preferable to flip as many as possible.
When consideration moves into the group of candidates with
$\delta_{ik} \neq 0$, the goal changes.
Quick selection of a good pivot will minimise further unpredictable changes
to other dual reduced costs (primal basic variables).
Since pivoting is the goal, it is reasonable to give preference to variables
that must be pivoted.

The process of scanning for candidates and sorting the resulting set is
implemented in the routines \pgmid{scanForDualInCands} and
\pgmid{dualcand_cmp}.

The sorting procedure just described may result in an ordered list where
one or more unflippable candidates $y_u$ with numerically unstable pivots
$\overline{a}_{iu}$ precede the first candidate $y_s$ with a stable
pivot $\overline{a}_{is}$.
In this case, a final attempt is made to promote the candidate with a
stable pivot so that it precedes the the unsuitable candidates $y_u$.
From the sort criteria, it must be that
$\abs{\delta_{is}} \geq \abs{\delta_{iu}}$.
For a given variable $y_u$, if $\abs{y_u - \overline{a}_{iu} \delta_{is}}$
is less than the dual feasibilty tolerance, the resulting dual infeasibility
will be tolerable and $y_s$ can be promoted over $y_u$.
This promotion of a stable pivot over an unstable pivot is implemented in
\pgmid{promoteSanePivot}.

At the end of the above sort algorithm, the list of candidates is ordered so
that it begins with a maximum sequence of flippable variables, followed by a
variable which must be pivoted.
The routine \pgmid{selectWithoutInf} scans the sorted list and selects the
actual pivot variable according to the criteria specified above for a maximum
sequence of flips and final pivot.

\dylp implements one additional experimental capability within generalised dual
pivoting.
As mentioned above, flipping nonbasic primal variables will, in general,
change the values of an arbitrary set of the basic primal variables.
It is possible, but expensive, to track this change; the major
cost is the calculation of $\overline{a}_k = B^{\,-1} a_k$ for each candidate
column.
With this information in hand, it is possible to locate, within the sequence of
variables eligible to be flipped or pivoted, the point at which the maximum
primal infeasibility is at a minimum over the basic variables; this variable
becomes the pivot variable.
This method of selecting the pivot variable is implemented in the routine
\pgmid{selectWithInf}.

Computational experience shows that using the minimum maximum
primal infeasibility to choose the pivot variable $x_f$ is not a good
strategy when dual simplex is behaving well.
Dual simplex moves through a sequence of primal infeasible basic solutions.
Observation of dual simplex in operation often shows a pattern where the values
of primal variables grow increasingly infeasible and then, within a relatively
few pivots, collapse to feasibility (hence optimality).
Attempting to suppress the initial growth of primal infeasibility is
counterproductive, lengthening the sequence of pivots required to attain
optimality.
However, very large infeasible primal values present challenges to numerical
accuracy, so that it may be desirable in extreme cases to choose pivots
with a goal of reducing primal infeasibility.

\dylp by default implements a flexible strategy which normally chooses
the maximum sequence of flips followed by a final pivot (\ie, the pivot is
chosen to maximise the improvement in the dual objective).
If it detects that the magnitude of the primal variables has grown to a point
where numerical accuracy may be compromised, it will switch to choosing the 
pivot variable to minimise the maximum infeasibility over the primal variables.

The strategy used for generalised dual pivoting is controlled by the same
option used to choose between standard and generalised dual pivoting.
The complete set of options is standard dual pivoting;
generalised dual pivoting to maximise dual objective improvement;
generalised dual pivoting to minimise maximum primal infeasibility;
and the flexible generalised strategy used as the default.

Antidegeneracy using perturbed subproblems is used with generalised dual
pivoting.
The alignment-based anti-degeneracy strategies are not implemented.


\section{Primal Simplex}
\label{sec:PrimalSimplex}

The primal simplex implementation in \dylp is a two-phase algorithm.
\dylp will choose primal simplex phase~II whenever the current basic
solution is primal feasible but not dual feasible.
It will choose primal simplex phase~I when the current basic solution
is neither primal or dual feasible.
The primary role of primal simplex in \dylp is to reoptimise following the
addition of variables.
Since primal phase~I requires neither primal or dual feasibility, it is the
fallback simplex.

The primal simplex implementation incorporates projected steepest edge (PSE)
pricing (\secref{sec:PSEPricing}),
standard (\secref{sec:PrimalStdSelectOutVar}) and
generalised (\secref{sec:PrimalGenSelectOutVar}) pivoting, and
perturbation-based (\secref{sec:PerturbedAntiDegeneracy}) and
alignment-based (\secref{sec:AntiDegenLite}) antidegeneracy algorithms.

Figure \ref{fig:PrimalCallGraph} shows the call structure of the primal simplex
implementation.
\begin{figure}[htb]
\centering
\includegraphics{\figures/primalcalls}
\caption{Call Graph for Primal Simplex} \label{fig:PrimalCallGraph}
\end{figure}

\subsection{Primal Top Level}

Primal simplex is executed when the dynamic simplex state machine enters one
of the states \pgmid{dyPRIMAL1} or \pgmid{dyPRIMAL2}.
If required, the PSE reference frame is initialised to the
nonbasic variables and the projected column norms are initialised to one
(\vid \secref{sec:PSEPricing}), and
the primal simplex routine \pgmid{dy_primal} is called.

\pgmid{dy_primal} controls the use of phase~I (\pgmid{primal1}) and
phase~II (\pgmid{primal2}) of the primal simplex algorithm.
The primary purpose of \pgmid{dy_primal} is to provide a loop which allows a
limited number (currently hardwired to 10) of reversions
to phase~I if primal feasibility is lost during phase II.
Loss of primal feasibility is treated as a numeric accuracy problem; with each
such reversion the minimum pivot selection tolerances are tightened by one
step.

To maintain primal feasibility when repairing a singular basis
(\secref{sec:BasisFactoring})
in primal phase~II, superbasic variables may be created.
Superbasic variables will not normally be created during phase~I and the
code assumes that it will not encounter them\footnote{%
More strongly, superbasic variables are introduced only in primal phase~II for
the purpose of maintaining feasibility during repair of a singular basis.
They will appear outside of \pgmid{primal2} only if the problem is unbounded
or if \pgmid{primal2} terminates with an error condition.}.
Rarely, a sequence of errors during phase~II will cause \dylp to lose
primal feasibility and revert to phase~I with superbasic variables still
present in the nonbasic partition.
The routine \pgmid{forcesuperbasic} is called to ensure that any superbasic
variables are forced to bound in such a phase~II to phase~I transition.

\subsection{Primal Phase I}
\label{sec:PrimalPhaseI}

The overall flow of phase I of the primal simplex is shown in Figure
\ref{fig:PrimalPhaseIFlow}.
\begin{figure}[htbp]
\centering
\resizebox{\linewidth}{!}{\includegraphics{\figures/primal1flow}}
\caption{Primal Phase I Algorithm Flow} \label{fig:PrimalPhaseIFlow}
\end{figure}
The body of the routine is structured as two nested loops.
The outer loop handles startup and termination, and the inner loop handles the
majority of routine pivots.
A pivot iteration in phase~I normally consists of three steps: the actual
pivot and variable updates, routine maintenance checks, and revision of
the objective.

A dynamically modified artificial objective is used to guide
pivoting to feasibility during phase~I\@.
The (minimisation) coefficients assigned to variables are
-1 for variables below their bound, 0 for variables within bounds, and +1 for
variables above their bound.
On entry to phase~I, \pgmid{dy_initp1obj} forms a working set containing
all infeasible variables, constructs the corresponding objective, swaps out
the original objective, and installs the phase~I objective.

Once the phase~I objective has been constructed, the outer loop is entered and
\pgmid{dy_primalin} is called to select the initial entering variable.
Then the inner loop is entered and \pgmid{dy_primalpivot} is called to
perform the pivot.
\pgmid{dy_primalpivot} (\vid \secref{sec:PrimalPivoting}) will choose a
leaving variable (\pgmid{primalout}), pivot the basis (\pgmid{dy_pivot}),
update the primal and dual variables (\pgmid{primalupdate}),
and update the PSE pricing information and reduced costs (\pgmid{pseupdate}).
For a routine pivot, \pgmid{pseupdate} will also select an entering variable
for the next pivot.
\pgmid{dy_duenna} evaluates the outcome of the pivot, handles error detection
and recovery where possible, and performs the routine maintenance activities
of accuracy checks and refactoring of the basis.

As the final step in a routine pivot, \pgmid{tweakp1obj} scans the
working set and removes any newly feasible variables.
The objective function is adjusted to reflect any changes and
reduced costs and dual variables are adjusted or recalculated as required.
If there are no problems, the pivoting loop iterates, using the leaving
variable selected in \pgmid{pseupdate}.
The loop continues until primal feasibility is reached, the problem
is determined to be infeasible, or an exception or fatal error occurs.

When the working set becomes empty, \pgmid{tweakp1obj}
will give a preliminary indication of primal feasibility.
If \pgmid{verifyp1obj} confirms that all variables are primal feasible,
the pivoting loop will end.
If accumulated numerical inaccuracy has caused previously feasible variables
to become infeasible, the pivot selection parameters will be tightened, 
\pgmid{dy_initp1obj} will be called to build a new working
set and objective, and pivoting will resume.

Changes to the objective coefficients may make it necessary
to select a new entering variable.
This situation arises when a variable gains feasibility but remains
basic, as changing an entry of $c^B$ can potentially affect all
reduced costs\footnote{%
Less commonly, the problem arises because the newly feasible leaving variable
of the just-completed pivot has been selected to reenter.
The objective coefficient for this variable is incorrect when it is used
by \pgmid{pseupdate}.}.
The variable selected in \pgmid{pseupdate} may no longer be the best
(or even a good) choice.
The flow of control is redirected to the outer loop, where \pgmid{dy_primalin}
will be called to select an entering variable.

It can happen that no entering variable is selected by \pgmid{pseupdate}
for use in the next iteration.
Here, too, control flow is redirected to \pgmid{dy_primalin}.
The single most common reason in primal simplex is
a bound-to-bound `pivot' of a nonbasic variable --- since there is no
basis change, \pgmid{pseupdate} is not called.

Another common reason for failure to select an entering variable is that all
candidates were previously flagged as unsuitable pivots.
In this case, \pgmid{dy_primalin} will indicate a `punt' and
\pgmid{dy_dealWithPunt} will be called to reevaluate the flagged variables.
If it is able to make new candidates available, control returns to
\pgmid{dy_primalin} for another attempt to find an entering variable.
If all flagged variables remain unsuitable, control flow moves to the
preoptimality actions with an indication that primal phase~I has punted.

If the current pivot is aborted due to numerical problems
(an unsuitable pivot coefficient being the most common of these),
\pgmid{pseupdate} is not executed.
Once \pgmid{dy_duenna} has taken the necessary corrective action, the flow
of control moves to the outer loop and \pgmid{dy_primalin}.

When \pgmid{dy_primalin} indicates optimality,
\pgmid{dy_primalpivot} indicates optimality or unboundedness, or 
\pgmid{tweakp1obj} indicates primal feasibility,
the inner pivoting loop ends and
\pgmid{verifyp1obj} is called to verify feasibility.
If feasibility is confirmed, \pgmid{preoptimality} is called to
refactor the basis, perform accuracy checks, and confirm primal and dual
feasibility.
If there are no surprises, primal phase I terminates with an indication of
optimality (primal feasibility), unboundedness, or primal infeasibility.
In any event, if \pgmid{preoptimality} reports that the
solution is primal
feasible, phase~I will end with an indication of optimality even if it was
not expected from the pivot loop termination condition.

If a primal feasible solution has been found, the original objective will
be restored before returning from \pgmid{primal1}.
The transition to phase~II entails calculating the objective, dual variables,
and reduced costs for the original objective.
If the problem is infeasible or unbounded, the phase~I objective is left
in place and \dylp will use it as it attempts to activate variables or
constraints to deal with the problem (\secref{sec:ErrorRecovery}).

Loss of primal feasibility can occur when the basis is factored during the
preoptimality checks.
The pivot selection parameters are tightened and pivoting resumes.

Loss of dual feasibility is considered only when it is accompanied
by lack of primal feasibility (\ie, a false indication of infeasibility).
Loss of dual feasibility can occur for two distinct reasons.
In the less common case, loss of dual feasibility stems from loss of numeric
accuracy.
The pivot selection rules are tightened and pivoting resumes.

The more common reason for apparent loss of dual feasibility at the termination
of phase~I primal simplex is that it is ending with a punt, as described
above.
The variables flagged as unsuitable for pivoting are not dual feasible, and
when the flags are removed to perform the preoptimality checks, dual
feasibility is revealed as an illusion.
No further action is possible within primal simplex; the reader is again
referred to \secref{sec:ErrorRecovery}.

When the number of false indications of optimality exceeds a hard-coded limit
(currently 15), primal simplex terminates with a fatal error.
Other errors also result in termination of the primal simplex algorithm, and
ultimately in an error return from \dylp.

\subsection{Primal Phase II}
\label{sec:PrimalPhaseII}

The overall flow of phase~II of the primal simplex is shown in Figure
\ref{fig:PrimalPhaseIIFlow}.
\begin{figure}[htbp]
\centering
\resizebox{\linewidth}{!}{\includegraphics{\figures/primal2flow}}
\caption{Primal Phase II Algorithm Flow} \label{fig:PrimalPhaseIIFlow}
\end{figure}
The major differences from phase~I are that the problem is know to be feasible
and the original objective function is used instead of an artificial
objective function.
This considerably simplifies the flow of \pgmid{primal2}.

The inner pivoting loop has only two steps: the pivot itself
(\pgmid{dy_primalpivot}) and the maintenance and error recovery functions
(\pgmid{dy_duenna}).
When \pgmid{dy_primalin} indicates optimality or
\pgmid{dy_primalpivot} indicates optimality or unboundedness
the inner loop ends and
\pgmid{preoptimality} is called for confirmation.
\pgmid{preoptimality} will refactor the basis, perform accuracy checks,
recompute the primal and dual variables, and confirm primal and dual
feasibility.
If there are no surprises, primal phase~II will end with an indication of
optimality or unboundedness.

Loss of dual feasibility (including punts) is handled as described for primal
phase~I.
Loss of primal feasibility causes \pgmid{primal2} to return with an indication
that it has lost primal feasibility, and \pgmid{dy_primal} will arrange a
return to primal phase~I\@.

\subsection{Pivoting}
\label{sec:PrimalPivoting}

\dylp offers two flavours of primal pivoting: A standard primal pivot
algorithm in which a single primal variable is selected and pivoted into
the basis, and an extended primal pivot algorithm which allows somewhat greater
flexibility in the choice of leaving variable.
By default, \dylp will use the extended algorithm.

Figure~\ref{fig:PrimalPivotCallGraph} shows the call structure of the primal
pivot algorithm.
The routine \pgmid{primalout} implements standard primal pivoting;
\pgmid{primmultiout} implements extended primal pivoting.

\begin{figure}[htb]
\centering
\includegraphics{\figures/primalpivcalls}
\caption{Call Graph for Primal Pivoting} \label{fig:PrimalPivotCallGraph}
\end{figure}

The first activity in \pgmid{dy_primalpivot} is the calculation of the
coefficients of the pivot column, $\overline{a}_{j} = B^{\,-1} a_j$, by
the routine \pgmid{dy_ftran}.
With the entering primal variable and the ftran'd column in hand, one of
\pgmid{primalout} or \pgmid{primmultiout} are called to select the leaving
variable.

If the entering and leaving variables are the same (\ie, a nonbasic variable is
moving from one bound to the other), all that is required is to call
\pgmid{primalupdate} to update the values of the primal variables.
The basis, dual variables, reduced costs, and PSE pricing information are
unchanged.

If the entering and leaving variables are distinct, the pivot
is performed in several steps.
Prior to the pivot, the $i$\textsuperscript{th} row of the basis inverse,
$\beta_i$, and the vector $\trans{\tilde{a}_j} B^{\,-1}$ are calculated for
use during the update of the PSE pricing information.
The basis is pivoted next; this involves calls to \pgmid{dy_ftran} and
\pgmid{dy_pivot}, as outlined in \secref{sec:BasisPivoting}.
If the basis change succeeds, the primal and dual variables are updated by
\pgmid{primalupdate} using the iterative update formul\ae{} of
\secref{sec:UpdatingFormulas}, and then the PSE pricing information
and reduced
costs are updated by \pgmid{pseupdate}, using the update formul\ae{} of
\secref{sec:PSEPricing}.
As a side effect, \pgmid{pseupdate} will select an entering variable for the
next pivot.

\subsection{Selection of the Entering Variable}
\label{sec:PrimalStdSelectInVar}

Selection of the entering variable $x_j$ for a primal pivot is made using
the primal steepest edge criterion described in \secref{sec:PSEPricing}.
As outlined above, the normal case is that the entering variable for the
following pivot will
be selected as \pgmid{pseupdate} updates the PSE pricing information
for the current pivot.
In various exceptional circumstances where this does not occur, the routine
\pgmid{dy_primalin} is called to make the selection.

\subsection{Standard Primal Pivot}
\label{sec:PrimalStdSelectOutVar}

Selection of the leaving variable $x_i$ is made using standard
primal pivoting rules and a set of tie-breaking strategies.

Abstractly, we need to check
$x_k = \overline{b}_k - \overline{a}_{kj}\Delta_{kj}$ to find the maximum
allowable $\Delta_{kj}$ such that
$l_k \leq x_k \leq u_k \: \forall k \in B$ and $x_i = l_i$ or $x_i = u_i$ for
some $i$.
The index $i$ of the leaving variable will be
\begin{displaymath}
i = \arg \min_{k} \abs{\frac{\overline{b}_k}{a_{kj}}}
\end{displaymath}
for suitable $x_k \in B$.

The primal pivoting rules are the standard set for revised simplex with bounded
variables, and are summarised in Table \ref{Tbl:PrimalPivotRules}.
\begin{table}[htb]
\renewcommand{\arraystretch}{2.5}\setlength{\tabcolsep}{.75\tabcolsep}
\begin{center}
\begin{tabular}{*{3}{>{$}c<{$}}}

\text{leaving } x_i & \text{entering } x_j &
\text{pivot } \overline{a}_{ij} \\[.5\baselineskip]

\nearrow \mathit{ub} & \mathit{lb} \nearrow & < 0 \\

		     & \mathit{ub} \searrow & > 0 \\

\searrow \mathit{lb} & \mathit{lb} \nearrow & > 0 \\

		     & \mathit{ub} \searrow & < 0 \\
\end{tabular}
\end{center}
\caption{Summary of Primal Simplex Pivoting Rules} \label{Tbl:PrimalPivotRules}
\end{table}
During phase~I, when a variable is infeasible below its lower bound and must
increase to become feasible, \dylp sets the limiting $\Delta_j$ based on the
upper bound, if it is finite, and uses the lower bound only when the upper
bound is infinite.
Similarly, when a variable must decrease to its upper bound, the lower bound
is used to calculate the limiting $\Delta_j$ if it is finite.

\dylp provides a selection of tie-breaking strategies when there are multiple
candidates with equal
$\abs{\Delta_{kj}} = \Delta_{\mathrm{min}}$.
The simplest is to select the first variable $x_k$ such that $\Delta_{kj} = 0$.
A slightly more sophisticated strategy is to scan all variables
eligible to leave and pick $x_i$ such that
$i = \arg \max_{k \in K} \abs{\overline{a}_{kj}}$,
$K = \{ k \mid \abs{\Delta_{kj}} =  \Delta_{\mathrm{min}} \}$;
\dylp will use this strategy by default.
\dylp also provides four additional strategies based on hyperplane alignment,
as described in \secref{sec:AntiDegenLite}.
An option allows the tie-breaking strategy to be selected by the client.

In case of degeneracy, the perturbed subproblem anti-degeneracy algorithm
described in \secref{sec:PerturbedAntiDegeneracy} is also available.
The client can control the use of perturbed subproblems through two options
which specify whether a perturbed subproblem can be used, and how many
consecutive degenerate pivots must occur before the perturbed subproblem
is created.
By default, \dylp uses perturbed subproblems aggressively and will
introduce one when faced with a second consecutive degenerate pivot.

\subsection{Extended Primal Pivot}
\label{sec:PrimalGenSelectOutVar}

All dual variables have a single finite bound of zero, so it's not possible to
develop a generalised primal pivoting algorithm analogous to the dual pivoting
algorithm of \secref{sec:DualGenSelectInVar}.
It is, however, possible to introduce some flexibility in the selection of the
leaving variable.
We can also apply the same strategy used in generalised dual
pivoting to promote a numerically stable pivot candidate over an unstable 
candidate.

In phase~I, for an infeasible basic variable with finite upper and
lower bounds, there are two points where the variable 
can be pivoted out of the basis: When the variable moves from infeasibility
to one of its bounds (the `near' bound), and when it has crossed the feasible
region to the opposite (`far') bound.
Pivoting when the near bound is reached is optional; pivoting at the far
bound is mandatory if primal feasibility is to be maintained.
The same notion can be applied in phase~II, but its utility is much more
limited: In cases where a basic variable is at its near bound and
could be pushed to the far bound, we may prefer
to choose a degenerate and numerically stable pivot over a degenerate and
numerically unstable pivot.

\dylp implements extended primal pivoting by first collecting the set of
candidates $x_i$ to leave the basis.
Variables with two finite bounds get two entries, one with the value of
$\Delta_{ij}$ associated with the near bound, the other the value associated with
the far bound.
The set is then sorted using nondecreasing value of $\abs{\Delta_{kj}}$,
with numerical stability as the tie-breaker.

The process of scanning for candidates and sorting the resulting set is
implemented in the routines \pgmid{scanForPrimalOutCands} and
\pgmid{primalcand_cmp}.
For efficiency, \pgmid{scanForPrimalOutCands} keeps a `best candidate' using
the standard primal pivoting rules.
If this candidate is good (nondegenerate and numerically stable), it is
accepted as the leaving variable and no further processing is required.

If a good candidate is not identified by the scan, an attempt is made to
promote a good candidate to the front of the sorted list.
The criteria is as outlined for generalised dual pivoting: If the amount of
primal infeasibility that would result from promoting a stable, nondegenerate
candidate is tolerable, that candidate is promoted and made the leaving
variable.
This promotion of a stable pivot over an unstable pivot is implemented in
the primal version of \pgmid{promoteSanePivot}.

Antidegeneracy using perturbed subproblems is used with extended primal
pivoting.
The alignment-based anti-degeneracy strategies are not implemented.

\section{Variable Management}
\label{sec:VariableManagement}

Activation and deactivation of variables and constraints is a core activity
for dynamic simplex.
The activation or deactivation of variables can occur as an independent
activity or as a consequence of constraint activation and deactivation
(\vid \secref{sec:ConstraintManagement}).
During normal execution (\vid Fig.~\ref{fig:DylpFlow}) variables are
activated (\pgmid{dy_activateVars}) when primal simplex returns an
indication of infeasibility or when
primal or dual simplex achieve optimality.
Variables are deactivated (\pgmid{dy_deactivateVars}) when dual simplex
achieves optimality and returns to primal phase~II after adding variables.

In a somewhat different context, dual feasible variables are evaluated as
dual bounding constraints and activated (\pgmid{dy_dualaddvars}) when dual
simplex indicates an unbounded dual (infeasible primal).
Dual feasible variables are also activated (\pgmid{dy_activateVars}) when
dual simplex will be reentered
after adding constraints without an intervening primal simplex phase.
The motivation is to increase the probability that the dual problem
will remain bounded.

Figure \ref{fig:VarmgmtCalls} shows the call structure for the top-level
variable activation and deactivation routines.
\begin{figure}[htbp]
\centering
\includegraphics{\figures/varmgmtcalls}
\caption{Call Graph for Variable Management Routines}\label{fig:VarmgmtCalls}
\end{figure}

\subsection{Variable Management Primitives}

There are two primitive variable management routines:
\begin{itemize}
  \item
  \pgmid{dy_actNBPrimArch} activates a primal architectural variable into the
  nonbasic partition.

  \item
  \pgmid{dy_deactNBPrimArch} deactivates a nonbasic primal architectural
  variable.
\end{itemize}

\dylp assumes that inactive variables are feasible and at bound and provides
no independent way to specify the value of the variable.
As a special case, inactive free variables are assumed to have the value zero.
A consequence of this is that it is not possible to deactivate a basic
variable;
the variable must first be forced into the nonbasic partition.
Unless the variable is basic at bound, this will change the variable's value.
The special-purpose routine \pgmid{dy_deactBPrimArch} performs this service
when \dylp is attempting to force primal feasibility by deactivating infeasible
basic variables.

\dylp provides no method for activating an architectural variable into the
basic partition.
When activating a constraint, the logical variable associated with the
constraint is always used as the new basic variable.

\subsection{Activation of Variables}
\label{sec:VariableActivation}

\dylp looks for variables to activate whenever optimality
is attained for the current set of constraints and variables, or when
the active system is found to be infeasible.
The set of inactive variables is scanned
and any variables with favourable
reduced costs are activated and placed in the primal nonbasic partition.

If an optimal solution has been found for the active constraint system
by either primal or dual simplex, \pgmid{scanPrimVarStdAct} is called
to select a set of variables to be activated under the assumption that
primal phase~II iterations will resume after the variables are added.
The reduced costs are calculated using the original objective function
for the problem.
Variables are selected for activation if their reduced cost indicates they
are not at their optimal bound (\ie, dual infeasible).

If phase~I of the primal simplex has found the problem to be infeasible,
\pgmid{scanPrimVarStdAct} is again used to select the set of variables to be
activated, but the reduced costs are calculated using the phase~I objective
(as described in \secref{sec:PrimalPhaseI}).
Primal phase~I iterations resume after variables are added.

Normally, when dual simplex indicates optimality, primal phase~II is executed
after adding variables with favourable (dual infeasible) reduced costs.
It can happen, however, that there are no such variables.
In this case, \dylp will attempt to add violated constraints and, if any are
found, resume execution of dual simplex.
To increase the likelihood that the dual problem will remain bounded,
\dylp will again attempt to add variables before resuming dual simplex
iterations, but the criteria in this case will be variables whose
reduced costs are dual feasible (\ie, unfavourable from a primal perspective).

Activating a variable into the nonbasic partition
will not change to the basis, primal or dual variable values, or DSE pricing
information.
The reduced cost and the projected column norm used for PSE pricing must
be properly initialised for the new variable.
The action taken for the projected column norm depends on the context of
variable activation.
If primal simplex was executing prior to variable activation and will
be resumed after variable activation, the projected column norms are
up-to-date and correct values must be calculated for the new variables.
In other cases, PSE pricing information will be initialised when
primal simplex iterations resume and no action is required.

If the dual simplex has found the problem to be primal infeasible (dual
unbounded), the problem of selecting variables to add should be viewed
from the perspective of looking for dual constraints which will bound
the problem.
The goal is to activate one or more dual constraints and return to
dual simplex iterations.

The selection of the candidate entering dual variable $y_i$ (leaving primal
variable $x_i$) has fixed the direction of travel, $\zeta_i$.
The best outcome will be to add dual constraints (primal variables) which
block travel in the direction $\zeta_i$.
If that isn't possible (because activating any bounding dual constraint would
result in the loss of dual feasibility) a second possibility is to activate
variables which will change the dual reduced costs (the values of the primal
basic variables) so that a different dual variable $y_k$ is selected to enter.
The hope is that motion in a different direction $\zeta_k$ may make it
possible to activate constraints which will bound the dual without loss of
feasibility.

The subroutine \pgmid{dy_dualaddvars} controls the search process, and can
activate three classes of variables, for convenience called type~1, type~2,
and type~3.

Type~1 variables are those variables which constitute feasible dual
constraints which bound the dual problem.
These can be activated and placed in the primal nonbasic partition without
losing dual feasibility.
Type~1 variables are preferred, as \pgmid{dy_dualaddvars} can activate any
number of them in a given call.

If there are no type~1 variables, \pgmid{dy_dualaddvars} considers type~2
variables.
Type~2 variables are those variables which constitute dual constraints that
bound the dual problem and which, while not dual feasible if activated into
the primal nonbasic partition, will give a dual feasible solution if
activated and immediately pivoted into the basis.
This is equivalent to adding a cutting plane which renders the current
solution infeasible and executing a single pivot to regain feasibility;
necessarily, the objective will deteriorate.
In the context of Table \ref{Tbl:DualPivotRules}
in \secref{sec:DualStdSelectInVar},
this amounts to
selecting a pivot with the signs of $\overline{c}_j$ and $\overline{a}_{ij}$
reversed.
The pivot is sufficiently similar to a normal dual pivot that it can be
handled by \pgmid{dy_dualpivot}.
It is not standard in that the entering primal variable will move
away from its bound toward the infeasible side (\eg, $x_j$ would enter falling
from its lower bound with $\overline{c}_j < 0$ and $\overline{a}_{ij} > 0$).
One such variable can be activated on each call to \pgmid{dy_dualaddvars}.

In the absence of type~1 or type~2 variables, type~3 variables are considered.
These are variables which are not dual feasible at their current
bound but which will reduce the infeasibility of the leaving primal variable
if activated and changed to their opposite bound.
The motivation for activating a type~3 variable is that it makes the reduced
cost of $y_i$ less desirable, so that some other variable $y_k$ can be
selected to enter (thus moving in a different direction $\zeta_k$).
The routine \pgmid{type3activate} will attempt to activate as many type~3
variables as required in order to change the entering dual variable $y_i$.

Activation of type~2 or type~3 variables is generally not cost-effective.
By default, \dylp limits \pgmid{dy_dualaddvars} to type~1 activations.
The dynamic simplex algorithm will revert to primal phase~I if no
type~1 variables exist.
An option allows the client to specify whether type~1, type~2, or type~3
variables will be considered.

Activation of a type~1 variable is no different from any other activation into
the nonbasic partition, as described above.
For type~2 variables, the pivot will cause a change of basis.
\pgmid{dy_dualpivot} will take care of the required calculations and updates
in the context of dual simplex.
For type~3 variables, the basis doesn't change, and the values of the dual
variables and DSE norms are unchanged.
The values of the primal variables do change, however, and this changes the
DSE pricing information.

\subsection{Deactivation of Variables}
\label{sec:VariableDeactivation}

Deactivation of variables occurs when dual simplex finds an optimal
solution for the active constraint system and variable activation identifies
dual infeasible variables for activation.
In this case, variable deactivation is performed before entering
primal phase~II simplex.
The subroutine \pgmid{dy_deactivateVars} is called to deactivate variables
according to a client-specified threshold, expressed as a percentage of
the maximum unfavourable reduced cost over all active variables.

Specifically, \pgmid{dy_deactivateVars} scans the reduced costs of the active
variables and determines a pair of values
$\displaystyle \check{c} = \max_{\{k : c_k < 0\}}\abs{c_k}$ and
$\displaystyle \hat{c} = \max_{\{k : c_k > 0\}}\abs{c_k}$.
It then deactivates variables with $c_k > \hat{c}(\pgmid{dy_tols.purgevar})$
or $c_k < -\check{c}(\pgmid{dy_tols.purgevar})$.

\subsection{Initial Variable Selection}

For a cold start, the initial set of active variables is completely determined
by the initial set of constraints.
All variables referenced in the constraints are activated.
As noted in \secref{sec:Startup}, the client can set parameters which will
cause variable deactivation to be executed prior to starting simplex
iterations.

For a hot start, the initial set of active variables is the set that was
active at return from the previous call to \pgmid{dylp}.

For a warm start, the set of active constraints is specified by the basis.
The initial set of active variables can be determined from the constraints
as for a cold start, or the client can specify a set of variables which
should be activated as the active constraint system is created.

As noted in \secref{sec:Startup}, for a hot or warm start the client can
set parameters which will cause variable activation to be executed prior
to starting simplex iterations.

\section{Constraint Management}
\label{sec:ConstraintManagement}

Constraint management activities can be separated into selection of the initial
constraint set, activation of violated or bounding constraints, and
deactivation of loose constraints.
In general, the goal is to maintain an active constraint system which is
a subset of the original constraint system, consisting only of
equalities and those inequalities necessary to define an optimal extreme point.
\dylp expects that all constraints will be equalities or $\leq$
inequalities.
Figure \ref{fig:ConmgmtCalls} shows the call structure for the constraint
activation and deactivation routines.
\begin{figure}[htbp]
\centering
\includegraphics{\figures/conmgmtcalls}
\caption{Call Graph for Constraint Management Routines}\label{fig:ConmgmtCalls}
\end{figure}

During construction of the initial constraint system, any variables referenced
in a constraint are activated along with the constraint.
During subsequent constraint activation phases, variable activation is more
selective.
The logical variable for the constraint is created and used as the new basic
variable.
If the next simplex will be primal simplex, activation is restricted to
the subset of referenced variables with dual infeasible (favourable)
reduced cost.
If the next simplex will be dual simplex, activation is restricted to
the subset of referenced variables that are dual feasible.

When a constraint is deactivated, only the slack variable for the constraint
is deactivated.
This minimises the work that must be performed to repair the basis.

\subsection{Initial Constraint Selection}
\label{InitialConSelect}

For a warm or hot start, the initial active constraint system is completely
determined from the basis supplied by the client.
As noted in \secref{sec:Startup}, the client can
set parameters which will cause constraint activation to be executed prior
to starting simplex iterations.
In this specific case, variable activation is not automatic and must be
requested independently if desired.

For a cold start, where no initial basis is supplied,
the initial active constraint system will include all equalities and
a client-specified selection of inequalities.
See \secref{sec:ColdStart} for a more detailed description.

\subsection{Activation of Constraints}
\label{ConstraintActivation}

\dylp enters the constraint activation phase 
whenever the system is found to be primal unbounded or optimal for
the set of active constraints and all variables (active and inactive).
When the system is found to be optimal, \dylp calls
\pgmid{dy_activateCons} to search the inactive constraints
for violated constraints.
When the system is found to be unbounded, \dylp first calls
\pgmid{dy_activateBndCons} to search the inactive
constraints for feasible constraints which block the direction of recession.
If such bounding constraints exist, they are activated and
primal phase~II simplex is resumed.
Otherwise, \pgmid{dy_activateCons} is called to add any violated
constraints and execution will go to primal phase~I or dual simplex as
appropriate.

Violated constraints are identified using a straightforward scan of
the inactive constraints.
The routine \pgmid{scanPrimConStdAct} evaluates each constraint at the
current value of $x$ and returns a list of violated constraints.
The routines \pgmid{dy_activateBLogPrimConList} and
\pgmid{dy_activateBLogPrimCon} perform the activations.
Following activations, the logical variables for the new constraints are made
basic, the basis is refactored, and a new basic solution is calculated.
If the call to \pgmid{dy_activateCons} requested activation of referenced
variables, \pgmid{dy_activateBLogPrimConList} will collect a set of variable
indices for activation.
After the basis has been refactored, the set is passed to
\pgmid{dy_activateVars} for activation.
If dual simplex will be the next simplex executed, only dual-feasible variables
are activated.

In \dylp, unboundedness is detected by the primal simplex implementation;
dual simplex is not called until primal simplex has found an initial
optimal solution.
When unboundedness is discovered,
\dylp calls \pgmid{dy_activateBndCons} to search for
bounding constraints which are feasible at the current basic solution.
A constraint will block motion in the
direction $\eta_j$ if $a_i \cdot \eta_j > 0$ for
$x_j$ increasing, or $a_i \cdot \eta_j < 0$ for $x_j$ decreasing.
This scan is performed by \pgmid{scanPrimConBndAct}.
Once the list of constraints is returned, constraint activation and basis
repair proceed as in the case of violated constraints, but referenced variables
are not activated.

When a constraint is activated, the set of basic variables is augmented
with the slack variable for the constraint.
Because the slack is basic, the value of the associated dual is zero.
The basis will change, but the values of other active primal variables will
remain the same.
Since the new slack variables are not part of the PSE reference frame,
the projected column norms associated with PSE pricing are unchanged.
Because the objective coefficients associated with the slack variables
are 0, the values of the preexisting dual variables and the reduced costs
remain unchanged.

\subsection{Deactivation of Constraints}

Constraint deactivation is handled by \pgmid{dy_deactivateCons}.
\dylp implements three options for constraint deactivation, `normal',
`aggressive', and `fanatical'.
When normal constraint deactivation is specified, \dylp will only deactivate
inequalities which are strictly loose.
Eligible inequalities are identified by scanning the basis for slack
variables which are strictly within bounds.
When aggressive constraint deactivation is specified, \dylp will
also deactivate tight inequalities whose associated dual
variable is zero.
When fanatical constraint deactivation is specified, \dylp will deactivate any
constraint (equality or inequality) whose associated dual is zero.
The set of constraints to be deactivated is identified by the routine
\pgmid{scanPrimConStdDeact}.

Once a set of constraints has been identified for deactivation, the routines
\pgmid{deactBLogPrimConList} and \pgmid{dy_deactBLogPrimCon} are called to
perform the deactivations.
The corresponding constraint is deactivated and removed from the active
constraint system along with its associated logical variable.
The basis is patched, if necessary, by moving the variable which is basic
in the position of the deactivated constraint to the basis position which was
occupied by the constraint's associated logical.

As with activation of constraints, deactivation of constraints changes
the basis and \dylp will refactor and recalculate the primal and dual
variables.
The dual variables do not change, nor do the reduced costs of the remaining
variables, since the cost coefficient of a logical variable is zero.
In general, the PSE column norms will be changed because the
deleted logical variables may be part of the reference frame.
\dylp opts to reset the reference frame to deal with this, rather than
updating or recalculating the column norms.


\section{\dylp Interface}
\label{sec:DylpInterface}

This section describes the native interface for \dylp.
In addition to the main routine, \pgmid{dylp}, routines are provided
for access to the solution, including rays and tableau vectors; for pricing; 
for printing; and for a few miscellaneous services.
Sections~\ref{sec:SimplexSolver}~--~\ref{sec:StartUpShutdownSummary} document
the basic use of \dylp.
Sections~\ref{sec:SolutionRoutines}~--~\ref{sec:UtilityRoutines} describe the
routines used to access the solution, and the routines provided for pricing,
printing, and miscellaneous services.
Sections ~\ref{sec:LPProbSpec}~--~\ref{sec:DylpTolerances} describe the
parameters, options, and tolerances provided by \dylp.
For additional details on how to use \dylp, consult the comments
in the source, particularly in \coderef{dylp.h}{} and \coderef{dy_setup.c}{},
and the example drivers supplied in the distribution.

\dylp's native interface is peculiar to \dylp and a bit low-level in places.
Many individuals will find it more convenient to use \dylp as an
embedded component within the software infrastructure provided by
the \coin project \cite{COIN}.
For details of the \coin OSI layer for \dylp, OsiDylp, please consult
the \textsc{Coin}
documentation.
An added advantage of this approach is that the OSI API provides
a solver-independent interface.
The underlying solver can be easily changed because the OSI layer
insulates
the client from the details of the solver's native interface.

The \dylp distribution provides a simple C driver program using \dylp's
native interface in the file \coderef{osi_dylp.c}{}.
The command `\pgmid{osi_dylp -h}' will print a message describing the
available command line options.

\dylp assumes that the constraint system passed to it as a parameter
\textit{does not} contain logical variables (\ie, slacks and artificials).
On occasion, it must return values for logical variables; in such cases,
it will use the negative of the index of the associated constraint.

\subsection{Simplex Solver}
\label{sec:SimplexSolver}

\dylp is called as

\begin{subrdoc}
\item
\subrhdr{lpret_enum dylp}
	{lpprob_struct *orig_lp, lpopts_struct *orig_opts,
         lptols_struct *orig_tols, \\ lpstats_struct *orig_stats}

The \pgmid{orig_lp} structure (\secref{sec:LPProbSpec}) specifies the constraint
system, control options, and (optionally) an initial basis and status vector
and an initial active variable set.
It is used to return the final status, primal and dual variable values,
basis, and status vector, and (optionally) the active variables.

The \pgmid{orig_opts} structure (\secref{sec:DylpOptions}) specifies option
settings to control \dylp's actions.
The \pgmid{orig_tols} structure (\secref{sec:DylpTolerances}) specifies
numeric tolerances and related control information.

The optional structure \pgmid{orig_stats} (\secref{sec:DylpStatistics}) can
be used (in conjunction with
conditionally compiled code) to return detailed statistics about \dylp's
actions.
\end{subrdoc}

\subsection{Parameter Routines}
\label{sec:ParameterRoutines}

The normal sequence to establish parameter values for \dylp is as follows:
\begin{enumerate}
  \item
  The client calls \pgmid{dy_defaults} to allocate option and tolerance
  structures and populate them with default values.
  The client can then adjust the parameters as desired.

  \item
  The client somehow establishes the original copy of the constraint system.
  Typically, this will be a call to a constraint system generator\footnote{%
  Consult the \pgmid{consys} documentation for information on how to use
  the routines in the \pgmid{consys} package to build a constraint system
  from scratch.}, or a call
  to a routine which will read an MPS file.

  \item
  The client calls \pgmid{dy_checkdefaults} to
  to set parameter values which are calculated based on
  properties of the constraint system, and to ensure that all parameters
  are within acceptable bounds.
\end{enumerate}


\begin{subrdoc}
  \item
  \subrhdr{void dy_defaults}%
	  {lpopts_struct **opts, lptols_struct **tols}
  
  This routine will allocate an options structure \pgmid{opts} and a
  tolerance structure \pgmid{tols} and populate them with the standard
  default values for \dylp.
  Note that default values for some parameters are calculated
  in \pgmid{dy_checkdefaults}
  based on the size of the constraint system.

  \item
  \subrhdr{void dy_checkdefaults}%
	  {consys_struct *sys, lpopts_struct *opts, lptols_struct *tols}
  
  This routine checks limits on parameter values and calculates
  values which depend on the size of the constraint system.
  User-supplied values are \textit{not} overridden unless they are outside
  of \dylp's bounds for the parameter.

  \item
  \subrhdr{void dy_setprintopts}%
	  {int lvl, lpopts_struct *opts}

  This routine is provided purely for convenience; it will set all of
  \dylp's print levels based on the single value supplied for \pgmid{lvl}.
  Roughly, $\pgmid{lvl} = 0$ suppresses all output,
  $\pgmid{lvl} = 1$ establishes the default print levels, which allow
  messages about extraordinary events, and
  $\pgmid{lvl} \geq 2$ provides increasing amounts of information.
  Consult the code for details.
\end{subrdoc}

\subsection{Basis Package Initialisation}
\label{sec:GLPKBasisInit}

The \glpk basis package used in \dylp maintains static data structures that
must be initialised before use and freed after use.
For efficiency, it is useful to postpone initialisation until the size of the
constraint system is known and can be used to estimate the size of the basis
package's data structures, but \dylp will expand the basis structures if it
detects that the constraint system has grown too large for the allocated
capacity.
Initialisation must occur before the first call to \pgmid{dylp}.
The basis structures should be freed when they are no longer needed.

\begin{subrdoc}
  \item
  \subrhdr{void dy_initbasis}%
	  {int concnt, int factor_freq, double zero_tol}

  \pgmid{dy_initbasis} initialises the data structures used by the \glpk basis
  maintenance package.
  \pgmid{concnt} specifies the maximum allowable number of rows (constraints).
  \pgmid{factor_freq} is the maximum number of basis updates which can occur
  between each (re)factorisation of the basis.
  A conservative value will be a bit larger than the regular refactorisation
  interval; for \dylp, $\pgmid{lpopts.factor}+5$.
  The final parameter, \pgmid{zero_tol}, can be used to override \glpk's
  default zero tolerance if it is set to any value other than zero.
  Be sure you understand the ramifications of overriding the default.

  The routine sets several other parameters important to pivoting.
  Interested readers should consult the comments in the code
  (\coderef{dy_basis.c}{dy_initbasis}).

  \item
  \subrhdr{void dy_freebasis}{void}

  This routine will free the data structures allocated by the call to
  \pgmid{dy_initbasis}.
\end{subrdoc}

\subsection{Information and Error Messages}
\label{sec:IOandErrorMsgs}

\dylp uses private library packages for information and error
messages\footnote{%
This usage is historical, rooted in an ancient era when i/o was still a
roll-your-own enterprise that differed dramatically from one operating system
and programming environment to the next.}.
The most visible value-added service provided by the libraries is
integration of file and terminal output.
Routines which generate output accept parameters to specify whether the
output generated by a call should be sent to a file, to the terminal, both,
or neither.
The library packages must be initialised during startup.
A brief explanation is provided here.

\subheading[l]{Information Messages}

The I/O library provides a convenient means to generate information messages.
Information messages may use any of the standard C conversion
specifications; the
underlying print engine for the current implementation is \pgmid{vfprintf}.
In addition to integrated file and terminal i/o, the library manages open file
descriptors and coordinates activity with the error message library.
See the code for examples of usage of the routines used to generate information
messages (\pgmid{outchr}, \pgmid{outfmt}, and \pgmid{outfxd}).
The simple driver in \coderef{osi_dylp.c}{} contains a fragment of code which
uses the \pgmid{chgerrlog} routine to merge information and error messages in a
single log file.

Initialisation and shutdown of the error message package is accomplished with
the routines \pgmid{ioinit} and \pgmid{ioterm}.
\begin{subrdoc}
  \item
  \subrhdr{bool ioinit}{void}

  Initialises internal data structures.

  \item
  \subrhdr{void ioterm}{void}

  Cleans up and shuts down the i/o package.
  Note that \pgmid{ioterm} \textit{does not} close open streams.
  It is assumed that the client will close open streams as appropriate, and
  that remaining streams can be left open until closed by the operating system
  at program termination.
\end{subrdoc}

\subheading[l]{Error Messages}

The error message library provides a convenient means to generate warning and
error messages.
Error messages may use any of the standard C conversion specifications; the
underlying print engine for the current implementation is \pgmid{vfprintf}.
The text of error messages reside in a file
(\coderef{bonsaierrs.txt}{} in the \dylp distribution).
Error messages are printed using the routines \pgmid{warn} and \pgmid{errmsg}.
In calls to these routines, the error message is specified by a number.
If an error message file cannot be located, a generic error message giving the
error number will be produced.
See the code for examples of usage of the routines used to generate
warning (\pgmid{warn}) and error (\pgmid{errmsg}) messages.

Initialisation and shutdown of the error message package is accomplished with
the routines \pgmid{errinit} and \pgmid{errterm}.
\begin{subrdoc}
  \item
  \subrhdr{void errinit}%
	  {const char *emsgpath, const char *elogpath, bool errecho)}

  The parameter \pgmid{emsgpath} specifies the file containing the error
  messages.
  The parameter \pgmid{elogpath} specifies a file name to be used to log
  error messages; if null, error messages are not logged.
  The parameter \pgmid{errecho} should be set to true if error messages
  should be echoed to \pgmid{stderr}, false otherwise.

  \item
  \subrhdr{void errterm}{void}

  Cleans up and shuts down the error message package.
  In keeping with the behaviour of \pgmid{ioterm}, it is left to the
  client or operating system to close any error log file.
\end{subrdoc}

On startup, the error message package should be initialised first, followed by
the i/o package.
At termination, the i/o package should be shut down first.

\subsection{Summary of \dylp Startup and Shutdown}
\label{sec:StartUpShutdownSummary}

Pulling together the information from the previous sections, the sequence of
actions required to use \dylp is listed below.
\begin{enumerate}
  \item
  Initialise the error message and i/o packages.
  Open log files for information and error messages (optional).

  \item
  Establish default parameter structures.
  Open and parse a file of \dylp option specifications (optional).

  \item
  Create a constraint system using a constraint generator or by reading an
  input file.
  Adjust options and tolerances to match the constraint system.
  At some point between creating the constraint system and calling
  \pgmid{dylp}, convert any `$\geq$' constraints to other forms.

  \item
  Initialise the basis package.

  \item
  Construct parameter structures and call \pgmid{dylp}.

  \item
  Process the answer, restoring `$\geq$' constraints and adjusting the answer
  appropriately, if the application demands it.

  \item
  Free data structures.
  This may require an additional call to \pgmid{dylp}, if the parameters
  given in the previous call instructed \dylp to retain internal data
  structures for efficient reoptimisation.
  It will certainly require calls to \pgmid{dy_freebasis},
  \pgmid{dy_freesoln}, and \pgmid{consys_free}.

  \item
  Close files and shut down the i/o and error message packages.
\end{enumerate}
Consult the sample drivers provided with \dylp for example implementations.

\subsection{Solution, Ray, and Tableau Routines}
\label{sec:SolutionRoutines}

\dylp's natural inclination is to return a compact version of the basis
and the primal and dual solutions
within the \pgmid{lpprob_struct} passed as the \pgmid{orig_lp} parameter to
\pgmid{dylp}.
This is not always convenient or sufficient for the client, so \dylp also
provides a variety of routines to return individual components of the
solution, rays (when present), and tableau vectors.

\subheading[l]{Solution Routines}

The following routines return the values of primal and dual variables and the
status of primal variables.
In all cases, if the client does not provide a vector for the result, a vector
will be allocated.

\begin{subrdoc}
  \item
  \subrhdr{void dy_colDuals}%
	  {lpprob_struct *orig_lp, double **p_cbar, bool trueDuals}

  Returns the values of the dual variables associated with nonbasic primal
  variables, $\overline{c}^N = c^N - c^B B^{\,-1}N$, in column order.
  Put another way, the dual variables associated with
  architecturals at bound (tight implicit bound constraints) and 
  logicals at bound (tight explicit constraints).
  If \pgmid{trueDuals} is false, the sign convention is that of the
  minimisation primal problem with implicit bounds solved by \dylp,
  hence the vector returned is precisely the reduced costs.
  If \pgmid{trueDuals} is true, the sign convention is that of the true
  dual problem with all implicit bound constraints made explicit, hence
  reduced costs corresponding to primal variables nonbasic at upper bound are
  negated to get the correct value for the dual variable.

  \item
  \subrhdr{void dy_rowDuals}%
	  {lpprob_struct *orig_lp, double **p_y, bool trueDuals}

  Returns the values of the dual variables associated with the constraints,
  $y = c^B B^{\,-1}$, in row order.
  If \pgmid{trueDuals} is false, the sign convention is that of the
  minimisation primal problem solved by \dylp.
  If \pgmid{trueDuals} is true, the sign convention is that of the true
  dual problem.
  In either case, dual variables associated with `$\geq$' constraints are
  negated.
  (This convention is adopted so that $yA$ `just works' on a constraint
  system with a mix of `$\leq$' and `$\geq$' constraints.)

  \item
  \subrhdr{void dy_colPrimals}%
	  {lpprob_struct *orig_lp, double **p_x}

  Returns the values of all primal variables, in column order.
  (See also \pgmid{dy_expandxopt}.)

  \item
  \subrhdr{void dy_rowPrimals}%
	  {lpprob_struct *orig_lp, double **p_xB, int **p_indB}

  Returns the values of the basic primal variables, in row (basis) order.
  The indices of the variables are returned in \pgmid{p_indB}.
  The indices of basic logical variables are encoded as the negative of the
  index of the associated constraint.

  \item
  \subrhdr{void dy_logPrimals}%
	  {lpprob_struct *orig_lp, double **p_logx}

  Returns the values of the primal logical variables, in row order.

  \item
  \subrhdr{void dy_colStatus}%
	  {lpprob_struct *orig_lp, flags **p_colstat}

  Returns the status of all primal variables, in column order.
  Note that this routine returns the full range of status codes used by
  \dylp.

  \item
  \subrhdr{void dy_logStatus}%
	  {lpprob_struct *orig_lp, flags **p_logstat}

  Returns the status of all logical variables, in row order.
  Note that this routine returns the full range of status codes used by
  \dylp.

  \item
  \subrhdr{bool dy_expandxopt}%
	  {lpprob_struct *lp, double **p_xopt}

  This routine examines the \pgmid{lp} data structure and assembles the values
  of the basic and nonbasic primal variables into a single vector.
  (See also \pgmid{dy_colPrimals}.)
\end{subrdoc}

\subheading[l]{Ray Routines}

This pair of routines return the rays emanating from the current primal or
dual extreme point, when available, up to the number specified by the
client.
The number of rays returned will be less than or equal to the number
requested.
\dylp always returns a status code in the context of the primal problem.
An indication of primal unbounded guarantees the existence of primal rays.
An indication of primal infeasibility will \textit{almost always} indicate dual
unboundedness, but keep in mind that it is possible for a problem to be both
primal and dual infeasible.

\begin{subrdoc}
  \item
  \subrhdr{bool dy_primalRays}%
	  {lpprob_struct *orig_lp, int *p_numRays, double ***p_rays}

  Returns all primal rays emanating from the current extreme point, up to the
  limit specified by \pgmid{p_numRays}.
  On return, \pgmid{p_numRays} will be set to the number of rays actually
  returned, and each entry in \pgmid{p_rays} will point to a vector containing
  one ray with components in column order.
  If no vector is supplied for \pgmid{p_rays}, one will be allocated.
  Do not supply vectors for individual rays, these are allocated as needed.

  This routine should only be called when \dylp has found the problem to be
  primal unbounded.
  If the problem was solved to optimality or found to be primal infeasible,
  a warning will be issued but the routine will still return true.
  If the routine is called for any other result, it will print an error
  message and return false.

  \item
  \subrhdr{bool dy_dualRays}%
	  {lpprob_struct *orig_lp, bool fullRay, int *p_numRays,
	   double ***p_rays, \\ bool trueDuals}

  Returns all dual rays emanating from the current extreme point, up to the
  limit specified by \pgmid{p_numRays}.
  On return, \pgmid{p_numRays} will be set to the number of rays actually
  returned.

  If \pgmid{fullRay} is false, each ray is returned as an $m$-vector
  containing only the components associated with the row duals, in row order.
  If \pgmid{fullRay} is true, each ray is returned as an $(m+n$)-vector.
  The first $m$ entries contain the components associated
  with the row duals.
  The remaining $n$ entries contain the components associated with nonbasic
  primal variables (\ie, the ray components associated with tight
  implicit bounds).
  If no vector is supplied for \pgmid{p_rays}, one will be allocated.
  Do not supply vectors for individual rays, these are allocated as needed.

  If \pgmid{trueDuals} is false, the sign convention used is appropriate for a
  minimisation primal with implicit bounds (matching the values returned by
  \pgmid{dy_rowDuals} and \pgmid{dy_colDuals}).
  If \pgmid{trueDuals} is true, the sign convention used is appropriate for the
  true dual problem with all implicit bound constraints made explicit.
  In either case, duals associated with explicit `$\geq$' constraints are
  negated.

  This routine should only be called when \dylp has returned an indication
  that the problem is primal infeasible.
  As mentioned above, this will almost always correspond to dual unboundedness.
  If the problem was solved to optimality or found to be primal unbounded
  (hence dual infeasible),
  a warning will be issued but the routine will still return true.
  If the routine is called for any other result, it will print an error
  message and return false.
  
\end{subrdoc}


\subheading[l]{Tableau Routines}

These four routines return rows and columns of the basis inverse and rows
and columns of the constraint matrix multiplied by the basis inverse.

\begin{subrdoc}
  \item
  \subrhdr{bool dy_abarj}%
	  {lpprob_struct *orig_lp, int tgt_j, double **p_abarj}

  Returns $\overline{a}_j = B^{\,-1}a_j$ for a column $a_j$ of the constraint
  matrix.

  \item
  \subrhdr{bool dy_betaj}%
	  {lpprob_struct *orig_lp, int tgt_j, double **p_betaj}

  Returns column $j$ of the basis inverse, $\overline{\beta}_j = B^{\,-1}e_j$.

  \item
  \subrhdr{bool dy_abari}%
	  {lpprob_struct *orig_lp, int tgt_i, double **p_abari,
	   double **p_betai}

  Returns $\overline{a}_i = a_i B^{\,-1}$ for a row $a_i$ of the constraint
  matrix.

  Since the routine actually calculates $\beta_i A$, it can return
  $\beta_i$ at no extra computational cost.
  If \pgmid{p_betai} is provided, it will be loaded with $\beta_i$ on return.

  \item
  \subrhdr{bool dy_betai}%
	  {lpprob_struct *orig_lp, int tgt_i, double **p_betai}

  Returns row $i$ of the basis inverse, $\overline{\beta}_i = e_i B^{\,-1}$.
\end{subrdoc}

\subsection{Pricing Routines}
\label{sec:PricingRoutines}

\dylp provides two additional routines which are useful in a mixed-integer
linear programming environment.
\pgmid{dy_pricenbvars} will calculate the reduced cost for a specified set of
nonbasic variables
and \pgmid{dy_pricedualpiv} will calculate the cost of the first dual pivot
(a generalised penalty calculation) given a specified set of candidate nonbasic
variables.

\begin{subrdoc}
  \item
  \subrhdr{bool dy_pricenbvars}%
	  {lpprob_struct *orig_lp, flags priceme, \\
	   double **p_ocbar, int *p_nbcnt, int **p_nbvars}
  
  This routine calculates the reduced cost of nonbasic variables, tapping
  the \dylp data structures for active variables and calculating the reduced
  cost as needed for inactive variables.
  \pgmid{priceme} provides limited additional control by allowing the client
  to specify the status of the nonbasic variables that should be priced.
  For example, to price all variables that are nonbasic at their upper or
  lower bound,
  \pgmid{priceme} should be set to \pgmid{vstatNBUB|vstatNBLB}.
  Other nonbasic variables (fixed, free, or superbasic) will not be priced.
  (See the section on status codes in \coderef{dylp.h} for additional
  information.)
  The routine returns a compact list of \pgmid{p_nbcnt} indices of priced
  variables in \pgmid{p_nbvars}, with the corresponding reduced costs in
  \pgmid{p_ocbar}.
  The indices returned in \pgmid{p_nbvars} are the indices used in the original
  constraint system, which does not contain logical variables.
  Where nonbasic logical variables are present in the active system, they are
  identified in \pgmid{p_nbvars} by the negative of the index of the
  associated constraint.
  In particular, the values returned are appropriate for use as
  the \pgmid{nbcnt},
  \pgmid{nbvars}, and \pgmid{cbar} parameters to \pgmid{dy_pricedualpiv}.

  \item
  \subrhdr{bool dy_pricedualpiv}%
	  {lpprob_struct *orig_lp, int oxindx,
	   double nubi, double xi, double nlbi, \\
	   int nbcnt, int *nbvars,
	   double *cbar, double *p_upeni, double *p_dpeni}
  
  This routine calculates the cost of the first dual pivot associated with
  forcing the value of the basic variable $x_i$ 
  down to a new upper bound $u_i$ (a down penalty) or up to a new
  lower bound $l_i$ (an up penalty).
  
  The up penalty is
  $\displaystyle \mathit{upen}_i = \min_k \left\lgroup -(l_i - x_i)
		\frac{\overline{c}_k}{\overline{a}_{ik}} \right\rgroup$
  for $\{k \in N \mid \overline{a}_{ik} < 0 \wedge x_k < u_k \vee
		   \overline{a}_{ik} > 0 \wedge x_k > l_k\}$.

  The down penalty is
  $\displaystyle \mathit{dpen}_i = \min_k  \left\lgroup -(u_i - x_i)
		\frac{\overline{c}_k}{\overline{a}_{ik}}\right\rgroup$
  for $\{k \in N \mid \overline{a}_{ik} > 0 \wedge x_k < u_k \vee
		   \overline{a}_{ik} < 0 \wedge x_k > l_k\}$.

  To perform the standard penalty calculation for forcing a basic variable to
  an integral value, the new lower bound would be $\ceil{x_i}$
  and the new upper bound would be $\floor{x_i}$.
  The basic variable $x_i$ can be an architectural or a logical variable.
  The routine is capable of pricing a pivot involving the logical variable
  for a constraint that is not currently active.

  \pgmid{oxindx} specifies the basic variable to be priced (a logical is
  specified as the negative of the index of the associated constraint).
  \pgmid{xi} is the current value of $x_i$ in the optimal solution to the
  LP.
  (In the case of the logical for an inactive constraint, the value is obtained
  by evaluating the constraint at the current solution.)
  \pgmid{nubi} is the new upper bound $u_i$, and \pgmid{nlbi} is the new
  lower bound $l_i$.
  It should be true that $\pgmid{nubi} \leq \pgmid{xi} \leq \pgmid{nlbi}$.
  \pgmid{nbcnt}, \pgmid{nbvars}, and \pgmid{cbar} are as described for
  \pgmid{dy_pricenbvars}.
  The up and down penalties will be returned in \pgmid{p_upeni} and
  \pgmid{p_dpeni}, respectively.

\end{subrdoc}

\subsection{Print Routines}
\label{sec:PrintRoutines}

There are three routines to supply strings for \dylp status, phase, and return
codes, a routine to print the compact solution returned by \pgmid{dylp},
and a routine to print the contents of the statistics structure.

\begin{subrdoc}
  \item
  \subrhdr{dy_dumpcompact}%
	  {ioid chn, bool echo, lpprob_struct *soln,  bool nbzeros}

  This routine prints the solution returned by \pgmid{dylp} in \pgmid{soln}
  using a human-readable format.
  Output is directed to the channel specified by \pgmid{chn}, and echoed to the
  terminal if \pgmid{echo} is true.
  Normally, nothing is printed for nonbasic variables with a value of zero;
  set \pgmid{nbzeros} to true to force them to be printed.

  \item
  \subrhdr{dy_dumpstats}%
	  {ioid chn, bool echo, lpstats_struct *lpstats,
	   consys_struct *orig_sys}

  This routine prints the contents of the \pgmid{lpstats} structure in a
  human-readable format.
  \pgmid{chn} and \pgmid{echo} are as for \pgmid{dy_dumpcompact}.
  \pgmid{orig_sys} should be the same constraint system referenced in the
  \pgmid{orig_lp} parameter to \pgmid{dylp}

  \item
  \subrhdr{dy_prtlpret}{lpret_enum lpret}

  Returns a pointer to a string for the return code
  specified in \pgmid{lpret}.

  \item
  \subrhdr{dy_prtlpphase}{dyphase_enum phase, bool abbrv}

  Returns a pointer to a string for the return code
  specified in \pgmid{phase}.
  If \pgmid{abbrv} is true, this will be a two-character abbreviation.

  \item
  \subrhdr{dy_prtvstat}{flags status}

  Returns a pointer to a static buffer containing a string representation
  of the status flags specified in \pgmid{status}.
  The buffer is overwritten at each call.
\end{subrdoc}

\subsection{Utility Routines}
\label{sec:UtilityRoutines}

And a final pair of utility routines.

\begin{subrdoc}
  \item
  \subrhdr{bool dy_dupbasis}%
	  {int dst_basissze, basis_struct **p_dst_basis,
	   basis_struct *src_basis, \\ int dst_statussze,
	   flags **p_dst_status, int src_statuslen, flags *src_status}

  This routine will duplicate the basis and status arrays.
  Data structures will be allocated as required if they are not supplied as
  parameters.

  \item
  \subrhdr{dy_freesoln}%
	  {lpprob_struct *lpprob}

  This routine will free the data structures used to hold the LP solution,
  including data structures for the basis, status vector, primal and dual
  variable values, and the active variables vector.
\end{subrdoc}


\subsection{The LP Problem Specification}
\label{sec:LPProbSpec}

The structure \pgmid{lpprob_struct *orig_lp} is used to define the LP
problem to \dylp and to return the answer to the client.
It holds pointers to the constraint system, an active
variable vector, a basis vector, a status vector, and vectors
for the primal and dual variables, as well as fields for information and
control.
Each field is discussed below; for precise details, the reader should consult
the file \coderef{dylp.h}{}.

\begin{codedoc}
  \item\varhdr{actvars}
  A vector used to specify and/or return the set of active variables.
  The vector supplied as an input parameter will be overwritten on output.
  \begin{description}[\textbf{(o)}]
    \item[\textbf{(i)}]
    For a warm start, an initial set of active variables can be specified.
    This information will be used only if the \pgmid{lpctlACTVARSIN} flag
    is set in the \pgmid{ctlopts} field.
    For a cold or hot start, a vector can be provided to return the
    final set of active variables.

    \item[\textbf{(o)}]
    The final set of active variables.
    If no vector was supplied as an input parameter, \dylp will allocate
    one on output.
    Active variable information is returned only if the \pgmid{lpctlACTVARSOUT}
    flag is set when \dylp is called.
    Valid information is returned only if an optimal solution is found.
    If valid information is not returned, the \pgmid{lpctlACTVARSOUT} flag
    will be reset.
  \end{description}

  \item\varhdr{basis}
  A data structure for the LP basis.
  Because the set of active constraints at optimum will not, in general,
  include all constraints, the basis vector specifies the constraint and
  the primal variable in each basis position.
  \begin{description}[\textbf{(o)}]
    \item[\textbf{(i)}]
    For a warm start, an initial basis must be provided.
    For a cold or hot start, a structure can be provided to return the
    final basis.

    \item[\textbf{(o)}]
    The final basis.
    If no vector was supplied as an input parameter, \dylp will allocate
    one on output.
  \end{description}

  \item\varhdr{colsze}
  The allocated column capacity of the data structure.
  The \pgmid{status} and \pgmid{actvars} data structures,
  if provided by the client, must be capable of holding this many entries.
  If \pgmid{colsze} is insufficient to return the answer, \dylp will
  reallocate the data structures.

  \item\varhdr{consys}
  The constraint system, in the format described for the \consys constraint
  system subroutine library \cite{Haf98b}.

  \item\varhdr{ctlopts}
  A vector of flags used to specify optional actions and status.
  The current set of flags can be used to control allocation and deallocation
  of internal \dylp data structures
  (\pgmid{lpctlDYVALID}, \pgmid{lpctlNOFREE}, \pgmid{lpctlONLYFREE}),
  specify the presence of changes to the problem bounds
  (\pgmid{lpctlUBNDCHG}, \pgmid{lpctlLBNDCHG}, \pgmid{lpctlRHSCHG})
  and objective (\pgmid{lpctlOBJCHG}),
  specify initial variable and/or constraint activation
  (\pgmid{lpctlINITACTVAR}, \pgmid{lpctlINITACTCON}),
  and specify the exchange of active variable information
  (\pgmid{lpctlACTVARSIN}, \pgmid{lpctlACTVARSOUT}).

  \item\varhdr{iters}
  The total number of simplex iterations.

  \item\varhdr{lpret}
  The return code from the simplex routine.
  
  If no errors occur, the code should be one of \pgmid{lpOPTIMAL} (optimal),
  \pgmid{lpINFEAS} (primal infeasible), or \pgmid{lpUNBOUNDED}
  (primal unbounded).

  Error returns include
  \pgmid{lpPUNT} (nonbasic variables exist with favourable reduced costs, but
  they cannot be pivoted due to unsuitable pivot coefficients),
  \pgmid{lpLOSTFEAS} (primal feasibility has been lost and \dylp has
  exceeded its limit on attempts to regain feasibility),
  \pgmid{lpSTALLED} (the limit on pivots without improvement in the objective
  has been exceeded, due to cycling or stalling),
  \pgmid{lpITERLIM} (a limit on pivots per phase or total pivots has been
  exceeded),
  \pgmid{lpACCCHK} (a numerical accuracy check has occurred),
  \pgmid{lpNOSPACE} (the \glpk basis routines could not acquire sufficient
  space to maintain the basis inverse),
  \pgmid{lpFATAL} (an unspecified fatal error has occurred), and
  \pgmid{lpINV} (\dylp aborted due to internal confusion).

  \item\varhdr{obj}
  For an optimal result, the value of the objective function.
  For an infeasible result, the total primal infeasibility.
  For an unbounded result, the index of the unbounded variable, negated
  if the variable can decrease without bound, positive if it can increase
  without bound.
  For any other return status, this field is undefined.

  \item\varhdr{phase}
  \begin{description}[\textbf{(o)}]
    \item[\textbf{(i)}]
    If the phase is set to \pgmid{dyDONE}, \dylp will assume that the only
    purpose of the call is to free internal data structures.
    Other values are ignored.
  
    \item[\textbf{(o)}]
    The termination phase of the dynamic simplex algorithm; should be
    \pgmid{dyDONE} unless an error has occurred, in which case it'll be
    \pgmid{dyINV}.
  \end{description}

\item\varhdr{rowsze}
  The allocated row capacity of the data structure.
  The \pgmid{basis}, \pgmid{x}, and \pgmid{y} data structures, if provided by
  the client, must be capable of holding this many entries.
  If \pgmid{rowsze} is insufficient to return the answer, \dylp will
  reallocate the data structures.

\item\varhdr{status}
  A data structure to hold the status of variables.
  For nonbasic variables, an entry is a \dylp status code
  (\pgmid{vstatNBFX}, \pgmid{vstatNBUB}, \pgmid{vstatNBLB}, or
   \pgmid{vstatNBFR}).
  For basic variables, an entry is the negative of the basis position.
  \begin{description}[\textbf{(o)}]
    \item[\textbf{(i)}]
    For a warm start, an initial status must be provided.
    For a cold or hot start, a structure can be provided to return the
    final status.

    \item[\textbf{(o)}]
    The final status vector.
    The value of nonbasic primal variables is returned through this
    vector.
    If no vector was supplied as an input parameter, \dylp will allocate
    one on output.
  \end{description}

  \item\varhdr{x} A data structure to hold the values of the basic primal
  variables.
  \begin{description}[\textbf{(o)}]
    \item[\textbf{(i)}]
    A structure can be provided to return the final values.

    \item[\textbf{(o)}]
    The values of the basic primal variables, indexed by basis position.
    If no vector was supplied as an input parameter, \dylp will allocate
    one on output.
  \end{description}

  \item\varhdr{y} A data structure to hold the values of the dual variables.
  \begin{description}[\textbf{(o)}]
    \item[\textbf{(i)}]
    A structure can be provided to return the final values.

    \item[\textbf{(o)}]
    The values of the dual variables, indexed by basis position.
    If no vector was supplied as an input parameter, \dylp will allocate
    one on output.
  \end{description}
\end{codedoc}


\subsection{\dylp Options}
\label{sec:DylpOptions}

\dylp is intended to be a flexible testbed, and as such has a large number of
options.
Many, in fact, have argued that it has entirely too many options.
The author offers two observations in his own defense:
\begin{itemize}
  \item
  All of them, at some point, were useful to him, and
  
  \item
  if you're not interested, ignore them all and
  let \dylp choose what it thinks are reasonable values.
\end{itemize}
If you look through the code, you may notice a few options that aren't
documented here.
By and large, this is because the best choice is clear and choices
other than the current default give uniformly poor performance.

Options are held internally in a \pgmid{lpopts_struct} structure.
Each field is described briefly below, including default values.
The reader is encouraged to consult \coderef{dylp.h}{} for details, and
\coderef{dy_setup.c}{} to confirm that default values have not changed since
this documentation was written.

Most options can be set using commands read from an options file.
This file is parsed by a simple command interpreter (contained
in \coderef{cmdint.c}{}) and support routines in \coderef{dy_setup.c}{} and
in the i/o library (\vid \secref{sec:IOandErrorMsgs}).
If your application has some other way to acquire options from the user,
all that's really necessary is a way to create and load
a \pgmid{lpopts_struct} to pass as a parameter to \pgmid{dylp}.
As described in \secref{sec:ParameterRoutines}, the routines
\pgmid{dy_defaults} and \pgmid{dy_checkdefaults} will, respectively, initialise
a \pgmid{lpopts_struct} with default values and adjust those values to match
the constraint system.

In the individual option descriptions which follow, the first line provides
the syntax expected by the simple command interpreter mentioned above.
information about acceptable values.
Where applicable, for simple numeric parameters, the next line gives
the lower bound, default value, and upper bound for the option in the
notation
$(\text{lower bound}) \leq (\text{default value}) \leq
 (\text{upper bound})$.
The remainder of the entry describes the action of the option.

\begin{codedoc}
  \item\Varhdr{active}{cons, vars}

  \bgroup \raggedright
  \kw{lpcontrol active}
    \bnflist[\raise2pt\hbox{\kw{,}}]{\nt{size-spec}} \kw{;} \\
  \nt{size-spec} \bnfeq
    \kw{variables} \te{float} | \kw{constraints} \te{float}
  \egroup

  $0.0 \leq .25 \leq 1.0$ for both

  The values \pgmid{active.vars} and \pgmid{active.cons} specify the fraction
  of variables and constraints, respectively, which are expected to be active
  at any one time.
  The initial allocated capacity of the active constraint system data structure
  will be the specified fraction of the number of variables and constraints
  in the constraint system passed to \pgmid{dylp}.
  They do not represent limits --- the constraint system will be expanded as
  required.
  They are exposed for efficiency in the event that the client can provide
  a better estimate for the expected size of the active constraint system.

  Note that specifying $\pgmid{active.vars} = 1.0$ and
  $\pgmid{active.cons} = 1.0$ is \textit{not} the same as specifying
  that \dylp use the full constraint system (\cf \pgmid{fullsys}).
  The data structure for the active constraint system will be created with
  the capacity to hold the full constraint system, but constraint and variable
  activation and deactivation will proceed as usual.


  \item\varhdr{addvar}
  \kw{lpcontrol actvarlim} \te{integer} \kw{;}

  Limits the maximum number of variables which can be activated in any
  one execution of the variable activation phase.
  A value of 0 (the default) means that no limit is enforced.

  \item\varhdr{check}
  \kw{lpcontrol check} \te{integer} \kw{;}

  $1 \leq \pgmid{factor}/2 \leq \infty$

  The nominal interval between accuracy checks, expressed in terms of the
  number of pivots which actually change the basis.

  Accuracy checks attempt to detect the accumulation of numerical inaccuracy,
  and \dylp will perform a check earlier if it suspects numerical problems.
  While there's no enforced upper limit on the number of pivots between
  accuracy checks, in practice an accuracy check is performed each time the
  basis is factored during simplex phases.

  \item\varhdr{coldvars}
  \kw{lpcontrol coldvars} \te{integer} \kw{;}

  $0 \leq 5000 \leq 100000$.

  When the number of active variables in the constraint system on a cold start
  exceeds \pgmid{coldvars}, and the client has not requested that \dylp work
  with the full constraint system, \dylp will attempt to deactivate
  variables before beginning simplex iterations.

  The upper limit is soft; \dylp will issue a warning if a higher value is
  requested, but will not enforce the limit.

  \item\varhdr{con}{actlvl, actlim, deactlvl}
  \begin{itemize}
    \item[\pgmid{con.actlvl}]
    \kw{lpcontrol actconlvl} \te{integer} \kw{;}

    Specifies the constraint activation strategy.
    There are two levels:
    \begin{description}
      \item[0 (strict)] Activate only constraints which are strictly
	violated.

      \item[1 (tight)] Activate constraints which are tight or strictly
	violated.
    \end{description}

    \item[\pgmid{con.actlim}]
    \kw{lpcontrol actconlim} \te{integer} \kw{;}

    Limits the maximum number of constraints which can be activated in any
    one execution of the constraint activation phase.
    A value of 0 (the default) means that no limit is enforced.

    \item[\pgmid{con.deactlvl}]
    \kw{lpcontrol deactconlvl}
	[\kw{normal}|\kw{aggressive}|\kw{fanatic}] \kw{;}

    Specifies the constraint deactivation strategy.
    There are three levels:
    \begin{description}
      \item[0 (\kw{normal})] Deactivate only inequalities which are strictly
	loose (\ie, the associated slack is basic and not at bound).

      \item[1 (\kw{aggressive})] (default) Deactivate loose inequalities and
	tight inequalities whose associated dual variable is zero.

      \item[2 (\kw{fanatic})] Deactivate loose inequalities and any
	tight constraint (inequality or equality) whose associated dual
	variable is zero.
    \end{description}

  \end{itemize}

  \item\varhdr{copyorigsys}
  \kw{lpcontrol forcecopy} \te{boolean} \kw{;}

  If set to true, \dylp will always make a local copy of the original system.
  By default, a local copy is made only when necessary.

  \dylp needs access to a copy of the original constraint system in order to
  scan it for constraints or variables that should be added.
  Normally this access is read-only, and \dylp uses the constraint system
  supplied as a parameter.
  When scaling is needed, \dylp makes a local copy of the original constraint
  system, applies scaling, and uses the scaled local copy as the original
  constraint system.

  \item\varhdr{degen}
  \kw{lpcontrol antidegen} \te{boolean} \kw{;}

  If set to false, \dylp will not use the perturbation-based anti-degeneracy
  algorithm described in \secref{sec:PerturbedAntiDegeneracy}.
  The default is to use perturbation-based anti-degeneracy.

  \item\varhdr{degenlite}
  \bgroup\raggedright
    \kw{lpcontrol degenlite} \\
    \hfil
    [\kw{pivotabort}|\kw{pivot}|\kw{alignobj}|
     \kw{alignedge}|\kw{perpobj}|\kw{perpedge}] \kw{;}
  \egroup

  This option specifies the tie-breaking strategy used for choosing
  between candidates with equal deltas when selecting the
  leaving primal or dual variable, as described in~\secref{sec:AntiDegenLite}.
  The options are:
  \begin{description}[0 (\kw{pivotabort})]
    \item[0 (\kw{pivotabort})]
    Break ties using the magnitude of the pivot coefficient, and abort the
    search at the first basic variable which gives a delta of zero.

    \item[1 (\kw{pivot})] (default)
    Break ties using the magnitude of the pivot coefficient, scanning all
    basic variables.

    \item[2 (\kw{alignobj})]
    Break ties by choosing the leaving variable which will make tight the
    hyperplane most closely aligned with the normal of the objective
    function (\ie, the normal most nearly lies in the hyperplane).

    \item[3 (\kw{alignedge})]
    Break ties by choosing the leaving variable which will make tight the
    hyperplane most closely aligned with the direction of motion specified
    by the entering variable (\ie, the edge most nearly lies in the
    hyperplane).

    \item[4 (\kw{perpobj})]
    Break ties by choosing the leaving variable which will make tight the
    hyperplane most nearly perpendicular to the normal of the objective
    function (\ie, the hyperplane most nearly blocks motion in the direction
    of the normal of the objective)

    \item[5 (\kw{perpedge})]
    Break ties by choosing the leaving variable which will make tight the
    hyperplane most nearly perpendicular to the direction of motion specified
    by the entering variable (\ie, the hyperplane most nearly blocks motion
    in the direction of the edge).
  \end{description}


  \item\varhdr{degenpivlim}
  \kw{lpcontrol degenpivs} \te{boolean} \kw{;}

  $1 \leq 1 \leq \infty$

  Limits the number of consecutive degenerate pivots which are
  required to trigger the perturbation-based anti-degeneracy algorithm.
  A perturbed subproblem is formed when the number of consecutive degenerate
  pivots exceeds \pgmid{degenpivlim}.
  The current default of 1 is very aggressive.

  \item\Varhdr{dpsel}{strat, flex, allownopiv}

  \kw{lpcontrol dualmultipiv} \te{integer} \kw{;}

  There are four dual pivoting strategies accessible from the
  \kw{dualmultipiv} command, specified by the following integer codes:
  \begin{description}
    \item[0] standard dual pivoting
	     (\vid \secref{sec:DualStdSelectInVar})

    \item[1] generalised dual pivoting
	     (\vid \secref{sec:DualGenSelectInVar});
	     pivot chosen for maximum dual objective improvement

    \item[2] generalised dual pivoting; pivot chosen to mimimise the maximum
	     infeasibility over primal variables

    \item[3] generalised dual pivoting; pivot chosen to minimise the maximum
	     infeasibility over primal variables only if the infeasibility
	     can be reduced; otherwise the pivot is chosen for maximum
	     dual objective improvement
  \end{description}
  The pivoting strategy currently in use is held in \pgmid{dpsel.strat}.

  Two additional values are used to control generalised dual pivoting; these
  can only be changed under program control.
  \pgmid{dpsel.flex} defaults to true, allowing \dylp to move between
  strategies~1 and~3.
  If the client specifies a pivoting strategy using the \kw{dualmultipiv}
  command, \pgmid{dpsel.flex} is set to false.
  \pgmid{dpsel.allownopiv} controls whether \dylp will consider a generalised
  dual `pivot' which consists of a sequence of variable flips without a final
  pivot.
  Computational experience says that this is very prone to cycling and
  \pgmid{dpsel.allownopiv} is set to false by default.

  The default initial setting for the dual pivoting options is
  $\pgmid{dpsel.strat} = 1$, $\pgmid{dpsel.flex} = \pgmid{true}$, and
  $\pgmid{dpsel.allownopiv} = \pgmid{false}$.


  \item\varhdr{dualadd}
  \kw{lpcontrol dualacttype} \te{integer} \kw{;}

  This option controls the amount of effort that \dylp will expend attempting
  to add variables (dual constraints) to bound a constraint system which
  is dual unbounded (\vid \secref{sec:VariableActivation}).
  \begin{itemize}
    \item[0]
    Variable activation is not attempted.

    \item[1]
    Type~1 variables are activated.
    These are variables which could potentially bound the dual problem and
    which will be dual feasible if activated and placed in the nonbasic
    partition.
    Multiple variables of this type can be activated simultaneously.

    \item[2]
    Type~2 variables will be activated if there are no type~1 variables.
    Type~2 variables are variables which would be dual infeasible if placed
    in the nonbasic partition, but which can be activated and immediately
    pivoted into the basis to regain dual feasibility.
    Only one variable of this type can be activated at a time, so this
    level is computationally expensive.

    \item[3] (default)
    Type~3 variables will be activated if there are no type~1 or
    type~2 variables.
    Type~3 variables are variables which can be activated and placed in the
    nonbasic partition with a bound-to-bound pivot.
  \end{itemize}
  If the limits placed on dual variable activation do not allow the dual
  to be bounded \dylp will revert to primal simplex.
  Allowing up to type~3 activations by default is somewhat risky; limiting
  activations to type~1 would be a more conservative choice.

  \item\varhdr{factor}
  \kw{lpcontrol factor} \te{integer} \kw{;}

  $1 \leq 50 \leq 100$

  The nominal interval for refactoring the basis, in terms of the number of
  pivots which actually change the basis.

  Put another way, \pgmid{factor} limits the total number of eta matrices
  in the multiplicative representation of the basis.
  As eta matrices accumulate, the work required to perform multiplication by
  the basis inverse increases, numerical inaccuracy increases, and the data
  structure grows (\vid \secref{sec:GLPKBasisInit}).
  This parameter attempts to balance these considerations against the work
  required to refactor the basis.
  \dylp will refactor earlier if it suspects numerical problems.

  The upper limit is soft; \dylp will issue a warning if a higher value is
  requested, but will not enforce the limit.

  \item\varhdr{finpurge}{vars, cons}

  \bgroup \raggedright
  \kw{lpcontrol final purge} 
    \bnflist[\raise2pt\hbox{\kw{,}}]{\nt{purge-spec}} \kw{;} \\
  \nt{purge-spec} \bnfeq [ \kw{variables}|\kw{constraints}] \te{boolean}
  \egroup

  Specifies whether \dylp should perform a final round of constraint and/or
  variable deactivation when the problem has been solved to optimality.
  By default, \dylp will perform a final round of constraint deactivation and a
  final round of variable deactivation before it returns.

  This application of constraint and/or variable deactivation is \textit{not}
  suppressed by the \pgmid{fullsys} option.


  \item\varhdr{forcecold}
  \kw{lpcontrol cold} \te{boolean} \kw{;}

  When set to true, this option will force \dylp to perform a cold
  start.
  \pgmid{forcecold} dominates \pgmid{forcewarm}.
  The absence of \pgmid{forcecold} and \pgmid{forcewarm} allows a hot start.

  \item\varhdr{forcewarm}
  \kw{lpcontrol warm} \te{boolean} \kw{;}

  When set to \pgmid{true}, this option will force \dylp to perform a warm
  start.
  The absence of \pgmid{forcecold} and \pgmid{forcewarm} allows a hot start.

  \item\varhdr{fullsys}
  \kw{lpcontrol fullsys} \te{boolean} \kw{;}

  When set to true, \pgmid{fullsys} forces the use of the full constraint
  system at all times.
  \dylp will load the entire constraint system at startup and no constraint or
  variable activation or deactivation will be performed.

  In the context of a branch-and-bound MIP code, where the bulk of the LPs
  are reoptimisations
  from a known basis, the use of dynamic simplex can save considerable work.
  To solve an LP once from scratch, or to solve the initial LP relaxation in a
  branch-and-bound context, use of the full system is usually (but not always)
  more efficient.

  \item\varhdr{groom}
  \kw{lpcontrol groom} [\kw{silent}|\kw{warn}|\kw{abort}]  \kw{;}

  Specifies the action taken when \dylp detects a nontrivial change in the
  status of a variable when it performs a check following refactoring.
  The possible values are
  \begin{description}
    \item[0 (\kw{silent})] Do nothing.

    \item[1 (\kw{warn})] (default) Issue a warning message.

    \item[2 (\kw{abort})] Issue an error message and force an abort.
  \end{description}

  The working assumption is that refactoring the basis removed accumulated
  numerical inaccuracy, causing the change in the status of the variable.

  \item\Varhdr{heroics}{d2p, p2d}

  These parameters control whether \dylp will attempt difficult deactivations
  when trying to force a transition to dual or primal feasibility.
  \begin{description}
    \item[\pgmid{d2p}]
    If true, \dylp will attempt to deactivate primal infeasible basic
    architectural variables when trying to force primal feasibility.

    \item[\pgmid{p2d}]
    If true, \dylp will attempt to deactivate tight constraints (\ie, nonbasic
    logicals) when trying to force dual feasibility.
  \end{description}
  Both of these default to false.
  Computational experience says that setting them to true is not useful.
  They can be adjusted only under program control.

  \item\varhdr{idlelim}
  \kw{lpcontrol idle} \te{integer} \kw{;}

  $0 \leq 1000 \leq 2*(\pgmid{concnt}+\pgmid{archvcnt}) \le 50000
    \leq 2^{\pgmid{sizeof(int)}-3}$

  The limit on the number of pivots allowed without an improvement in the
  value of the objective function.

  A pivot in which the change in the objective function value is less than
  \pgmid{dy_tols.dchk} is defined to be an idle pivot.
  Too many consecutive idle pivots are taken as an indication that the
  LP has stalled and may be cycling.
  If the number of pivots without change in the
  objective exceeds \pgmid{idlelim}, \dylp aborts and returns
  \pgmid{lpSTALLED}.
  Left to its own devices, \dylp will enforce the inner limits of
  $1000 \leq \pgmid{idlelim} \leq 50000$; the client can explicitly specify
  any value within the outer limits.

  \item\varhdr{initbasis}
  \kw{lpcontrol coldbasis}
      [\kw{slack}|\kw{logical}|\kw{architectural}] \kw{;}

  This parameter specifies the type of initial basis constructed for a cold
  start, as described in \secref{sec:ColdStart}.
  \begin{description}[2 (\kw{architectural})]
    \item[1 (\kw{logical})]
    (default) Prefer slack, then artificial, variables for basic variables.
    Architectural variables will not be used.

    \item[1 (\kw{slack})] Prefer slack, then architectural, variables for
    basic variables.
    Artificial variables will be used if absolutely necessary.

    \item[2 (\kw{architectural})] Prefer architectural, then slack, variables
    for basic variables.
    Artificial variables will be used if absolutely necessary.
  \end{description}

  \item\Varhdr{initcons}{frac, i1lopen, i1l, i1uopen, i1u,
			 i2lopen, i2l, i2uopen, i2u}
  \bgroup \raggedright
  \kw{lpcontrol load} [\nt{load-fraction}] 
    \bnflist[\raise2pt\hbox{\kw{,}}]{\nt{interval}} \kw{;} \\
  \nt{load-fraction} \bnfeq \te{float} \\
  \nt{interval} \bnfeq \nt{open-delim} \nt{ub} \nt{lb} \nt{close-delim} \\
  \nt{ub} \bnfeq \te{float} \\
  \nt{lb} \bnfeq \te{float} \\
  \nt{open-delim} \bnfeq \kw{(} | \kw{[} \\
  \nt{close-delim} \bnfeq \kw{)} | \kw{]}
  \egroup

  These parameters control the loading of a partial constraint system during
  a cold start.
  As described in \secref{sec:ColdStart}, constraints are ranked by the angle
  formed by the constraint normal and the objective normal, and a specified
  fraction of one or two angular intervals is loaded.

  The parameter \pgmid{frac} specifies what fraction of the inequalities
  in the specified intervals will be loaded.
  The parameters \pgmid{i1l} and \pgmid{i1u} specify the upper and lower bounds
  of one interval.
  If \pgmid{i1lopen} is true, the lower boundary is open; if \pgmid{i1uopen} is
  true, the upper boundary is open.
  The parameters \pgmid{i2l}, \pgmid{i2u}, \pgmid{i2lopen}, and \pgmid{i2uopen}
  can be used to specify an optional second interval.

  A few examples will make the usage clear.
  By default, \dylp loads 50\% of all inequalities, with the exception of
  inequalities which form an angle of $\degs{90}$ with the objective.
  This is specified as
  \begin{flushleft}
  \kw{lpcontrol load .5 [180 90) (90 0] ;}
  \end{flushleft}
  To load 75\% of the inequalities with angles between $\degs{100}$ and
  $\degs{80}$, inclusive, the specification would be
  \begin{flushleft}
  \kw{lpcontrol load .75 [100 80] ;}
  \end{flushleft}
  Loading the complete constraint system with the specification
  \begin{flushleft}
  \kw{lpcontrol load 1.0 [180 0] ;}
  \end{flushleft}
  is \textit{not} equivalent to asking \dylp to always use the full constraint
  system (\cf \pgmid{fullsys}).
  It will look pretty much the same from the outside, but \dylp will
  spend time internally
  performing scans related to constraint and variable activation and
  deactivation.


  \item\varhdr{iterlim}
  \kw{lpcontrol iters} \te{integer} \kw{;}

  $0 \leq 10000 \leq 5*(\pgmid{concnt}+\pgmid{archvcnt}) \le 100000
    \leq 2^{\pgmid{sizeof(int)}-3}$

  The pivot limit for each occurrence of a simplex phase
  (primal phases~I and II and dual phase~II).
  The overall pivot limit, cumulative over all occurrences of all phases,
  is $3*\pgmid{iterlim}$.
  If either the per phase or total limit is exceeded, \dylp terminates the
  problem and returns \pgmid{lpITERLIM}.
  Left to its own devices, \dylp will enforce the inner limits of
  $10000 \leq \pgmid{iterlim} \leq 100000$; the client can explicitly specify
  any value within the outer limits.

  \item\varhdr{patch}
  \kw{lpcontrol patch} \te{boolean} \kw{;}

  If set to false, \dylp is forbidden from patching a singular basis.
  By default, \dylp will patch a singular basis and keep going.
  You really don't want to set this to false.

  \item\varhdr{ppsel}
  \kw{lpcontrol primmultipiv} \te{integer} \kw{;}

  There are two primal pivoting strategies accessible from the
  \kw{primmultipiv} command, specified by the following integer codes:
  \begin{description}
    \item[0] standard primal pivoting
	     (\vid \secref{sec:PrimalStdSelectOutVar})

    \item[1] (default) extended primal pivoting
	     (\vid \secref{sec:PrimalGenSelectOutVar})
  \end{description}
  The pivoting strategy currently in use is held in \pgmid{ppsel.strat}.

  \item\varhdr{print}
  \bgroup \raggedright
  \kw{lpprint} \nt{what} \te{integer} \kw{;} \\
  \nt{what} \bnfeq \kw{basis}|\kw{conmgmt}|\kw{crash}|\kw{degen}|\kw{dual}|
      \kw{major}|\kw{phase1}|\kw{phase2}|\kw{pivoting}| \\
      \hspace{8ex}
      \kw{pivreject}|\kw{pricing}|\kw{scaling}|\kw{setup}|\kw{varmgmt}
  \egroup

  The print options control the amount of output which \dylp produces as it
  runs.
  This can be varied from absolutely nothing to copious output useful only
  during detailed debugging.
  Printing options are covered in detail in \secref{sec:DylpDebugging}, which
  describes debugging options and capabilities.
  If \dylp is compiled with the compile-time constant \pgmid{NDEBUG} defined,
  virtually all informational printing is removed.

  \item\varhdr{scaling}
  \kw{lpcontrol scaling} \te{integer} \kw{;}

  Specifies how \dylp should scale the constraint system (\secref{sec:Scaling}).
  \begin{itemize}
    \item[0] \dylp is not allowed to apply scaling.

    \item[1] \dylp should use scaling vectors attached to the constraint
	     system.

    \item[2] (default) \dylp should evaluate the constraint system and apply
	     scaling if necessary.
  \end{itemize}

  \item\varhdr{scan}
  \kw{lpcontrol scan} \te{integer} \kw{;}

  $200 \le \pgmid{archvcnt}/2 \le 1000$.

  Specifies the minimum number of columns which will be scanned in primal
  simplex to select a new candidate entering variable.
  This parameter applies only when \pgmid{dy_primalin} is called to select
  the entering variable (\vid \secref{sec:PrimalStdSelectInVar}).

  \item\varhdr{usedual}
  \kw{lpcontrol usedual} \te{boolean} \kw{;}

  When set to false, this option prevents \dylp from using dual simplex.
  By default, \dylp will use dual simplex when possible.
\end{codedoc}

\subsection{\dylp Tolerances}
\label{sec:DylpTolerances}

\dylp has a number of numeric tolerances and related control information
which are used in equality and accuracy checks and associated algorithms
which attempt to control the accumulation of numerical accuracy.
Each is described briefly below; again, the reader is encouraged to consult
\coderef{dylp.h}{} for details.

Several of the tolerances described below are dynamically adjusted by \dylp
in response to its assessment of the numerical stability of the current
basis.
As a general rule, tread carefully when overriding \dylp's defaults, and
please take the time to read the code comments and consider the
interrelationships between the tolerances.

\begin{codedoc}
  \item\varhdr{bogus}
  \kw{lpcontrol bogus} \te{double} \kw{;}

  Default: 1.0

  The `bogus number' tolerance.
  Values such that $\pgmid{zero} < \abs{x} \le \pgmid{zero}*\pgmid{bogus}$
  are considered likely to be the result of accumulated numerical inaccuracy,
  rather than legitimate values.
  Pivot coefficients and primal variable values within this range
  will trigger refactoring of the basis.
  For dual variables, the same test is applied, using the dual zero
  tolerance (\pgmid{cost}).
  The default value is $1.0$.

  Experience seems to show that for the majority of problems
  increasing this value will cause the basis to be refactored more
  often and will not improve performance or accuracy.
  It's better to rely on \dylp's accuracy checks to determine if the basis
  should be refactored before the normal refactor interval has passed.
  Increasing \pgmid{bogus} may be useful if scaling is disabled,
  or if \pgmid{factor} has been set to a very large value.

  \item\varhdr{cost}
  \kw{lpcontrol costz} \te{double} \kw{;}

  Default: $1.0\times10^{-11}$

  The zero tolerance applied to values associated with the dual problem
  (dual variables and reduced costs).

  This tolerance may be tightened if \dylp scales the constraint system for
  numerical stability.
  Let $\psi = ((\max_{ij} \abs{a_{ij}})/(\min_{ij} \abs{a_{ij}}))^{1/2}$.
  Let $\psi_u$ be the value calculated for the unscaled matrix $A$ and
  $\psi_s$ be the value calculated for the scaled matrix $\breve{A}$.
  Let $s = \max (0, \floor{\log \psi_u/\psi_s + .5}-2$.
  The dual zero tolerance will be tightened by $10^{-s}$
  (\ie, $\pgmid{cost} = \pgmid{cost} \times 10^{-s}$).
  In english, if scaling really did make a
  difference, so that the scaled matrix is significantly more stable than the
  unscaled matrix, \dylp should be extra careful about accuracy so that the
  scaled solution is still a solution after unscaling.

  \item\varhdr{dchk}
  \kw{lpcontrol dchk} \te{double} \kw{;}

  Default: $1.0 \times 10^{s-4}$,
      where $s = \max (0, \floor{\log \pgmid{archccnt} + .5}-2$

  The dual accuracy check tolerance, as described in
  \secref{sec:AccuracyChecks}.
  The adjustment by $s$ progressively loosens the accuracy check tolerance
  for systems with more than $10^{2.5} \approx 300$ dual variables.
  In english, when there are many dual variables, accumulating numerical
  inaccuracy warrants some relaxation of the accuracy check tolerance.
  This adjustment is made in \coderef{}{dy_checkdefaults}.

  \item\varhdr{dfeas}

  The dual feasibility check tolerance, dynamically calculated using
  \pgmid{cost} as the base value, as described in \secref{sec:AccuracyChecks}.

  \item\varhdr{dfeas_scale}
  \kw{lpcontrol dfeas} \te{double} \kw{;}

  Default: $1.0 \times 10^{s+2}$, 
      where $s = \max (0, \floor{\log \pgmid{archccnt} + .5}-2$

  Decoupling multiplier for scaling \pgmid{dfeas}.
  This multiplier may be increased if the constraint system contains many
  dual variables or if the constraint system is scaled.

  The adjustment for a large number of dual variables is the same
  adjustment applied for \pgmid{dchk}.

  The adjustment for matrix scaling follows the adjustment described for
  \pgmid{cost}.
  Using the definitions for $\psi_u$ and $\psi_s$ given for \pgmid{cost},
  $s = \max (0, \floor{\log \psi_u/\psi_s + .5}-1$
  and \pgmid{dfeas_scale} will be increased by $10^s$.
  In english, the separation between the dual zero tolerance and the dual
  feasibility tolerance is increased to compensate for tightening the
  dual zero tolerance.

  \item\varhdr{inf}
  \kw{lpcontrol infinity} [\kw{IEEE}|\kw{DBL\_MAX}|\te{double}] \kw{;}

  Infinity.
  \dylp can work with an infinite or finite infinity.

  Default: \pgmid{HUGE_VAL}
  
  \pgmid{HUGE_VAL} will be IEEE 754 infinity on most modern systems.

  Many numerical programs still use that mathematical oxymoron, a finite
  infinity.
  Most commonly, this will be the value defined for the ANSI C symbol
  \coderef{float.h}{DBL_MAX}, the maximum representable value for type
  \pgmid{double}.
  Finite and infinite infinity do not play well together.
  If \dylp is being used by a client program which uses a finite infinity,
  set \pgmid{inf} to the client's value of infinity.

  \item\varhdr{pchk}
  \kw{lpcontrol pchk} \te{double} \kw{;}

  Default: $1.0 \times 10^{s-5}$,
      where $s = \max (0, \floor{\log \pgmid{archvcnt} + .5}-2$

  The primal accuracy check tolerance, as described in
  \secref{sec:AccuracyChecks}.
  The adjustment by $s$ progressively loosens the accuracy check tolerance
  for systems with more than $10^{2.5} \approx 300$ variables.
  In english, when there are many variables, accumulating numerical inaccuracy
  warrants some relaxation of the accuracy check tolerance.
  This adjustment is made in \coderef{}{dy_checkdefaults}.

  \item\varhdr{pfeas}

  The primal feasibility check tolerance, dynamically calculated using
  \pgmid{zero} as the base value, as described in \secref{sec:AccuracyChecks}.

  \item\varhdr{pfeas_scale}
  \kw{lpcontrol pfeas} \te{double} \kw{;}

  Default: $1.0 \times 10^{s+2}$, 
      where $s = \max (0, \floor{\log \pgmid{archvcnt} + .5}-2$

  A decoupling multiplier used to adjust the separation of \pgmid{pfeas}
  and \pgmid{zero} as described in \secref{sec:AccuracyChecks}.
  This multiplier may be increased if the constraint system contains many
  variables or if the constraint system is scaled.

  The adjustment for a large number of variables, specified with the default
  value, is the same adjustment applied for \pgmid{pchk}.
  In english, when there are many variables, accumulating numerical inaccuracy
  warrants some relaxation of the feasibility tolerance.

  The adjustment for matrix scaling follows the adjustment described for
  \pgmid{zero}.
  Using the definitions for $\psi_u$ and $\psi_s$ given for \pgmid{zero},
  $s = \max (0, \floor{\log \psi_u/\psi_s + .5}-1$
  and \pgmid{pfeas_scale} will be increased by $10^s$.
  In english, the separation between the zero tolerance and the feasibility
  tolerance is increased to compensate for tightening the zero tolerance.

  \item\varhdr{pivot}
  \kw{lpcontrol pivot} \te{double} \kw{;}

  Default: $1.0 \times 10^-5$

  The pivot selection multiplier.
  A pivot coefficient $\overline{a}_{ij}$ will be accepted as
  numerically stable in the primal algorithm if
  $\abs{\overline{a}_{ij}} \ge
    (\pgmid{pivot})(\pgmid{piv_tol})\norm[1]{\overline{a}_j}$,
  where \pgmid{piv_tol} is the stable pivot tolerance used during factoring
  in \glpk.
  In the dual algorithm, the 1-norm is calculated over the pivot row
  $\overline{a}_i$.

  The pivot selection multiplier may be reduced if \dylp finds itself at an
  extreme point where all potential pivots $x_i$, $x_j$ have been rejected
  because the pivot coefficients $\overline{a}_{ij}$ were judged numerically
  unstable (\vid \secref{sec:ErrorRecovery}).

  In english, if \pgmid{pivot} were set to 1, the pivot coefficient
  $\overline{a}_{ij}$ for every simplex pivot would have to satisfy the same
  stability criterion that the \glpk basis package applies when factoring
  the basis.
  This would be overly restrictive, however --- when executing simplex pivots,
  \dylp needs to choose the pivot row and column to maximise progress toward
  an optimal extreme point.
  Some compromise is necessary; the value of \pgmid{pivot} controls the
  balance between numerical stability and progress toward an optimal solution.
  When \dylp finds itself in a difficult spot, it will tilt the balance in
  order to make progress toward optimality.

  \item\varhdr{purge}
  \kw{lpcontrol purgecon} \te{double} \kw{;}

  Default: $1.0 \times 10^{-4}$

  The required percentage change in the value of the objective function before
  constraint or variable deactivation is allowed.
  This should be strictly greater than zero in order to minimise the
  possibility of a cycle involving activation/deactivation of constraints or
  variables.

  \item\varhdr{purgevar}
  \kw{lpcontrol purgevar} \te{double} \kw{;}

  Default: .5

  Used to calculate the variable deactivation threshold as a percentage of the
  maximum unfavourable reduced costs, as described in
  \secref{sec:VariableDeactivation}.

  \item\varhdr{reframe}
  \kw{lpcontrol reframe} \te{double} \kw{;}

  Default: .1

  The percentage error in the updated column or row norms which is required
  to trigger a reset of the PSE reference frame or the DSE row norms,
  respectively.
  A relatively large error can be tolerated here.
  The consequence of inaccuracy, a chance of a suboptimal choice of primal
  entering or dual leaving variable, is not too serious.
  In contrast, for the dual the computational cost of recalculating the
  basis inverse row norms $\norm{\beta_k}$ is high.
  For the primal, all column norms are reset to 1, effectively reverting to
  unscaled (`Dantzig') pricing.

  \item\varhdr{swing}
  \kw{lpcontrol swing} \te{double} \kw{;}

  Default: $1.0 \times 10^{15}$

  This tolerance is used to detect excessive change in the values of the primal
  variables.
  The magnitude of the value prior to a pivot is compared to the magnitude
  after the pivot.
  If the  ratio exceeds the value of \pgmid{swing}, the simplex phase will
  abort and \dylp will attempt to bound the primal swing (\vid
  \secref{sec:ErrorRecovery}).

  \item\varhdr{toobig}

  Default: $1.0\times 10^{30}$.

  This value is used to control changes in the dual multipivot strategy.
  The breakpoints are currently hardcoded in
  \coderef{dy_dualmultipivot}{dualmultiin} (which see).

  \item\varhdr{zero}
  \kw{lpcontrol zero} \te{double} \kw{;}

  Default: $1.0\times 10^{-11}$.

  The zero tolerance.
  Values smaller than $\abs{\pgmid{zero}}$ are set to a clean floating-point
  zero.

  This tolerance may be tightened if \dylp scales the constraint matrix for
  numerical stability.
  Let $\psi = ((\max_{ij} \abs{a_{ij}})/(\min_{ij} \abs{a_{ij}}))^{1/2}$.
  Let $\psi_u$ be the value calculated for the unscaled matrix $A$ and
  $\psi_s$ be the value calculated for the scaled matrix $\breve{A}$.
  Let $s = \max (0, \floor{\log \psi_u/\psi_s + .5}-2$.
  The zero tolerance will be tightened by $10^{-s}$
  (\ie, $\pgmid{zero} = \pgmid{zero} \times 10^{-s}$).
  In english, if scaling really did make a
  difference, so that the scaled matrix is significantly more stable than the
  unscaled matrix, \dylp should be extra careful about accuracy so that the
  scaled solution is still a solution after unscaling.

\end{codedoc}


\section{\dylp Statistics}
\label{DylpStatistics}

\dylp will collect detailed statistics if the conditional compilation
symbol \pgmid{DYLP_STATISTICS} is defined.
The available statistics are described briefly in the paragraphs which
follow; for details on subfields, consult \coderef{dylp.h}{}.
Routines in the file \coderef{statistics.c}{} provide initialisation
(\pgmid{dy_initstats}), printing (\pgmid{dy_dumpstats}), and release of the
data structure (\pgmid{dy_freestats}).

\begin{codedoc}
  \item
  \Varhdr{angle}{max, min, hist}
  Statistics on the angles of inequality constraints to the objective function.
  For constraint $i$, this is calculated as
  $\displaystyle \frac{180}{\pi} \cos^{-1} \frac{a_i c}{\norm{a_i}\norm{c}}$.
  The maximum and minimum angle is recorded, and a histogram in
  $\degs{5}$ increments with a dedicated $\degs{90}$ bin.

  \item
  \Varhdr{cons}{sze, angle, actcnt, deactcnt, init, fin}
  Information about individual constraints: the angle of the constraint with
  the objective function, the number of times it's activated and deactivated,
  and booleans to indicate if the constraint is active in the initial and final
  active systems.

  \item\Varhdr{d2}{pivs, iters}
  Total pivot and iteration counts for \dylp.
  The pivot count is the number of successful simplex pivots.
  The iteration count also includes pivot attempts which did not succeed
  for some reason
  (\eg, a primal pivot in which the entering variable was eventually rejected
  because the pivot element was numerically unstable).

  \item\Varhdr{ddegen}{cnt, avgsiz, maxsiz, totpivs, avgpivs, maxpivs}
  Statistics on the amount of time spent in restricted subproblems trying to
  escape dual degeneracy.

  For each level (\ie, each nested level of restricted subproblem), \dylp
  records the number of times this level was reached, the average and maximum
  number of variables involved in a degeneracy, the total and average
  number of pivots executed at this level, and the maximum number of pivots
  executed in any one subproblem at this level.
  The array is generously sized (by compile time constant) to accommodate
  a maximum of 25 levels.

  \item\Varhdr{dmulti}%
	      {flippable, cnt, cands, promote, nontrivial, evals, flips,
	       pivrnks, maxrnk}
  Statistics on the behaviour of the generalised dual pivoting algorithm.
  Each call to \pgmid{dualmultiin} collects a list of candidate variables to
  enter the basis and sorts the list.
  This process may produce a unique candidate for entry, or it may leave a list
  of requiring further evaluation to determine the best sequence of flips and
  final pivot.

  The \pgmid{flippable} field records the number of flippable variables in the
  problem (\ie, variables with finite lower and upper bounds).
  The \pgmid{cnt} field records the total number of calls to
  \pgmid{dualmultiin}, and \pgmid{nontrivial} records the number of times
  the initial scan and sort did not identify a unique entering variable.

  The remaining fields, with one exception, are totals.
  They record the number of candidates queued for evaluation,
  the number of times that a sane pivot was promoted over an unstable pivot,
  the number of columns transformed ($B^{-1}a_k$) for evaluation,
  the number of bound-to-bound flips,
  the rank in the sorted list of the variable selected to
  enter, and the maximum rank for a variable selected to enter.


  \item\Varhdr{factor}{cnt, prevpiv, avgpivs, maxpivs}
  Statistics about basis factoring.
  The \pgmid{cnt} field records the total number of times the basis was
  refactored.
  The \pgmid{avgpivs} and \pgmid{maxpivs} fields record the average and maximum
  number of pivots between basis refactoring.

  \item\Varhdr{infeas}{prevpiv, maxcnt, totpivs, maxpivs, chgcnt1, chgcnt2}
  Statistics on the resolution of infeasibility during primal phase I.

  The maximum number of infeasible variables is recorded, as well as the total
  pivots in phase I and the maximum number of pivots with no change in the
  number of infeasible variables.
  \dylp also counts the number of times that the number of infeasible variables
  changed without requiring recalculation of the reduced costs
  (\pgmid{chgcnt1}), and the number of times when it did (\pgmid{chgcnt2}).
  Specifically, if exactly one variable gains feasibility, and it leaves the
  basis as it does so, the reduced costs do not have to be recalculated.

  \item\Varhdr{p1}{pivs, iters}
  Total pivot and iteration counts for primal phase~1 simplex.

  \item\Varhdr{p2}{pivs, iters}
  Total pivot and iteration counts for primal phase~2 simplex.

  \item\Varhdr{pdegen}{cnt, avgsiz, maxsiz, totpivs, avgpivs, maxpivs}
  Statistics on the amount of time spent in restricted subproblems trying to
  escape primal degeneracy.
  The content of individual fields is as for \pgmid{ddgen}.

  \item\Varhdr{pivrej}%
	      {max, mad, sing, pivtol_red, min_pivtol, puntcall, puntret}
  Statistics on the management of variables judged unsuitable for pivoting.
  Variables are queued on the rejected pivot list when a pivot attempt fails
  because the pivot element is numerically unstable or because the pivot
  produced a singular basis.
  During primal simplex, candidate entering variables are queued; during dual
  simplex, candidate leaving variables.

  The \pgmid{max} field records the maximum length of the rejected pivot list.
  The fields \pgmid{mad} and \pgmid{singular} record the number of variables
  queued for unstable pivots and singular basis, respectively.

  The \pgmid{puntcall} field records the number of times the routine
  \pgmid{dy_dealWithPunt} was called in an attempt to remove variables from the
  rejected pivot list.
  The \pgmid{pivtol_red} field records the number of times that the pivot
  selection multiplier was reduced in order to consider candidate variables
  previously rejected for numeric instability; \pgmid{min_pivtol} is the
  minimum multiplier value used.
  The \pgmid{puntret} field records the number of times \pgmid{dy_dealWithPunt}
  was unable to remove any candidates from the rejected pivot list and
  therefore recommended termination of the current simplex phase.

  \item\Varhdr{pmulti}{cnt, cands, nontrivial, promote}
  Statistics on the behaviour of the extended primal pivoting algorithm.
  Each call to \pgmid{primalmultiout} collects a list of candidate variables to
  leave the basis.
  This process may produce a unique candidate to leave, or it may leave a list
  of candidates requiring further evaluation to determine the final pivot.

  The \pgmid{cnt} field records the total number of calls to
  \pgmid{primalmultiout}, and \pgmid{nontrivial} records the number of times
  the initial scan did not identify a unique leaving variable.
  The \pgmid{promote} field records the number of times
  that a sane pivot was promoted over an unstable pivot,

  \item\Varhdr{tot}{pivs, iters}
  Total pivot and iteration counts for the call to \pgmid{dylp}.

  \item\Varhdr{vars}{sze, actcnt, deactcnt}
  Information about individual variables: the number of times a variable is
  activated and deactivated.
\end{codedoc}


\section{\dylp Debugging Features}
\label{sec:DylpDebugging}

\dylp incorporates two types of debugging features: a controllable printing
facility and paranoid checks.
The printing facility is enabled when the symbol \pgmid{NDEBUG}
is not defined at compile time, and is intended to allow the generation of
log information at whatever level of detail is desired by the user.
The paranoid checks are enabled when the symbol \pgmid{PARANOIA} is defined
at compile time and are intended to provide significant (and expensive)
cross-checks during code development.

\subsection{Printing}

The amount of output generated by \dylp can be varied from next to nothing
to a level of detail intended only for detailed debugging.
The paragraphs which follow briefly outline the capabilities; for specific
output at a given print level, please refer to the file \coderef{dylp.h}{}.

\begin{codedoc}
  \item\varhdr{basis}
  Prints information related to management of the basis, including adjustments
  to suppress numerical instability and recover from singularity.

  \item\varhdr{conmgmt}
  Prints information on the management of constraints, including activation and
  deactivation, changes to primal and dual variables, and (at the highest level)
  a running commentary on all constraint and variable additions, deletions, and
  motions attributable to activation and deactivation of constraints.

  \item\varhdr{crash}
  Prints information regarding the generation of the initial basis,
  including factoring, the initial set of basic variables, and their values.
  For a cold start, information on the selection of the basic variables can be
  printed.

  \item\varhdr{degen}
  Prints information about degenerate pivots and restricted
  subproblem formation to deal with degeneracy.

  \item\varhdr{dual}
  Prints information about the execution of the dual simplex, with capabilities
  similar to \pgmid{phase1}.

  \item\varhdr{major}
  Tracks the major state transitions of the dynamic simplex algorithm
  as \dylp solves an LP.

  \item\varhdr{phase1}
  Prints information about the execution of phase I of the primal simplex.
  At the low end, messages are printed for extraordinary events ---
  unboundedness, serious pivoting problems, \etc
  At a medium level, a one line message is printed summarising each pivot, as
  well as messages about routine but infrequent events --- refactoring, accuracy
  checks, and various minor problems.
  At the highest level, all primal and dual variables are printed as they are
  recalculated for each pivot, along with detailed information about reduction
  of infeasibility and changes to the phase I objective function.
  This is \textit{an enormous amount} of output for large problems. 

  \item\varhdr{phase2}
  Prints information about the execution of phase II of the primal simplex,
  with capabilities similar to \pgmid{phase1}.

  \item\varhdr{pivoting}
  Prints information on the evaluation of candidates for the leaving primal
  or entering dual variable and details of the pivot column or row.
  At least one line per pivot; at the highest level, produces
  \textit{a lot} of output.

  \item\varhdr{pivreject}
  Prints information on the operation of \dylp's pivot rejection mechanism.

  \item\varhdr{pricing}
  Prints information regarding the pricing of candidates for the entering
  primal or leaving dual variable.
  At any level above 1 you'll get \textit{many} lines of output per pivot;
  that's \textit{an enormous amount} of output for large problems.

  \item\varhdr{scaling}
  Prints information regarding numerical scaling of the constraint system.

  \item\varhdr{setup}
  Prints information regarding the loading and initialisation of an LP problem,
  including the constraints and variables which are activated and the angle of
  inequalities to the objective function.

  \item\varhdr{varmgmt}
  Prints information on the activation and deactivation of variables, much as
  \pgmid{conmgmt}.
\end{codedoc}

\subsection{Paranoia}

Because it is intended as a development code, \dylp incorporates a large number
of sanity checks, enabled by defining the conditional compilation symbol
\pgmid{PARANOIA}.
Many of these tests are cheap and simple --- checks for null parameters,
sensible constraint and variable counts, proper major phase, and range checks
on indices.
Others are more elaborate and expensive.

There are two dedicated subroutines which are used at several points to check
the integrity of the current simplex point (basis, status, and primal
variable values) and the constraint system:
\begin{itemize}
\item
\pgmid{dy_chkstatus} implements extensive checks to make
sure that the status and value of a primal variable agree across multiple
data structures and are appropriate for the current major phase.

\item
\pgmid{dy_chkdysys} implements extensive checks to ensure that the
active constraint system and associated data structures are correct and
consistent.
\end{itemize}

There is another set of checks which track the numerical accuracy of
calculations by performing an independent calculation of a quantity.
These are of little use unless there is some reason to doubt the correctness
of the calculation, hence the separate conditional compilation symbols.
\begin{itemize}
  \item
  Checks on the accuracy of the calculations to produce unscaled rows of the
  basis inverse are controlled by the symbol \pgmid{CHECK_UNSCALED_BETAI}.

  \item
  Checks on the accuracy of iterative updating for PSE column norms and
  DSE row norms are controlled by the conditional compilation symbols
  \pgmid{CHECK_PSE_UPDATES} and \pgmid{CHECK_DSE_UPDATES}.
  These checks calculate the norms directly for comparison with
  the updated values, and the computational expense is
  unacceptable unless there is specific reason to suspect an error.
\end{itemize}



\bibliographystyle{louplain}
\bibliography{dylp}

\end{document}
